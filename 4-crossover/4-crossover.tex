\documentclass[nols,twoside,nofonts,nobib,nohyper]{tufte-handout}

\usepackage{fixltx2e}
\usepackage{tikz-cd}
\usepackage{tcolorbox}
\usepackage{appendix}
\usepackage{listings}
\lstset{language=TeX,
       frame=single,
       basicstyle=\ttfamily,
       captionpos=b,
       tabsize=4,
  }

\begin{acronym}
\acro{sfa}{Scopal Function Application}
\acro{fa}{Function Application}
\acro{wco}{Weak Crossover}
\acro{ScoT}{Scope Transparency}
\acro{vfs}{Variable Free Semantics}
\acro{acd}{Antecedent Contained Deletion}
\acro{qr}{Quantifier Raising}
\acro{doc}{Double Object Construction}
\acro{ccp}{Context Change Potential}
\acro{dmg}{Dynamic Montague Grammar}
\acro{dr}{Discourse Referent}
\acro{qp}{Quantificational Phrase}
\acro{dp}{Determiner Phrase}
\acro{dpp}{Dynamic Predication Principle}
\acro{ds}{Dynamic Semantics}
\acro{gq}{Generalized Quantifier}
\acro{npi}{Negative Polarity Item}
\acro{lf}{Logical Form}
\acro{pm}{Predicate Modification}
\acro{pfa}{Pointwise Function Application}
\acro{cg}{Common Ground}
\end{acronym}

\renewcommand*{\acsfont}[1]{\textsc{#1}}

\makeatletter
% Paragraph indentation and separation for normal text
\renewcommand{\@tufte@reset@par}{%
  \setlength{\RaggedRightParindent}{0pt}%
  \setlength{\JustifyingParindent}{0pt}%
  \setlength{\parindent}{0pt}%
  \setlength{\parskip}{\baselineskip}%
}
\@tufte@reset@par

% Paragraph indentation and separation for marginal text
\renewcommand{\@tufte@margin@par}{%
  \setlength{\RaggedRightParindent}{0pt}%
  \setlength{\JustifyingParindent}{0pt}%
  \setlength{\parindent}{0pt}%
  \setlength{\parskip}{\baselineskip}%
}
\makeatother

\setcounter{secnumdepth}{3}

\title{Crossover ii\thanks{24.979: Topics in
    semantics\\\noindent\textit{Getting high: Scope, projection, and evaluation order}}}

\author[Patrick D. Elliott and Martin Hackl]{Patrick~D. Elliott\sidenote{\texttt{pdell@mit.edu}} \& Martin Hackl\sidenote{\texttt{hackl@mit.edu}}}

\addbibresource[location=remote]{/home/patrl/repos/bibliography/elliott_mybib.bib}

\lingset{
  belowexskip=0pt,
  aboveglftskip=0pt,
  belowglpreambleskip=0pt,
  belowpreambleskip=0pt,
  interpartskip=0pt,
  extraglskip=0pt,
  Everyex={\parskip=0pt}
}

\usepackage{float}


% \usepackage{booktabs} % book-quality tables
% \usepackage{units}    % non-stacked fractions and better unit spacing
% \usepackage{lipsum}   % filler text
% \usepackage{fancyvrb} % extended verbatim environments
%   \fvset{fontsize=\normalsize}% default font size for fancy-verbatim environments

% % Standardize command font styles and environments
% \newcommand{\doccmd}[1]{\texttt{\textbackslash#1}}% command name -- adds backslash automatically
% \newcommand{\docopt}[1]{\ensuremath{\langle}\textrm{\textit{#1}}\ensuremath{\rangle}}% optional command argument
% \newcommand{\docarg}[1]{\textrm{\textit{#1}}}% (required) command argument
% \newcommand{\docenv}[1]{\textsf{#1}}% environment name
% \newcommand{\docpkg}[1]{\texttt{#1}}% package name
% \newcommand{\doccls}[1]{\texttt{#1}}% document class name
% \newcommand{\docclsopt}[1]{\texttt{#1}}% document class option name
% \newenvironment{docspec}{\begin{quote}\noindent}{\end{quote}}% command specification environment

\begin{document}

\maketitle% this prints the handout title, author, and date

\section{Dynamic Semantics}

In dynamic semantics (\citealt{heim1982}, \citealt{groenendijk_dynamic_1991})
sentences denote \textit{relations between assignments} (equivalently: functions
from assignments, to sets of assignments).



\begin{figure}
\caption{Relations between assignments}
\begin{forest}
  [{$g$} [{Roger$^{n}$ arrived late.} [{$g^{[n ↦ \ml{roger}]}$}]]]
\end{forest}
%
\begin{forest}
  [{$g$} [{A linguist$^{n}$ arrived late}
    [{$g^{[n ↦ \ml{kai}]}$}]
    [{$g^{[n ↦ \ml{roger}]}$}]
    [{$g^{[n ↦ \ml{sabine}]}$}]
    [{$g^{[n ↦ \ml{athulya}]}$}]
    [{$g^{[n ↦ \ml{martin}]}$}]
  ]]
\end{forest}
\end{figure}

In dynamic semantics then, sentences are of type $\type{(g,g') → t}$

\ex
$\eval{Roger$^n$ arrived late} = λ g . \set{g^{[n ↦ \ml{r}]} | \ml{arrived-late r}}$\hfill$\type{g → G t}$
\xe

\ex~
$\eval{A linguist$^{n}$ arrived late} = λ xgg' . g[n ↦x]g' ∧ \ml{arrived-late} x ∧ \ml{linguist} x}$\\
\phantom{,}\hfill$\type{g → G t}$
\xe

Dynamic closure:

\ex
$m^{↯} ≔ λ g . ∃g' ∈ (m g)$
\xe

In dynamic semantics, the connectives manipulate dynamic values directly.

\ex Dynamic sequencing (def.)\\
$m;n ≔ n ∘ m$
\xe

\ex Dynamic lift (def.)\\
$f^{↑} ≔ λxg . \set{g | f x}$\hfill$\type{(e → t) → e → T}$
\xe

\ex Discourse referent introduction (def.)\\
$f^{Δₙ} ≔ λ xg . \set{g^{[n ↦ x]} | f x}$\hfill$\type{(e → T) → e → T}$
\xe

\ex Pronouns (def.)\\
$\ml{pro}_{n} ≔ λ g . gₙ$
\xw

\printbibliography


\end{document}
