\documentclass[nols,twoside,nofonts,nobib,nohyper]{tufte-handout}

\usepackage{fixltx2e}
\usepackage{tikz-cd}
\usepackage{tcolorbox}
\usepackage{appendix}
\usepackage{listings}
\lstset{language=TeX,
       frame=single,
       basicstyle=\ttfamily,
       captionpos=b,
       tabsize=4,
  }

\begin{acronym}
\acro{sfa}{Scopal Function Application}
\acro{fa}{Function Application}
\acro{wco}{Weak Crossover}
\acro{ScoT}{Scope Transparency}
\acro{vfs}{Variable Free Semantics}
\acro{acd}{Antecedent Contained Deletion}
\acro{qr}{Quantifier Raising}
\acro{doc}{Double Object Construction}
\end{acronym}

\renewcommand*{\acsfont}[1]{\textsc{#1}}

\usepackage[font=footnotesize]{caption}

\makeatletter
% Paragraph indentation and separation for normal text
\renewcommand{\@tufte@reset@par}{%
  \setlength{\RaggedRightParindent}{0pt}%
  \setlength{\JustifyingParindent}{0pt}%
  \setlength{\parindent}{0pt}%
  \setlength{\parskip}{\baselineskip}%
}
\@tufte@reset@par

% Paragraph indentation and separation for marginal text
\renewcommand{\@tufte@margin@par}{%
  \setlength{\RaggedRightParindent}{0pt}%
  \setlength{\JustifyingParindent}{0pt}%
  \setlength{\parindent}{0pt}%
  \setlength{\parskip}{\baselineskip}%
}
\makeatother

\usepackage{multicol}

\setcounter{secnumdepth}{3}

\title{Crossover ii\thanks{24.979: Topics in
    semantics\\\noindent\textit{Getting high: Scope, projection, and evaluation order}}}

\author[Patrick D. Elliott and Martin Hackl]{Patrick~D. Elliott \& Martin Hackl}

\addbibresource[location=remote]{/home/patrl/repos/bibliography/elliott_mybib.bib}

\lingset{
  belowexskip=0pt,
  aboveglftskip=0pt,
  belowglpreambleskip=0pt,
  belowpreambleskip=0pt,
  interpartskip=0pt,
  extraglskip=0pt,
  Everyex={\parskip=0pt}
}

\usepackage{float}


% \usepackage{booktabs} % book-quality tables
% \usepackage{units}    % non-stacked fractions and better unit spacing
% \usepackage{lipsum}   % filler text
% \usepackage{fancyvrb} % extended verbatim environments
%   \fvset{fontsize=\normalsize}% default font size for fancy-verbatim environments

% % Standardize command font styles and environments
% \newcommand{\doccmd}[1]{\texttt{\textbackslash#1}}% command name -- adds backslash automatically
% \newcommand{\docopt}[1]{\ensuremath{\langle}\textrm{\textit{#1}}\ensuremath{\rangle}}% optional command argument
% \newcommand{\docarg}[1]{\textrm{\textit{#1}}}% (required) command argument
% \newcommand{\docenv}[1]{\textsf{#1}}% environment name
% \newcommand{\docpkg}[1]{\texttt{#1}}% package name
% \newcommand{\doccls}[1]{\texttt{#1}}% document class name
% \newcommand{\docclsopt}[1]{\texttt{#1}}% document class option name
% \newenvironment{docspec}{\begin{quote}\noindent}{\end{quote}}% command specification environment

\begin{document}

\maketitle% this prints the handout title, author, and date

\begin{tcolorbox}
  Some reminders
  \tcblower
  If you're a \textbf{registered student}, please send us your project proposal
  by the end of the week.
\end{tcolorbox}

\section{From continuations to dynamics}

\subsection{Barker \& Shan: \textsc{wco} as a reflex of evaluation order}

At the beginning of the semester, we learned about a theory of scope-taking with
a built-in left-to-right bias -- \textit{continuation semantics}.

Concretely, due to the way that the composition rule \ac{sfa} was defined,
evaluation of quantificational effects \textit{mirrors} linear
order.\sidenote{To cash this out, we needed to say something concrete about the
  syntax-semantics interface -- concretely, we committed to the ideas that (a)
  the basic combinatoric operation \textsc{Merge} is asymmetric, and (b) the
  syntactic and semantic composition proceed in lockstep (direct compositionality).}

As we saw in the last class, an appealing consequence of this linear bias was a
natural account of \acf{wco} in terms of evaluation order.\sidenote{See
  especially \cite{shanBarker2006} and \cite[chapters 2 and 4]{barkerShan2015}.}

Recall, a simplified version of the \ac{wco} paradigm: scope can feed binding
(\ref{ex:wco1}), unless the binder doesn't precede the bound expression (\ref{ex:wco2}).

\ex
{}[Everyone$^{x}$'s mother] bought them$_{x}$ a bicycle.\label{ex:wco1}
\xe

\ex~
Their$_{x}$ mother showered everyone$^{x}$ with gifts.\\
\textit{cf. a different person showered everyone with gifts.}\label{ex:wco2}
\xe

The idea, briefly, was to generalize our notion of a scope-taker to make sense
of the idea that pronouns also \textit{scope}.

In \citeauthor{barkerShan2015}'s system, pronouns take scope in the following
way: they expect \textit{a proposition}, and they return an \textit{open
  proposition}.\sidenote{We can helpfully think of an open proposition in this
  framework as a proposition with anaphoric effects (i.e., environment sensitivity).}

In order for a \ac{qp} to bind a pronoun, it must first be
\textit{bind-shifted}. A bind-shifted \ac{qp} expects an open proposition, and
returns a proposition. Successful binding is illustrated in figure \ref{fig:cont1}.

\begin{figure}
\caption{Successful binding}\label{fig:cont1}
\begin{forest}
  [{$\type{t}$}
  [{$\type{\semtower{t}{e}}$\\$\ml{S}$},edge label={node[midway,left,font=\scriptsize]{$↓$}}
    [{$\type{\tower{t}{e → t}{e}}$} [{$\type{\semtower{t}{e}}$\\every boy},edge label={node[midway,left,font=\scriptsize]{$B$}}]]
    [{$\type{\tower{e → t}{t}{e → t}}$\\$\ml{S}$}
      [{$\type{\semtower{e → t}{e → e → t}}$\\loves}]
      [{$\type{\tower{e → t}{t}{e}}$} [{his mother},roof]]
    ]
  ]]
\end{forest}
\end{figure}

Putting mechanisms for inverse scope to one side, \ac{wco} follows
straightforwardly from this system. Since both pronouns and bind-shifted
\acp{qp} are scope-takers, for the pronoun to be bound, the \ac{qp} has to be
evaluated first. Scope can feed binding, but the \ac{qp} must precede the
pronoun, since \textit{evaluation order mirrors linear order}.\sidenote{One of
  the virtues of continuation semantics is that it straightforwardly accounts
  for scope out of \ac{dp} without requiring \textit{movement} out of \ac{dp}.}

Continuation semantics includes mechanisms for subverting the linear bias
(namely, higher-order continuations), in
order to account for inverse scope.

With mechanisms for inverse scope in the picture, things become a little less neat. \citet{barkerShan2015} must
stipulate that \textit{lower} -- the operation via which continuized meaning are
collapsed into ordinary meanings -- is rigidly typed. If we assume that
\textit{internal lower} is derived as lifted \textit{lower}, this also has
consequences for its type:

\begin{multicols}{2}
\ex
$↓ : \type{\tower{a}{t}{t} → a}$
\xe
\columnbreak
\ex
$⇊ : \type{\semtower{b}{\tower{a}{t}{t}} → \semtower{b}{a}}$
\xe
\end{multicols}

This move basically guarantees, via a syntactic stipulation, that in order for a
bind-shifted \ac{qp} to bind a pronoun, it must take scope at the same
tower-story as the pronoun. If it takes scope on a high level, then the
resulting meaning cannot ultimately be lowered by a rigidly typed
\textit{lower}.

An unsuccessful attempt at getting internal lift to feed binding is illustrated
in figure \ref{fig:cont2}.

\begin{figure}
\caption{Unsuccessful binding (\ac{wco})}\label{fig:cont2}
\begin{forest}
  [{\xmark}
  [{$\type{\tower{t}{e → t}{e → t}}$}
  [{$\type{\tower{t}{e → t}{\tower{e → t}{t}{t}}}$\\$\ml{S}$},edge label={node[midway,left,font=\scriptsize]{$⇊$}}
    [{$\type{\semtower{t}{\tower{e → t}{t}{e}}}$} [{$\type{\tower{e → t}{t}{e}}$\\his mother},edge label={node[midway,left,font=\scriptsize]{$↑$}}]]
    [{$\type{\tower{t}{e → t}{\semtower{t}{e → t}}}$\\$\ml{S}$}
      [{$\type{\semtower{t}{\semtower{t}{e → e → t}}}$\\loves$^{↑ ∘ ↑}$}]
      [{$\type{\tower{t}{e → t}{\semtower{t}{e}}}$} [{every boy},edge label={node[midway,left,font=\scriptsize]{$⇈ ∘ B$}}]]
    ]
  ]
  ]]
\end{forest}
\end{figure}

What's crucial here is that both lower and internal lower are rigidly typed.

Just how satisfying is this as an explanation though? If we look at what lower
actually \textit{does}, there's no intrinsic reason why it should be so rigidly
typed.

As shown in (\ref{def:lower}), all that lower does is feed its sole argument the
identity function. On a maximally polymorphic definition, therefore, the
argument need only be of type $\type{(a → a) → b}$.


\ex Lower (maximally polymorphic ver.)\\
$m^{↓} ≔ m id$\label{def:lower}
\xe

A maximally polymorphic lower could save the \ac{wco}-violating derivation in
figure \ref{fig:cont2}.

Based on what lower \textit{does}, there's no strong \textit{semantic}
motivation for making it rigidly typed. Therefore, despite the initial
conceptual appeal of \citeauthor{barkerShan2015}'s system, its success
ultimately rests on what looks like a syntactic stipulation.

\subsection{Chierchia: \textsc{wco} as a reflex of the dynamics of anaphora}

\citet{chierchia2020} develops a theory of \ac{wco} based on \textit{dynamic
  semantics}.

Much like continuation semantics, \ac{ds} is a semantic theory with a
\enquote{built-in} left-to-right bias.

\section{Dynamic semantics}

\todo[inline]{Add more references below}

\ac{ds} is one of the most empirically successful theories of anaphora
(\citealt{heim1982,groenendijk_dynamic_1991,dekker1994}, a.o.) and
presupposition projection (\citealt{heim1983,beaver_presupposition_2001}, a.o.).
It has also been extended to a variety of other phenomena, including epistemic
modality, exhaustification (\citealt{elliott-twosouls}),
and more.

Crowning achievements of \ac{ds} include analyses of the following
phenomena:\sidenote{The (b) examples are included to briefly show that the
  phenomena under consideration exhibit a \textit{left-to-right asymmetry}, thus
motivating a dynamic treatment.}

\begin{itemize}

  \item Presupposition projection.\sidenote{Approaches to dynamic semantics are
    split as to whether they collapse presupposition satisfaction and anaphora
    resolution (see, e.g., \citealt{vanDerSandt1992}) or not
    (\citealt{heim1983}).

    The dynamic semantics ultimately adopted by \citeauthor{chierchia2020}
    follows the latter tradition. This won't be so important for the purposes of
    this class, but will be relevant when we talk about presupposition, starting
    from next week!
    }

    \pex
    \a{}[Ka visited Rome last summer]$^{α}$,\\
    and [she visited Rome again]$_{α}$
    this summer.
    \a\ljudge{\#}{}[Ka visited Rome again]$_{α}$,\\
    and [she visited Rome last summer]$^{α}$.
    \xe

    \item Donkey anaphora.

    \pex
    \a Every farmer who owns a donkey$^{3}$ treasures it$_{3}$.
    \a\ljudge{?}Every farmer who owns it$_{3}$ treasures a donkey$^{3}$.
    \xe

    \item Cross-sentential anaphora.

    \pex\label{ex:dy}
    \a A man$^{1}$ walked in. He$_{1}$ sat down.\label{ex:dy1}
    \a\ljudge{*}He$_{1}$ sat down. A man$_{1}$ walked in.\label{ex:dy2}
    \xe

 \end{itemize}

    Dwelling on cross-sentential anaphora, the contrast in (\ref{ex:dy}) is
    clearly reminiscent of a \ac{wco} effect.

    As we'll see, orthodox dynamic
    semantics doesn't by itself explain \ac{wco}, once quantificational scope is in the picture (see
    \citealt{charlow2019static} for discussion of this point), but
    \citeauthor{chierchia2020}'s basic intuition is to build a theory of
    \ac{wco} based on this contrast.

    In the next section, we'll introduce dynamic semantics by constructing an
    orthodox fragment that can account for cross-sentential anaphora. We'll move
    on to show how it fails to capture \ac{wco}, before moving on to
    \citeauthor{chierchia2020}'s modification.

\end{itemize}

\subsection{A Heimian fragment}

Sentential meanings in \ac{ds} (\citealt{heim1982}, \citealt{groenendijk_dynamic_1991},
\citealt{chierchia_dynamics_1995}), have two essential components:

\begin{itemize}

    \item An input-output asymmetry -- sentences denote \textit{instructions} to
    change the input context (see especially \citealt{heim1982}).

    \item Indeterminacy -- certain expressions may induce an
    \textit{indeterminate} output.

\end{itemize}



To model this formally, many theories of \ac{ds} model sentence meanings as \textit{relations between assignments} (equivalently: functions
from assignments, to sets of assignments).

\acp{dp} introduce \acp{dr}, modeled as variables; indefinites, unlike definites
induce \textit{indeterminacy}, concerning the identity of the \ac{dr}. This is
illustrated schematically in figure \ref{fig:schema}.\sidenote[][-15ex]{One way of
  thinking about this: definites induce a \textit{functional} relation between
  assignments -- every input assignment is mapped to a unique output assignment,
whereas indefinites induce a \textit{non-functional} relation between
assignments -- each input assignment can mapped to one or more output assignments.}

\begin{figure}
\caption{Relations between assignments}\label{fig:schema}
\begin{forest}
  [{$ω$} [{Roger$^{n}$ arrived late.} [{$ω^{[n ↦ \ml{roger}]}$}]]]
\end{forest}
%
\begin{forest}
  [{$ω$} [{A linguist$^{n}$ arrived late}
    [{$ω^{[n ↦ \ml{kai}]}$}]
    [{$ω^{[n ↦ \ml{roger}]}$}]
    [{$ω^{[n ↦ \ml{sabine}]}$}]
    [{$ω^{[n ↦ \ml{athulya}]}$}]
    [{$ω^{[n ↦ \ml{martin}]}$}]
  ]]
\end{forest}
\end{figure}

\textit{Assignments} are functions from variables to individuals; as is
standard, we'll represent the set of variables as $ℕ$:\sidenote{We'll use
  $\type{o}$ as the type of assignments to distinguish between assignments used
  in a static setting.}

\ex Type of assignments\\
$\type{o} ≔ \type{n → e}$
\xe

\citeauthor{chierchia2020} assumes that assignments are \textit{partial}
functions.\sidenote{See also \citet{rothschildMandelkern2017esslli} for a
  \ac{ds} using partial assignments.} That is to say, an assignment
may only be defined for certain indices. The following are all valid assignments:

  \begin{multicols}{3}
  $$\left[
      1 ↦ \ml{roger}
    \right]$$
  \columnbreak
  $$\left[\begin{aligned}[c]
      &1 ↦ \ml{roger}\\
      &3 ↦ \ml{martin}
    \end{aligned}\right]$$
  \columnbreak
  $$\left[\begin{aligned}[c]
      &4 ↦ \ml{kai}\\
      &5 ↦ \ml{athulya}\\
      &7 ↦ \ml{sabine}\\
    \end{aligned}\right]$$
\end{multicols}

In order to characterize a dynamic sentential meaning, we define a type
constructor $\type{T}$ to abbreviate relations between assignments:

\ex Type of \acp{ccp}\\
$\type{T} ≔ \type{o → o → t}$
\xe

\todo[inline]{Add some example sentential meanings here}

% Here are some example sentence meanings:

% \ex
% $\eval{Roger arrived late} = λ g . λ g' . g[n ↦ \ml{r}]g' ∧ \ml{arrived-late r}$\sidenote{$g[n ↦ x]g'$
% is \textit{true}, just in case $g'$ \textit{at most} differs from $g$ with
% respect to $n$ ($g_{n}$ can be undefined), and $g'_{n} = \ml{r}$.}\hfill$\type{T}$
% \xe

% \exe% $\eval{A linguist$^{n}$ arrived late} = λ g . λ g' . ∃x[g[n → x]g' ∧ \ml{arrived-late} x]$\hfill$\type{T}$
% \xe

We can get back an \enquote{ordinary} sentential meaning from a \ac{ccp} by
existentially closing the output assignment, as defined in (\ref{def:dyclo}).


\ex Dynamic closure (def.)\\
$m^{↓}} ≔ λ ω . ∃ω'[m ω ω']$\label{def:dyclo}\hfill$\dyclo : \type{T → t}$
\xe

How do we build up \acp{ccp} compositionally? \citeauthor{chierchia2020} assumes
that predicates are fundamentally Montagovian (i.e., functions of type
$\type{e → t}$):

\ex
$\eval{swim} ≔ λ x . \ml{swim} x$\hfill$\type{e → t}$
\xe



Predicates are lifted into a dynamic setting by a type-shifter
\textit{dynamic lift}; d-lift takes a function from an individual to a
truth-value, and shifts it into a function from an individual to a \ac{ccp} --
specifically, a dynamic \textit{test}.\sidenote{
A different way of generalizing this to $n-$place predicates is by giving $Δ$ the following definition:

\ex
$m^{Δ} ≔ λ k . λ ωω' . ω = ω' ∧ m k$\\
$Δ: \type{((a → t) → t) → (a → t) → t}$
\xe

\Citeauthor{chierchia2020}'s d-lift can be derived as follows, using the (by now
very familiar) continuation semantics operations:

\ex
$λ x . (x^{↑} \ml{S} f^{Δ ∘ ↑})^{↓}$
\xe
}

\ex
Dynamic lift (def.)\\
$f^{\dlift} ≔ λ x . λ ω . λ ω' . ω = ω' ∧ f x$\hfill$\dlift : \type{(e → t) → e
  → T}$
\xe

\begin{tcolorbox}
\textbf{Exercise}
\tcblower
\citeauthor{chierchia2020} defines dynamic lift in such a way that it only can
apply to one-place predicates. This is not insignificant -- see the discussion
of event semantics later on. It is however trivial to generalize to d-lift to
$n$-place predicates.

Generalize \textit{dynamic lift} to $n$-place predicates by giving a recursive
definition a la \cite{parteeRooth}.
\end{tcolorbox}

Tests don't do anything interesting to input contexts. In orthodox dynamic
fragments, all of the interesting dynamic action is triggered by arguments --
specifically, pronouns and indefinites.


A pronoun indexed $n$ expects a dynamic predicate $k$ as its input, and returns
a \ac{ccp} -- a function from an input assignment $ω$ to the result of feeding
$ω_{n}$ into $k$, \textit{re-}saturated with $ω$.\sidenote{The type signature of
a pronoun betrays the fact that, in this dynamic grammar, pronouns are
\textit{scope-takers}, and in fact, we can abbreviate a pronominal meaning using
tower notation:

\ex
Pronouns (tower def.)\\
$\ml{pro}_{n} ≔ \semtower{λ ω . ([] ω)}{ω_{n}}$
\xe

Interestingly, this is what we get if we apply the \textit{bind} of the
\texttt{Reader} monad to the static entry for a pronoun.

\ex Pronoun (static def.)\\
$\ml{pro}_{n} ≔ λω . ω_{n}$\\
\phantom{,}\hfill$\type{o → e}$
\xe

\ex~ Bind of \texttt{Reader} (def.)\\
$m^{⋆} ≔ λ k . λ ω . k (m ω) ω$\\
\phantom{,}\hfill$\type{(o → a) → (a → o → b) → o → b}$
\xe
}

\ex
Pronouns (def.)\\
$\ml{pro}_{n} ≔ λ k . λ ω . k ω_{n} ω$\sidenote{
\citeauthor{chierchia2020} actually posits a syncategorematic rule for composing
pronouns and dynamic predicates -- instead, I've built what
\citeauthor{chierchia2020}'s rule does into the meaning of the pronoun.
}\hfill$\ml{pro}_{n} : \type{(e → T) → T}$
\xe

Pronouns now may compose with d-lifted predicates via \ac{fa}, as illustrated
in figure \ref{fig:pro}:

\begin{figure}
\caption{Pronouns in a dynamic fragment\\\enquote{He$_{3}$ swims}}\label{fig:pro}
\begin{forest}
  [{$λωω' . ω = ω' ∧ \ml{swim} ω_{3}$},fill=yellow
    [{$λ k . λ ω . k ω_{3} ω$\\he$_{n}$}]
    [{$λ x . λ ωω' . ω = ω' ∧ \ml{swim} x$} [{swim},edge label={node[midway,left,font=\scriptsize]{$\dlift$}}]]
  ]
\end{forest}
\end{figure}

The result is a dynamic \textit{test}, that saturates the argument of
\textit{swim} with whatever the input assignment $ω$ maps to pronominal index
$3$ to.

In an orthodox dynamic fragment (\citealt{heim1982,groenendijk_dynamic_1991}), indefinites introduce
\acp{dr}.\sidenote{\citeauthor{chierchia2020} will ultimately reject this
  assumption, but it will be useful to consider his claims in light of the
  standard theory.}\sidenote{$ω \stackrel{n/x}{=} ω'$ is defined iff $ω_{n}$ is
  \textit{un}defined, and is true just in case $ω'$ differs from $ω$ at most in
  what $n$ is mapped to.

\citeauthor{heim1982}'s \textit{novelty condition} is essentially built into the
rule for \ac{dr} introduction.}

\ex
Indefinites (Heimian def.)\\
$\ml{someone}_{n} ≔ λ k . λ ωω' . ∃x,w''[ω \stackrel{n/x}{=} ω'' ∧ k x ω'' ω']$\\
\phantom{,}\hfill$\ml{someone}_{n} \type{(e → T) → T}$
\xe

In figure \ref{fig:indef}, we show how a Heimian indefinite composes in a
dynamic fragment. The result maps each input assignment $ω$ to (the
characteristic function of) a \textit{set} of assignments $ω'$, s.t., $ω'_{n}$
is a swimmer.

\begin{figure}
\caption{Heimian indefinites in a dynamic fragment\\\enquote{Someone$_{7}$ swims}}\label{fig:indef}
\begin{forest}
  [{$λ ωω' . ∃x[ω \stackrel{7/x}{=} ω' ∧ \ml{swim} x]$},fill=yellow
  [{$λωω' . ∃x,w''[ω \stackrel{7/x}{=} ω'' ∧ ω'' = ω' ∧ \ml{swim} x]$},edge label={node[midway,left,font=\scriptsize]{equiv}}
    [{$λ k . λ ωω' . ∃x,w''[ω \stackrel{7/x}{=} ω'' ∧ k x ω'' ω']$\\someone$_{7}$}]
    [{$λ x . λ ωω' . ω = ω' ∧ \ml{swim} x$} [{swim},edge label={node[midway,left,font=\scriptsize]{$\dlift$}}]]
  ]]
\end{forest}
\end{figure}

A famous design feature of \ac{ds} is an account of cross-sentential
binding, as in the following famous examples:

\pex
\a Someone$^{1}$ walked in and he$_{1}$ sat down.\label{ex:seq1}
\a Someone$^{1}$ walked in. he$_{1}$ sat down.\label{ex:seq2}
\xe

In \ac{ds}, conjunction -- as in (\ref{ex:seq1}) -- is treated as a
special case of \textit{discourse sequencing} (\ref{ex:seq2}).

Discourse sequencing is an operation on \acp{ccp}:

\ex Dynamic sequencing (def.)\\
$m ; n ≔ λω . λ ω' . ∃ω''[m ω ω'' ∧ n ω'' ω']$\hfill$(;): \type{T → T → T}$\label{def:conj}
\xe

An illustration of how cross-sentential anaphora works in a Heimian fragment is
given in figure \ref{fig:anaph}: sequencing the \acp{ccp} gives rise to a
\ac{ccp} that relates $ω$ and $ω'$ just in case $ω_{4}$ is undefined and and
$ω'_{4}$ walked in and sat down.

\begin{figure}
\caption{Cross-sentential anaphora in a Heimian fragment}\label{fig:anaph}
\begin{forest}
  [{$λ ωω' . ∃ω''[(∃x[ω \stackrel{4/x}{=} ω'' ∧ \ml{walked-in} x]) ∧ (ω'' = ω' ∧ \ml{sat-down} ω''_{4})]$},fill=yellow,name=landing
  site
    [{$λ ωω' . ∃x[ω \stackrel{4/x}{=} ω' ∧ \ml{walked-in} x]$},draw=red [{Someone$^{4}$ walked in},roof]]
    [{...}
      [{$λ n . λ m . λωω' . ∃ω''[m ω ω'' ∧ n ω'' ω']$\\$;$}]
      [{$λωω' . ω = ω' ∧ \ml{sat-down} ω_{3}$},name=trace,draw=red [{he$_{4}$ sat down},roof]]
    ]
  ]
\end{forest}
\end{figure}

Why is \ac{ds} promising as a starting point for a theory of \ac{wco}? Recall the contrast below, reminiscent of \ac{wco}:

\pex
\a Someone$^{4}$ walked in and he sat down.
\a\ljudge{*}He$_{4}$ walked in and someone$^{4}$ sat down.\label{ex:cat2}
\xe

Just so long as \textit{someone} takes scope within its containing sentence,
\ac{ds} captures this contrast, by virtue of the left-to-right bias
built into the definition of discourse sequencing.

If we try to compute the \ac{ccp} for (\ref{ex:cat2}), the result is guaranteed
to be undefined. This is because, if the input assignment $ω$ is defined for
$4$, it can't also be \textit{undefined} for $4$, as is required by the meaning
contributed by the indefinite.

\ex
$λ ωω' . ∃ω''[ω = ω'' ∧ \ml{sat-down} ω_{4} ∧ (∃x[ω'' \stackrel{4/x}{=} ω' ∧ \ml{walked-in} x])]$
\xe

\ac{ds} doesn't by itself however capture \ac{wco} -- this is because,
independently, we need a mechanism that allows indefinites to \textit{take
  scope}; indefinites introduce discourse referents at their
scope site. We can therefore compute a bound reading for the following example,
by scoping the indefinite over the pronoun:\sidenote{For reasons that will
  become clear here, we're being a bit sneaky concerning how the scoped
  indefinite binds its trace.}

\begin{figure}
  \caption{Violating \ac{wco} in a Heimian fragment\\\enquote{He$_{4}$ wants to
      meet someone$^{4}$}}
  \begin{forest}
  [{$λ ωω' . ∃x[ω \stackrel{4/x}{=} ω' ∧ ω'_{4} \ml{want} (ω_{4} \ml{meet} x)]$},fill=yellow
    [{someone$_{4}$},name=intermediate]
    [{$λ x . λ ωω' . ω = ω' ∧ ω_{4} \ml{want} (ω_{4} \ml{meet} x)$}
    [{$λ x$}]
    [{$λ ωω' . ω = ω' ∧ ω_{4} \ml{want} (ω_{4} \ml{meet} x)$}
      [{he$_{4}$}]
      [{$λ y . λωω' . ω = ω' ∧ y \ml{want} (y \ml{meet} x)$} [{$λ y . y$ wants to meet $x$},edge label={node[midway,left,font=\scriptsize]{$Δ$}},name=trace]]
    ]
  ]]
  {
    \draw[semithick, dashed, ->] ([xshift=4em]trace.south) to[out=south,in=south] (intermediate); % e
  }
  \end{forest}
\end{figure}

Intuitively, a problematic feature of \ac{ds} in this regard is that it ties
together \ac{dr} introduction with quantificational scope.


\subsection{The Dynamic Prediction Principle}

At the heart of \citeauthor{chierchia2020} account of \ac{wco} is an apparently
minor modification to orthodox dynamics, with far reaching consequences: the
\acf{dpp}, stated in (\ref{def:dpp}).

\ex
The \acf{dpp}\\
\acp{dr} can only be introduced by predicates.\hfill (\citealt[p.\,32]{chierchia2020})\label{def:dpp}
\xe

\citeauthor{chierchia2020}'s innovation is to posit a second way of lifting
predicates into a dynamic setting: \textit{\ac{dr}-lifting}.\sidenote{If
  you try to generalize \ac{dr}-lift to $n$-place predicates, you'll
  find that it can't be done in quite the same way as for d-lift. As an
  exercise, try to figure out why this.}

\ex
\ac{dr}-lift (def.)\\
$f^{Δ_{n}} ≔ λ x . λ ω . λ ω' ω \stackrel{n/x}{=} ω' ∧ f x$\hfill$Δ_{n}: \type{(e → t) → e
  → T}$
\xe

Introducing \acp{dr} then, is no longer the job of \textit{indefinites}, but
rather the job of a \ac{dr}-lifted predicate.

What do indefinites do then? For \citeauthor{chierchia2020}, they're just
type-lifted first-order quantifiers.\sidenote{Looking at the definition in
  (\ref{def:someone}), you may be wondering how \textit{someone} binds its
  trace. \citeauthor{chierchia2020} does something rather sneaky here, which
  will be important later. For now, assume that it just works.}

\ex
Dynamic existential quantification (def.)\\
$\ml{someone}_{n} m ≔ λωω' . ∃x_{n}[m ω ω']$\hfill$\type{\ml{someone}_{n}: \type{T → T}}$\label{def:someone}
\xe

\textit{Someone} saturates the argument that a \ac{dr} was introduced relative
to, and cross-sentential anaphora proceeds as usual.

\begin{figure}
  \centering
  \caption{Example derivation}
  \begin{forest}
    [{$λ ωω' . ∃ω''[∃x[ω \stackrel{1/x}{=} ω'' ∧ \ml{walked-in} x] ∧ ω'' = ω' ∧ \ml{sat-down} ω''_1]$},fill=yellow
    [{$λ ωω' . ∃x[ω \stackrel{1/x}{=} ω' ∧ \ml{walked-in} x]$},draw=red
      [{someone$_{2}$}]
      [{$λ ωω' . ω \stackrel{1/x}{=} ω' ∧ \ml{walked-in} x_{2}$}
      [{$t_{2}$}]
      [{$λ x . λ ωω' . ω \stackrel{1/x}{=} ω' ∧ \ml{walked-in} x$} [{walked in},edge label={node[midway,left,font=\scriptsize]{$Δ_{1}$}}]]]]
    [{...}
    [{$;$}]
      [{$λ ωω' . ω = ω' ∧ \ml{sat-down} ω_1$},draw=red
        [{$λ k . λ ω . k ω_1 ω$\\he$_1$}]
        [{$λ y . λ ωω' . ω = ω' ∧ \ml{sat-down} y$} [{sat-down},edge label={node[midway,left,font=\scriptsize]{$Δ$}}]]
      ]
    ]]
  \end{forest}
\end{figure}

So far, we've constructed a system which replicates the basic results of
orthodox dynamic semantics, but with a different compositional regime.

\subsection{Accessibility}


In the previous section, we only gave definitions for \textit{dynamic
  conjunction/discourse sequencing} and the static first order existential.

\citeauthor{chierchia2020} adopts the standard dynamic definitions for the other
logical operators.

Negation is taken to be \textit{externally
  static}; any \acp{dr} introduced in the scope of negation are
subsequently wiped out.

\ex
Dynamic negation (def.)\\
$¬ m ≔ λ ω . λω' . ω = ω' ∧ ¬ (m^{↓} ω)$\hfill$¬ : \type{T → T}$
\xe

Externally static negation predicts the impossibility of binding in the following:

\ex
\ljudge{*}It's not the case that anyone$^{1}$ walked in. He$_{1}$ sat down.
\xe

To see why, first consider the prejacent of negation, with \ac{dr}-lift applied
to the predicate:\sidenote{We simplify here and assume that \ac{npi}
  \textit{any} is just an existential licensed in the scope of negation.}

\ex
$\eval{anyone walked in} = λ ωω' . ∃x[ω \stackrel{1/x}{=} ω' ∧ \ml{walked-in} x]$\label{first-conj}
\xe

Applying dynamic negation to the above \ac{ccp} \textit{existentially closes}
the output assignment, thereby rendering it dynamically inert. The resulting
\ac{ccp} is a dynamic test, and asserts that there is no way of extending the
input assignment s.t. $1$ is mapped to someone who walked in (in other words,
nobody walked in).

\ex
$¬ (\ref{first-conj}) = λ ωω' . ω = ω' ∧ ¬ ∃ω'',x[ω \stackrel{1/x}{=} ω'' ∧ \ml{walked-in} x]$
\xe

Sequencing this \ac{ccp} with the second conjunct will clearly not give rise to anaphora.

The remainder of the logical operations can be defined via first-order
equivalent via dynamic conjunction, negation, and existential quantification.
All are defined as operations on \acp{ccp}.

\ex Dynamic implication (def.)\\
$m → n ≔ ¬ (m ; ¬ n)$\hfill$(→) : \type{T → T → T}$
\xe

\ex~ Dynamic disjunction (def.)\\
$m ∨ n ≔ ¬ (¬ m ; ¬ n)$\hfill$(∨): \type{T → T → T}$
\xe

\ex~ Dynamic universal quantifier (def.)\\
$\ml{everyone}_{n} m ≔ ¬ ∃x_{n} (¬ m)$\hfill$\type{T → T}$
\xe

Famously, this way of dynamicizing the logical connectives gives rise to the
following accessibility hierarchy in complex sentences:

\ex Accessibility (def.)\\
A is \textit{accessible} to B if a \ac{dr} active in A can covary with a pronoun
in B.
\xe

    Accessibility in conjunctive sentences: [A and B]\sub{S}

    \begin{itemize}

        \item A is accessible to B (but not vice versa).

        \item B is accessible to whatever is conjoined with S.

    \end{itemize}

        \ex
        A man$^{1}$ walked in, and he$_{1}$ sat down. He$_{1}$ stood up again
        soon after.
        \xe


    Accessibility in conditional sentences: [if A then B]\sub{S}

    \begin{itemize}

        \item A is accessible to B (but not vice versa).

        \item A, B are \textit{not} accessible to what is conjoined with S.\sidenote{In dynamic semantics, conditional sentences are
        internally dynamic, but externally static.}

    \end{itemize}

            \ex
        \ljudge{\#}If someone$^{1}$ won the lottery, they$_{1}$ became rich. I
        shook their$_{1}$ hand.
        \xe


    Accessibility in negative sentences: [not A]\sub{S}

    \begin{itemize}

        \item Nothing in A is accessible to what is conjoined with S.

    \end{itemize}

            \ex
        \ljudge{*}It's not the case that anyone$^{1}$ sat down. He$_{1}$ walked in.
        \xe

    Accessibility in disjunctive sentences: [A or B]\sub{S}

    \begin{itemize}

        \item A is not accessible to B, nor is B to A.

        \item Neither A not B is accessible to what is conjoined with S.\sidenote{Dynamic disjunction is both internally static and
        externally static.}

    \end{itemize}

            \ex
        \ljudge{\#}Either Mary has a new dog$^{1}$, or I petted it$_{1}$.
        \xe


\subsection{Enter events}

So far, we've constructed a fragment that only accommodates one-place
predicates. This is actually by design -- \citeauthor{chierchia2020} argues that
such a system has a natural bed-fellow in neo-Davisonian event semantics.

Traditions in event semantics:

Davidsonian:

\ex
$\eval{love} ≔ λ exy . \ml{exp} e = y ∧ \ml{th} e = x ∧ \ml{love} e$\hfill$\type{v → e → e → t}$
\xe

Neo-Davidsonian (\citealt{castaneda1967,parsons_events_1990}):

\ex
$\eval{love} ≔ λ e . \ml{love} e$\hfill$\type{v → t}$
\xe

According to the neo-Davidsonian approach, all arguments are severed, and
instead introduced by thematic role heads (the compositional regime adopted here
is after \citealt{champollion_interaction_2015}):\sidenote{See, e.g. \citealt{ahn2016} and
\citealt{elliottDiss} for independent evidence for this position from different domains.}

\ex
$\ml{THEME} ≔ λ x . λ e . \ml{th} e = x$\hfill$\type{e → v → t}$
\xe

Abstracting away from dynamics for a moment, the composition of a simple
sentence in a neo-Davidsonian setting can proceed via \ac{pm}:

\begin{figure}
\centering
\caption{composition in a neo-Davidsonian event semantics}
\begin{forest}
    [{$λ e . \ml{exp} e = \ml{john} ∧ \ml{th} e = ιx[\ml{cat} x] ∧ \ml{love} e$\\\ac{pm}},fill=yellow
    [{$λ e . \ml{exp} e = \ml{john}$}
      [{$\ml{EXP}$}]
      [{DP\\John}]
    ]
        [{$λ xe . \ml{th} e = ιx[\ml{cat} x] ∧ \ml{love} e$\\\ac{pm}}
        [{$λ e . \ml{th} e = ιx[\ml{cat} x]$}
          [{$\ml{THEME}$}]
          [{DP} [{the cat},roof]]
        ]
          [{$λ e . \ml{love} e$\\love}]
      ]
    ]
\end{forest}
\end{figure}

Note that, since verbs denote one place predicates, they can be \ac{dr}-lifted.

Since the event argument of a verb is (by stipulation) existentially closed,
this accounts for the possibility of eventive \acp{dr}, as in the following
example:

\ex
It rained$^{4}$. It$_{4}$ was heavy.
\xe

We can assume the following \ac{lf}:

\begin{figure}
  \centering
  \caption{Eventive \acp{dr}}
  \begin{forest}
    [{$λ ωω' . ∃ω''[(∃e[ω \stackrel{4/e}{=} ω'' ∧ \ml{rain} e]) ∧ ω'' = ω' ∧ \ml{heavy} ω_{4}]$},fill=yellow
    [{$λωω . ∃e[ω \stackrel{4/e}{=} ω' ∧ \ml{rain} e]$},draw=red
      [{$∃$}]
      [{$λ eωω' . ω \stackrel{4/e}{=} ω' ∧ \ml{rain} e$} [{rain},edge label={node[midway,left,font=\scriptsize]{$Δ_{4}$}}]]
    ]
      [{...}
        [{$;$}]
        [{$λ ωω' . ω = ω' ∧ \ml{heavy} ω_{4}$},draw=red [{it was heavy},roof]]
      ]
    ]
  \end{forest}
\end{figure}

The discussion in the paper is quite confusing at this point, but
\citeauthor{chierchia2020} seems to assume that \ac{dr}-lift extends
straightforwardly to thematic argument-introducing heads. This is in fact not
the case -- we have to generalize \ac{dr}-lift to predicates of type
$\type{e → v → t}$.\sidenote{As far as I can see, the compositional details of
  the system as laid out by \citeauthor{chierchia2020} at this point are
  incoherent, but easily fixed.}

\ex Thematic \ac{dr}-lift (def.)\\
$f^{Δ_{n}} ≔ λ x . λ e . λωω' . ω \stackrel{n/x}{=} ω' ∧ f x e$\hfill$Δ_{n}:\type{(e → v → t) → e → v → T}$
\xe

We furthermore must assume that \textit{dynamic} \ac{pm} is a freely available
semantic composition rule.

\ex Dynamic \ac{pm} (def.): $\eval{\begin{aligned}[c]
    \begin{forest}
    [
      [{$m_\type{a → T}$}]
      [{$n_\type{a → T}$}]
    ]
    \end{forest}
  \end{aligned}} ≔ λ x_{\type{a}} . m x ; n x$
\xe

If we return to our simple example, we can now apply \ac{dr}-lift every step of
the way, as in figure \ref{ex:inter}:

\begin{figure}
\centering
\caption{Dynamic neo-Davidsonian event semantics}\label{ex:inter}
\begin{forest}
  [{$λ e . \begin{aligned}[t]
      &(λωω' . ω \stackrel{3/j}{=} ω' ∧ \ml{exp} e = j)\\
      &;(λωω' . ω \stackrel{2/ιx[\ml{cat} x ω_{3}]}{=} ω' ∧ \ml{th} e = ιx[\ml{cat} x ω_{3}])\\
      &;(λωω' . ω \stackrel{1/e}{=} ω' ∧ \ml{love} e)\end{aligned}$\\d-\ac{pm}},fill=yellow
    [{$λ e . λωω' . ω \stackrel{3/j}{=} ω' ∧ \ml{exp} e = j$},draw=red
      [{...} [{$\ml{EXP}$},edge label={node[midway,left,font=\scriptsize]{$Δ_{3}$}}]]
      [{DP\\John}]
    ]
        [{d-\ac{pm}}
        [{$λ e . λωω' . ω \stackrel{2/ιx[\ml{cat} x ω_{3}]}{=} ω' ∧ \ml{th} e = ιx[\ml{cat} x ω_{3}]$},draw=red
          [{...} [{$\ml{THEME}$},edge label={node[midway,left,font=\scriptsize]{$Δ_{2}$}}]]
          [{DP} [{his$_{3}$ cat},roof]]
        ]
          [{$λ e . λωω' . ω \stackrel{1/e}{=} ω' ∧ \ml{love} e$},draw=red [{love},edge label={node[midway,left,font=\scriptsize]{$Δ_{1}$}}]]
      ]
    ]
\end{forest}
\end{figure}

What this essentially buys us is a sentence-internal accessibility hierarchy --
the subject and object are accessible to the verb, but not nice versa, and the
subject is accessible to the object, but not vice versa.

This can be leveraged in order to account for the basic cases of \ac{wco}:

\todo[inline]{Show exactly how this works:w
}

\subsection{Binding into adjuncts}

\subsection{Bare plurals}

\todo[inline]{Finish this section}

\section{Problems}

\subsection{The problem of existentials}

\section{Rethinking the system}

Conceptual issues:

\begin{itemize}

    \item Much like, e.g., \ac{dmg}, \citeauthor{chierchia2020}'s system
    implicitly makes a syntactic distinction between two different kinds of
    variables.\sidenote{See \citeauthor{barkerShan2008}'s
    (\citeyear{barkerShan2008}) criticism of \ac{dmg} for related discussion.}
    Can we do better? Continuations provide a way of doing scope-taking without
    any need for traces, so this issue should (hopefully) dissolve if we shift
    to a continuation-based theory.

\end{itemize}

\subsection{A more general dynamic system}

\citet{charlow2019static} (see also \citealt{Charlowc}) generalizes orthodox
dynamic semantics. We'll adopt his system here.

\textit{Dynamic} $a$s are functions from an input assignment $g$, to a set of
$a$s paired with output assignments $g'$.

\ex
$\type{D a} ≔ g → \set{(a,g)}$
\xe

\ex Unit (def.)\\
$a^{η} ≔ λ g . \set{(a,g)}$
\xe

\todo[inline]{Redo definition of bind}

\ex Bind (def.)\\
$m^{⋆} ≔ λ k . λ g . \bigcup\limits_{(x,h) ∈ m g}\set{k x h}$\hfill$⋆: \type{D a → (a → D b) → D b}$
\xe

\ex
$\ml{pro}_{n} ≔ λ g . \set{(g_{n},g)}$\hfill$\type{D e}$
\xe

\ex
$\ml{some boy}_{n} ≔ λ g . \set{(x,g) | \ml{boy} x}$\hfill$\type{D e}$
\xe

\ex
$m^{Δ_{n}} ≔ λ g . \bigcup\limits_{(x,h) ∈ m g}\set{(x,h^{[n → x]})}$\hfill$\type{D e → D e}$
\xe

\ex
$\ml{Q} ≔ λ k . λ g . \set{(Q (λ x . (k x)^{↓_{g}})),g)}$\hfill$\ml{everyone}:\type{\semtower{D t}{e}}$
\xe



\printbibliography


\end{document}
