\documentclass[nols,twoside,nofonts,nobib,nohyper]{tufte-handout}

\usepackage{fixltx2e}
\usepackage{tikz-cd}
\usepackage{tcolorbox}
\usepackage{appendix}
\usepackage{listings}
\lstset{language=TeX,
       frame=single,
       basicstyle=\ttfamily,
       captionpos=b,
       tabsize=4,
  }

\begin{acronym}
\acro{sfa}{Scopal Function Application}
\acro{fa}{Function Application}
\acro{wco}{Weak Crossover}
\acro{ScoT}{Scope Transparency}
\acro{vfs}{Variable Free Semantics}
\acro{acd}{Antecedent Contained Deletion}
\acro{qr}{Quantifier Raising}
\acro{doc}{Double Object Construction}
\end{acronym}

\renewcommand*{\acsfont}[1]{\textsc{#1}}


\makeatletter
% Paragraph indentation and separation for normal text
\renewcommand{\@tufte@reset@par}{%
  \setlength{\RaggedRightParindent}{0pt}%
  \setlength{\JustifyingParindent}{0pt}%
  \setlength{\parindent}{0pt}%
  \setlength{\parskip}{\baselineskip}%
}
\@tufte@reset@par

% Paragraph indentation and separation for marginal text
\renewcommand{\@tufte@margin@par}{%
  \setlength{\RaggedRightParindent}{0pt}%
  \setlength{\JustifyingParindent}{0pt}%
  \setlength{\parindent}{0pt}%
  \setlength{\parskip}{\baselineskip}%
}
\makeatother

\setcounter{secnumdepth}{3}

\title{Getting high: scope, projection, and evaluation order\thanks{24.979: Topics in
    semantics}}

\author[Patrick D. Elliott and Martin Hackl]{Patrick~D. Elliott\sidenote{\texttt{pdell@mit.edu}} \& Martin Hackl\sidenote{\texttt{hackl@mit.edu}}}

\addbibresource[location=remote]{/home/patrl/repos/bibliography/elliott_mybib.bib}

\lingset{
  belowexskip=0pt,
  aboveglftskip=0pt,
  belowglpreambleskip=0pt,
  belowpreambleskip=0pt,
  interpartskip=0pt,
  extraglskip=0pt,
  Everyex={\parskip=0pt}
}


% \usepackage{booktabs} % book-quality tables
% \usepackage{units}    % non-stacked fractions and better unit spacing
% \usepackage{lipsum}   % filler text
% \usepackage{fancyvrb} % extended verbatim environments
%   \fvset{fontsize=\normalsize}% default font size for fancy-verbatim environments

% % Standardize command font styles and environments
% \newcommand{\doccmd}[1]{\texttt{\textbackslash#1}}% command name -- adds backslash automatically
% \newcommand{\docopt}[1]{\ensuremath{\langle}\textrm{\textit{#1}}\ensuremath{\rangle}}% optional command argument
% \newcommand{\docarg}[1]{\textrm{\textit{#1}}}% (required) command argument
% \newcommand{\docenv}[1]{\textsf{#1}}% environment name
% \newcommand{\docpkg}[1]{\texttt{#1}}% package name
% \newcommand{\doccls}[1]{\texttt{#1}}% document class name
% \newcommand{\docclsopt}[1]{\texttt{#1}}% document class option name
% \newenvironment{docspec}{\begin{quote}\noindent}{\end{quote}}% command specification environment

\begin{document}

\maketitle% this prints the handout title, author, and date

\begin{abstract}
\enquote{The seminar will provide a venue for discussing various mechanisms for scope-taking and projection, taking as our starting point continuations -- a perspective on scope-taking developed by Chris Barker and Chung-chieh Shan. We will attempt to develop a solid working knowledge of the relevant mechanics, as well as arrive at a comprehensive empirical assessment of their advantages and drawbacks in selected areas of application. These will include quantifier scope, variable binding, cross-over, and presupposition projection, paying particular attention to linearity effects which continuations are designed to handle in a principled manner.}
\end{abstract}

\section{Tentative schedule}


\textbf{Thursdays 14:00-17:00, 32-D461}

The first few weeks are fairly stable, but later weeks will probably
shift.\sidenote{Note, Patrick is down as teaching \textit{crossover i}, but will
be at a conference that week, so this session will need to be rescheduled. We'll
be in touch about that soon.}

\begin{table}[!ht]
\begin{tabular}{@{}lll@{}}
\toprule
  date & content  & lecturer  \\ \midrule
  today & continuations i &  Patrick \\
  02.13 & continuations ii & Patrick  \\
  02.20 & the view from QR i & Martin  \\
  02.27 & the view from QR ii & Martin  \\
  03.05 & crossover i (\citealt{shanBarker2006}) & Patrick \\
  03.12 & crossover ii (Chierchia, to appear) &  Martin \\
  03.19 & presupposition as scope (\citealt{grove2019}) & Patrick \\
  04.02 & Haddock's puzzle (\citealt{bumford2017}) & Martin \\
  04.09 & TBA & Patrick  \\
  04.16 & presupposition and reference resolution & Martin  \\
  04.23 & TBA & TBA \\
  04.30 & \textit{student presentations} & N/A  \\
  05.07 & \textit{student presentations} & N/A \\
 \bottomrule
\end{tabular}
\end{table}


\section{Requirements}

\subsection{General}

Students should attend and participate (obviously!). There will also
be occasional homework assignments, consisting of either (i) a p-set, or (ii)
submitting questions on an assigned reading. These will usually be due back the
day before the following class.

\subsection{Student presentations}

For the registered students, we expect two things:

\begin{itemize}

  \item A presentation in class.

  \item Eventually, a term paper.

\end{itemize}

We've blocked out the final two weeks of the seminar for student presentations,
but this is mutable. The presentation may take one of two forms: (a)
leading discussion of a paper relevant to the term paper topic,\sidenote{If you
  decide on this option, it may be desirable for you to present earlier in the
  semester, especially if the paper you'd like to discuss overlaps significantly
with a topic we plan to discuss. If you'd be interested in doing this, please
let us know.} (b)
presentation of WiP ideas that will eventually take the form of a term paper.
Just to emphasise, we don't expect the latter kind of presentation to be
especially polished or complete. Something more exploratory is fine. Given that
there are four registered students, each presentation should be around one hour
long.

Leading up to the presentations, we'll be in touch with registered students to
arrange semi-regular meetings.

\section{Readings}

The beginnings of a bibliography for this seminar:\sidenote{Note: readings aren't necessary until
  assigned. If you have any suggestions for papers on related topics we should
  look at, please let us know! You can shape the future of the seminar!!}

\subsection{Scope}

\begin{itemize}

\item \fullcite{barker2002}

\item \fullcite{barkerShan2015}

\item \fullcite{Charlowc}

\item \fullcite{shan2002}

\end{itemize}

\subsection{Crossover}

\begin{itemize}

    \item \fullcite{shanBarker2006}

  \item \fullcite{barkerShan2008}

  \item Chierchia, Gennaro. to appear. The origins of weak crossover: when dynamic semantics meets event semantics. \textit{Natural Language Semantics}.

\end{itemize}

\subsection{Haddock's puzzle and presupposition}

\begin{itemize}

    \item \fullcite{bumford2017}

    \item Champollion, Lucas and Sauerland, Uli. 2010. Move and accommodate: a solution to Haddock’s puzzle. \textit{Empirical issues in syntax and semantics} 8.

    \item \fullcite{grove2019}

\end{itemize}

% \printbibliography

\end{document}
