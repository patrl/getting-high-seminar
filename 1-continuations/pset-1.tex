\documentclass[nols,twoside,nofonts,nobib,nohyper]{tufte-handout}

\usepackage{fixltx2e}
\usepackage{tikz-cd}
\usepackage{tcolorbox}
\usepackage{appendix}
\usepackage{listings}
\lstset{language=TeX,
       frame=single,
       basicstyle=\ttfamily,
       captionpos=b,
       tabsize=4,
  }

\begin{acronym}
\acro{sfa}{Scopal Function Application}
\acro{fa}{Function Application}
\acro{wco}{Weak Crossover}
\acro{ScoT}{Scope Transparency}
\acro{vfs}{Variable Free Semantics}
\acro{acd}{Antecedent Contained Deletion}
\acro{qr}{Quantifier Raising}
\acro{doc}{Double Object Construction}
\end{acronym}

\renewcommand*{\acsfont}[1]{\textsc{#1}}


\makeatletter
% Paragraph indentation and separation for normal text
\renewcommand{\@tufte@reset@par}{%
  \setlength{\RaggedRightParindent}{0pt}%
  \setlength{\JustifyingParindent}{0pt}%
  \setlength{\parindent}{0pt}%
  \setlength{\parskip}{\baselineskip}%
}
\@tufte@reset@par

% Paragraph indentation and separation for marginal text
\renewcommand{\@tufte@margin@par}{%
  \setlength{\RaggedRightParindent}{0pt}%
  \setlength{\JustifyingParindent}{0pt}%
  \setlength{\parindent}{0pt}%
  \setlength{\parskip}{\baselineskip}%
}
\makeatother

\setcounter{secnumdepth}{3}

\title{p-set 1}

\author[Patrick D. Elliott and Martin Hackl]{Patrick~D. Elliott\sidenote{\texttt{pdell@mit.edu}} \& Martin Hackl\sidenote{\texttt{hackl@mit.edu}}}

\addbibresource[location=remote]{/home/patrl/repos/bibliography/elliott_mybib.bib}

\lingset{
  belowexskip=0pt,
  aboveglftskip=0pt,
  belowglpreambleskip=0pt,
  belowpreambleskip=0pt,
  interpartskip=0pt,
  extraglskip=0pt,
  Everyex={\parskip=0pt}
}


% \usepackage{booktabs} % book-quality tables
% \usepackage{units}    % non-stacked fractions and better unit spacing
% \usepackage{lipsum}   % filler text
% \usepackage{fancyvrb} % extended verbatim environments
%   \fvset{fontsize=\normalsize}% default font size for fancy-verbatim environments

% % Standardize command font styles and environments
% \newcommand{\doccmd}[1]{\texttt{\textbackslash#1}}% command name -- adds backslash automatically
% \newcommand{\docopt}[1]{\ensuremath{\langle}\textrm{\textit{#1}}\ensuremath{\rangle}}% optional command argument
% \newcommand{\docarg}[1]{\textrm{\textit{#1}}}% (required) command argument
% \newcommand{\docenv}[1]{\textsf{#1}}% environment name
% \newcommand{\docpkg}[1]{\texttt{#1}}% package name
% \newcommand{\doccls}[1]{\texttt{#1}}% document class name
% \newcommand{\docclsopt}[1]{\texttt{#1}}% document class option name
% \newenvironment{docspec}{\begin{quote}\noindent}{\end{quote}}% command specification environment

\begin{document}

\maketitle% this prints the handout title, author, and date

\textbf{Deadline: }02.13 (i.e., before next class)

\section{Warming up}

\ex
A philosopher has criticized most linguists.\hfill $\ml{most} > ∃$\label{ex:1}
\xe

\ex
Most linguists have read a paper by every German semanticist.\\
\phantom{,}\hfill
$∀ > ∃ > \ml{most}$
\xe


Give a derivation of the indicated readings of the examples above using:

\begin{itemize}

    \item Quantifier raising and predicate abstraction.\sidenote{I.e., in-line
    with \citet{heimKratzer1998} -- you should have covered this in semantics
    101. Don't worry about trace conversion, just treat traces of movement as variables.}

    \item Continuation semantics with \textit{flat lambda
    expressions}.\sidenote{No towers allowed! Make sure to be explicit about
    types, as well as any $β$-reductions and $α$-conversions necessary.}

   \item Continuation semantics with \textit{towers}.

\end{itemize}

\begin{tcolorbox}
  Bonus round
  \tcblower
  Can you come up with a general \textit{translation procedure} for going from
  a derivation using continuations to a derivation which makes use of quantifier
  raising? It might help to think about the role of \ml{LOWER} in continuation semantics.
\end{tcolorbox}

\section{Split scope}

Non upward-monotonic quantifiers give rise to so-called \textit{split scope}
readings across intensional verbs (Heim 2001).

\ex
The company need fire no employees.\\
\textit{It is not the case that the company is obligated to fire
  employees}\hfill (Potts 2000)
\xe

The split scope reading -- the one we're interested in -- entails a lack of
obligation for the company. It seems to involve a noun-phrase \textit{no
  employees} scoping in two different places at once.

\begin{itemize}

    \item Analyze this phenomenon using continuation semantics.\sidenote{Hint:
    think about towers with $n>2$ stories.}

\end{itemize}

\end{document}
