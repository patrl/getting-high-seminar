\documentclass[nols,twoside,nofonts,nobib,nohyper]{tufte-handout}

\usepackage{fixltx2e}
\usepackage{tikz-cd}
\usepackage{tcolorbox}
\usepackage{appendix}
\usepackage{listings}
\lstset{language=TeX,
       frame=single,
       basicstyle=\ttfamily,
       captionpos=b,
       tabsize=4,
  }

\begin{acronym}
\acro{sfa}{Scopal Function Application}
\acro{fa}{Function Application}
\acro{wco}{Weak Crossover}
\acro{ScoT}{Scope Transparency}
\acro{vfs}{Variable Free Semantics}
\acro{acd}{Antecedent Contained Deletion}
\acro{qr}{Quantifier Raising}
\acro{doc}{Double Object Construction}
\end{acronym}

\renewcommand*{\acsfont}[1]{\textsc{#1}}

\makeatletter
% Paragraph indentation and separation for normal text
\renewcommand{\@tufte@reset@par}{%
  \setlength{\RaggedRightParindent}{0pt}%
  \setlength{\JustifyingParindent}{0pt}%
  \setlength{\parindent}{0pt}%
  \setlength{\parskip}{\baselineskip}%
}
\@tufte@reset@par

% Paragraph indentation and separation for marginal text
\renewcommand{\@tufte@margin@par}{%
  \setlength{\RaggedRightParindent}{0pt}%
  \setlength{\JustifyingParindent}{0pt}%
  \setlength{\parindent}{0pt}%
  \setlength{\parskip}{\baselineskip}%
}
\makeatother

\setcounter{secnumdepth}{3}

\title{Continuation semantics i\thanks{24.979: Topics in
    semantics\\\noindent\textit{Getting high: Scope, projection, and evaluation order}}}

\author[Patrick D. Elliott and Martin Hackl]{Patrick~D. Elliott\sidenote{\texttt{pdell@mit.edu}} \& Martin Hackl\sidenote{\texttt{hackl@mit.edu}}}

\addbibresource[location=remote]{/home/patrl/repos/bibliography/elliott_mybib.bib}

\lingset{
  belowexskip=0pt,
  aboveglftskip=0pt,
  belowglpreambleskip=0pt,
  belowpreambleskip=0pt,
  interpartskip=0pt,
  extraglskip=0pt,
  Everyex={\parskip=0pt}
}


% \usepackage{booktabs} % book-quality tables
% \usepackage{units}    % non-stacked fractions and better unit spacing
% \usepackage{lipsum}   % filler text
% \usepackage{fancyvrb} % extended verbatim environments
%   \fvset{fontsize=\normalsize}% default font size for fancy-verbatim environments

% % Standardize command font styles and environments
% \newcommand{\doccmd}[1]{\texttt{\textbackslash#1}}% command name -- adds backslash automatically
% \newcommand{\docopt}[1]{\ensuremath{\langle}\textrm{\textit{#1}}\ensuremath{\rangle}}% optional command argument
% \newcommand{\docarg}[1]{\textrm{\textit{#1}}}% (required) command argument
% \newcommand{\docenv}[1]{\textsf{#1}}% environment name
% \newcommand{\docpkg}[1]{\texttt{#1}}% package name
% \newcommand{\doccls}[1]{\texttt{#1}}% document class name
% \newcommand{\docclsopt}[1]{\texttt{#1}}% document class option name
% \newenvironment{docspec}{\begin{quote}\noindent}{\end{quote}}% command specification environment

\begin{document}

\maketitle% this prints the handout title, author, and date


\begin{tcolorbox}
  Homework
  \tcblower
Please read \cite{barkerShan2015}, chapters 1 and 4; you can find a pdf on stellar. If you have
    any questions, feel free to send me them in advance of next
    week's class, and I'll do my best to address them.

There's also a brief problem set on stellar, due \textbf{next Thursday}. This
    is intended to get you comfortable working with continuations. Like
    any subsequent p-sets, it won't be graded.
\end{tcolorbox}

\section{Why bother with continuations?}

\textit{If you're a certain kind of person, the following answer should be
  sufficient:}\\
\textbf{playing with new toys is fun.}

\textit{The longer answer...}

\begin{itemize}

    \item In this seminar, we're going to be investigating mechanisms via which
    meanings \textit{get high}.

    \item To put this in a different way, expressions of natural language come
    with conventionalized meaning components. Often, meaning components end up
    interpreted in places that don't necessarily correspond to the pronounced
    position of the expression they're associated with.

  \item Special cases of this include, but are almost certainly not limited to:

    \begin{itemize}

        \item Scope-taking (the focus of the first part of this seminar).

        \item Presupposition projection (the focus of the latter part of this seminar).

        \item Other varieties of \enquote{projective} content (e.g.,
        expressives, appositives, etc.).

     \end{itemize}

    \item We have fairly well-established techniques for dealing with these
    phenomena, e.g., quantifier raising for scope
    (\citealt{may,heimKratzer1998}, etc.), trivalence for presupposition
    projection (Peters 1979, George 2010, etc.), and conventional implicature
    for expressives and appositives
    (\citealt{potts2005,mccready2010a,gutzmann2015}, etc.) -- which aren't without their problems, as we'll see.

    \item \textit{Continuations} provide a powerful abstraction for
    modeling meaning components \enquote{getting high} in a more general,
    uniform way.\sidenote{This goes beyond just \textit{meaning}, and can be
    generalized to, e.g., phonological and syntactic features.
    See \cite{elliott2019movement}.}

    \item This
    doesn't automatically make them preferable, but if we can get away with
    reducing our set of theoretical primitives while maintaining the same
    empirical coverage, we should at least pursue the possibility.

    \item Continuations were originally developed\sidenote{Or rather,
    \textit{discovered}. See, e.g., \citet{danvyFilinski1992} and
    \citet{wadler1994} for important foundational work on \textit{delimited
    continuations} -- the variety we'll be discussing in this course.} by computer scientists in order to account for
    computations that are, in some sense, \textit{delayed} until later on. As
    such, they're especially well-suited to modelling meaning components, the
    evaluation of which is \textit{delayed} until later in the derivation.

    \begin{itemize}

    \item See e.g., \citet{barkerShan2015} and related works for a rich
    literature devoted to applying continuations to \textit{scope taking}.

    \item \citet{grove2019} develops a theory of presupposition projection as a
        \textit{scopal} phenomenon, using continuations.

     \item \citet{deGroote2006} uses continuations to develop a
        \textit{compositional} dynamic semantics (another important reference is \citealt{Charlowc}).

     \item In recent work, I've used continuations to model non-local readings
        of expressive adjectives (\citealt{elliott-fuck}) and overt movement (\citealt{elliott2019movement})

    \end{itemize}

 \item One of the \textit{design features} of continuation semantics that is of
    special interest to us is its \textit{built-in} \textit{left-to-right} bias.

     \item One thing we'll be thinking about in some depth is how meaning that
     \enquote{gets high}, such as scope-takers, interact with anaphoric
     expressions and other dependees.

     \item As has been observed for quite some time, interaction between scope
     and anaphora exhibits a \textit{leftness bias}.

     \begin{fullwidth}
     \begin{multicols}{2}

     \ex
     Every student$&^i$'s mother adores their$_{i}$ advisor.
     \xe

     \columnbreak

     \ex~
     \ljudge{*}Their$_{i}$ advisor adores every student$^{i}$'s mother.
     \xe

     \end{multicols}
     \end{fullwidth}

     \item The contrast above is an example of a \textit{weak crossover} violation.

   \item Roughly, weak crossover can be stated as follows:

     \ex \acf{wco}\\
     Scope may feed binding \textit{unless} the bound expression
     \textit{precedes} the scope-taker.
     \xe

    \item We'd like to explore the possibility, tentatively, that weak crossover
    isn't just about quantificational scope, but parallel effects can be
    observed with projective meanings more generally.

    \item I'll give you a couple of examples here to whet your appetite,
    although we won't get back to this until quite a bit later in the course.

    \item First, I'd like to suggest that \textit{presupposition projection can
    feed anaphora}:

    \ex
    Daniel doesn't know Paul has a sister$^{x}$, although he has seen her$_{x}$.
    \xe

    \item Focus on the construal where the indefinite takes scope below the
    intensional verb. Nevertheless, anaphora is possible. The natural way to
    think about this is as presupposition projection feeding
    anaphora:\sidenote{A cautionary note: surprisingly, this doesn't follow from
    standard dynamic theories (such as
    \citealt{heim1982,beaver_presupposition_2001}), since presuppositions are
    themselves static. A natural move is to instead treat presuppositions as
    themselves dynamic statements; see Elliott \& Sudo (2019) for a theory with
    this character.}

    \ex
    \fbox{Paul has a sister} Daniel doesn't know \textcolor{gray}{Paul has a
    sister},\\
    although he has seen her.
    \xe

    \item Having established that presupposition projection can feed anaphora,
    let's see what happens when the pronoun is made to \textit{precede} the
    presupposition trigger:

    \ex
    \ljudge{\#}Her$_{x}$ boss doesn't know that Paul has a sister$^{x}$.
    \xe

    \item Again, focus on the construal where the indefinite takes scope below
    the intensional verb. Our example doesn't have the reading indicated. This is surprising
    given that (a) presuppositions project, (b) presupposition projection can
    feed anaphora.

    \item Our hunch is that, what is to blame here is a more general form of \ac{wco}
    -- \textit{projective meaning may feed anaphora unless the anaphor precedes
    the projector.}

    \item Continuations are sufficiently general, that we can use them to model
    a wide variety of phenomena. Since they come with a built in linear
    bias\sidenote{And, indeed, arguably one of the most empirically successful
    accounts of WCO is couched in terms of continuation semantics -- see
    \citet{shanBarker2006}. We'll discuss this in several weeks time.}, they
    seem like a prime candidate for modeling linear biases more generally.

\end{itemize}

\section{Plan}


\textit{This week:}

\begin{itemize}


    \item We're going to develop an understanding of \citeauthor{barkerShan2015}'s tower notation,
    and how to translate from towers to flat representations, and vice versa.

    \item We'll get a handle on continuation semantics' linear bias, encoded in the
    composition rules themselves.\sidenote{This will be necessary preparation for our
    discussion of \citeauthor{shanBarker2006}'s (\citeyear{shanBarker2006})
    theory of crossover.}

    \item We'll show how to account for scope ambiguities via a combination of
    \textit{lowering} and multi-story towers.

    \item Scope islands in terms of obligatory evaluation.

\end{itemize}

\textit{Next week:}

\begin{itemize}

    \item Using continuations to account for generalized con/dis-junction.

    \item Indexed continuations and a compositional semantics for determiners.

    \item Continuations and exceptionally scoping indefinites (\citealt{Charlowc}).

\end{itemize}

\textit{If we have time:}

\begin{itemize}
        \item Antecedent Contained Deletion.
        \item Extraposition and scope.
        \item Expressive adjectives.
    \end{itemize}


\section{Some notation conventions}\label{sec:notation}

Generally speaking, I'll be assuming \cite{heimKratzer1998} as background, but I'll depart from their notation slightly.

Expressions in the meta-language will be typeset in \textsf{sans serif}.

    $$\eval{[\textsubscript{DP} John]} ≔ \underbrace{\ml{John}}_{\text{individual}}$$

    Seeing as its primitive, we'll treat white-space as function
    application, e.g.:

    \ex
    $(λ x . \ml{left} x) \ml{paul} = \ml{left} \ml{paul}$
    \xe

  Function application associates to the \textit{left}:

    \ex
    $(λx . λ y . y \ml{likes} x) \ml{paul} \ml{sophie} ≡ ((λx . λ y . y \ml{likes} x) \ml{paul}) \ml{sophie}$
    \xe

    \begin{tcolorbox}
      Question
      \tcblower
      Can the following expression be reduced?

      \ex
      $(λ k . ∃z[k z]) (λ x . λ y . y  \ml{likes} x) \ml{Sam}$
      \xe

    \end{tcolorbox}

  We'll write \textit{types} in a \texttt{fixed width font}. We have our
    familiar primitive types...

    \ex
    $\mathtt{type} ≔ \type{e} | \type{t} | \type{s} | …$
    \xe

   ...and of course \textit{function types}. Unlike
    \citet{heimKratzer1998}, who use $⟨.⟩$ as the constructor for a function
    type, we'll be using the (more standard (outside of linguistics!)) arrow constructor $(→)$:

    \ex
    $\type{⟨e,t⟩} ≡ \type{e → t}$
    \xe

   The constructor for function types \textit{associates to the right}:

    \ex
    $\type{e → e → t} ≡ \type{e → (e → t)}$
    \xe

    \begin{tcolorbox}
      Question
      \tcblower
      Which of the following is the correct type for a quantificational
      determiner?

      \ex
      $\eval{every}:\type{(e → t) → (e → t) → t}$
      \xe

      \ex
      $\eval{every}:\type{e → t → (e → t) → t}$
      \xe

    \end{tcolorbox}

\end{itemize}

\section{The Partee triangle}

The purpose of this section is to show that continuation semantics is, in a
certain sense, \textit{already implicit} in the inventory of type-shifters
standardly assumed in formal semantics.

\ex
$\ml{LIFT} x ≔ λk  . k x$\hfill$\type{\ml{LIFT} : e → (e → t) → t}$
\xe

\ex~
$\ml{IDENT} x ≔ λ y . y = x$\hfill$\ml{IDENT} : \type{e → e → t}$
\xe

\ex~
$\ml{BE} Q ≔ λx . Q (λ y . y = x)$\hfill$\ml{BE} : \type{((e → t) → t) → e → t}$
\xe

Evidence for $\ml{IDENT}$:

\ex
This man is John.
\xe

Evidence for $\ml{BE}$:

\ex
This man is an authority on heavy metal.
\xe

We'll come back to \ml{LIFT} in a moment.

\begin{tcolorbox}
\textbf{Commutative diagrams}
\tcblower
The \textit{Partee triangle} is a \textit{commutative diagram}. We say that a diagram \textit{commutes} if, when there are multiple paths
between two points, those paths are equivalent. The equivalence in (\ref{tri1})
is therefore expressed by the triangle.
\end{tcolorbox}


\ex The Partee triangle\sidenote{\cite{partee1986}}\\
\begin{tikzcd}
  \type{e} \arrow[rr, "\ml{LIFT}"] \arrow[dr, "\ml{IDENT}"] & & \type{(e → t) → t}
  \arrow[dl, "\ml{BE}"]\\
  & \type{e → t} &
\end{tikzcd}\label{def:ptri}
\xe

\ex
$\ml{ident} ≡ \ml{BE} ∘ (↑)$\label{tri1}
\xe

\textsc{Observation:} quantificational and non-quantificational DPs
can be coordinated:

\ex
{}[Howie and a woman] entered the club\label{ex:conj1}
\xe

\ml{LIFT} allows something that, by virtue of its quantificational nature, is an
\textit{inherent} scope-taker, to combine with something that
\textit{isn't}:\sidenote{If you're familiar with \cite{parteeRooth} you'll notice
  that the \textit{and} that coordinates quantificational DPs (written here as
  \textit{and}$_{Q}$), is just the result of applying their \textit{generalized
    conjunction} rule. We'll return to generalized conjunction, and the
  connection to continuations next week.}

\ex
\begin{forest}
  [{$λ k . ∃x[\ml{woman} x ∧ k x] ∧ k \ml{Howie}$}
  [{$λk . k \ml{Howie}$} [{\ml{LIFT}} [{\ml{Howie}}]]]
  [{$λ Q' . λ k . ∃x[\ml{woman} x ∧ k x] ∧ Q' k$}
    [{$λQ . λ Q' . λ k . Q k ∧ Q' k$\\and$_{Q}$}]
    [{$λ k . ∃x[\ml{woman} x ∧ k x]$} [{a woman},roof]]
  ]
  ]
\end{forest}
\xe

\subsection{Generalizing the triangle}

Based on the way in which \ml{LIFT} and friends are defined, in theory
we could replace $\type{e}$ with \textit{any} type. Let's give a more general
statement of \ml{LIFT} and friends as polymorphic functions:

\ex
$\ml{LIFT} x ≔ λk  . k x$\hfill$\type{\ml{LIFT} : a → (a → t) → t}$
\xe

\ex~
$\ml{IDENT} x ≔ λ y . y = x$\hfill$\ml{IDENT} : \type{a → a → t}$
\xe

\ex~
$\ml{BE} Q ≔ λx . Q (λ y . y = x)$\hfill$\ml{BE} : \type{((a → t) → t) → a → t}$
\xe

The diagram, of course, still commutes:

\ex The (generalized) Partee triangle\\
\begin{tikzcd}
  \type{a} \arrow[rr, "\ml{LIFT}"] \arrow[dr, "\ml{IDENT}"] & & \type{(a → t) → t}
  \arrow[dl, "\ml{BE}"]\\
  & \type{a → t} &
\end{tikzcd}\label{def:ptri}
\xe

Why might a polymorphic \ml{LIFT} be useful? Recall that we used a typed
instantiation of \ml{LIFT} in order to allow a quantificational thing to combine
with a non-quantificational thing. Polymorphic \ml{LIFT} allows us to type-lift,
e.g., a function that takes multiple arguments:

\ex
\(\ml{LIFT} (λ xy . y \ml{hug} x) = \overbrace{λ k . k (λ
  xy . y \ml{hug} x)}^{\type{((e → e → t) → t) → t}}\)
\xe

One tantalizing possibility is that this allow us to combine a non-quantificational transitive verb with a
quantificational DP.

\ex
Howie \fbox{hugged a woman}
\xe

\ex~
\begin{forest}
  [{\xmark}
    [{$\type{((e → e → t) → t) → t}$\\$λ k . k (λ xy . y \ml{hug} x)$} [{\ml{LIFT}} [{$λ xy . y \ml{hug} x$}]]]
    [{$\type{(e → t) → t}$\\$λ k . ∃x[\ml{woman} x ∧ k x]$}]
  ]
  \end{forest}
\xe

Unfortunately, assuming the usual inventory of composition rules (i.e.,
\textit{function application}, \textit{predicate modification}, and
\textit{predicate abstraction}), we're stuck.\sidenote{Other, more exotic
  composition rules such as \textit{restrict} won't help either. Take my word
  for this!} So, \textbf{let's invent a new one}.

Here's the intuition we're going to pursue. Let's look again at the types. One
way of thinking about what \ml{LIFT} as follows: it takes an \type{a}-type thing
and adds a \enquote{wrapper}. Quantificational DPs, on the other hand come
\enquote{pre-wrapped}.

\begin{itemize}

    \item $\ml{LIFT} \eval{hug}: \type{(\fbox{(e → e → t)} → t) → t}$

    \item $\eval{a woman} : \type{(\fbox{e} → t) → t}$

\end{itemize}

If we look at the wrapped-up types, we see a \textit{function} from individuals,
and an individual -- namely, two things that can combine via function application.

What we need to accomplish is the following:

\begin{itemize}

  \item Unwrap lifted \textit{hug}.

  \item Unwrap \textit{a woman}.

  \item Use function application to combine the unwrapped values.

  \item Finally, wrap the result back up! Think of the quantificational meaning as being
    like a taco -- it isn't really a taco without the wrapper, therefore we
    don't want to throw the wrapper away.

\end{itemize}

    In order to accomplish this, we'll define a new composition rule:
    \acf{sfa}. We're going to define \ac{sfa} in terms of \acf{fa}; we haven't been
      explicit about how \ac{fa} is defined yet, so let's do that now. We'll
      write \ac{fa} as the infix operator
      \ml{A}.\sidenote{Here, function application is made bidirectional by \textit{overloading}
      -- we've defined forwards and backwards application, and given them the
      same function name.}

      \pex \acf{fa} (def.)
      \a $f \ml{A} x ≔ f x$\hfill$\ml{A}:\type{(a → b) → a → b}$
      \a $x \ml{A} f ≔ f x$\hfill$\ml{A}:\type{a → (a → b) → b}$
      \xe

      \begin{tcolorbox}

       \textbf{\acf{sfa}}

       \tcblower

       We'll write \ac{sfa} as the infix operator \ml{S}. Note that, since
       \ml{A} is overloaded already, and \ml{S} is defined in terms of \ml{A},
       \ml{S} gets overloaded too.

       \ex
       \acf{sfa} (def.)\\
       $m \ml{S} n ≔ λ k . m (λ a . n (λ b . k (a \ml{A} b)))$\\
       \phantom{,}\hfill$\ml{S}: \type{(((a → b) → t) → t) → ((a → t) → t) → (b
         → t) → t}$\\
       \phantom{,}\hfill$\ml{S}: \type{((a → t) → t) → (((a → b) → t) → t) → (b
         → t) → t}$
       \xe

      \end{tcolorbox}

      Now, let's illustrate how \ac{sfa} plus generalized \ml{LIFT} allows to
       compose a quantificational thing with non-quantificational things, using
       a simple example sentence:

              \ex
       Howie hugged some woman.
       \xe




       \begin{fullwidth}
         \begin{multicols}{2}
       \ex Step 1: compose \textit{some woman} with \ml{LIFT}-ed \textit{hug}.\\
       \begin{forest}
         [{$\type{((e → t) → t) → t}$\\\fbox{$λ k . ∃x[\ml{woman} x ∧ k (λ y . y \ml{hug} x)]$}}
         [{$\ml{S}$}
           [{$\type{((e → e → t) → t) → t}$\\$λ k . k (λ xy . y \ml{hug} x)$} [{\ml{LIFT}} [{hug}]]]
           [{$\type{(e → t) → t}$\\$λ k . ∃x[\ml{woman} x ∧ k x]$} [{some woman},roof]]
         ]]
       \end{forest}
       \xe
       \columnbreak
       \ex~ Step 2: compose the resulting VP-denotation with \ml{LIFT}-ed \textit{Howie}\\
       \begin{forest}
         [{$\type{(t → t) → t}$\\\fbox{$λ k . ∃x[\ml{woman} x ∧ k (\ml{Howie} \ml{hug} x])$}}
         [{$\ml{S}$}
           [{$\type{(e → t) → t}$\\$λ k . k \ml{Howie}$} [{$\ml{LIFT}$} [{Howie}]]]
           [{$\type{((e → t) → t) → t}$\\$λ k . ∃x[\ml{woman} x ∧ k (λ y . y \ml{hug} x)]$} [{hug some woman},roof]]
         ]]
       \end{forest}\label{ex:step2}
       \xe
     \end{multicols}
     \end{fullwidth}

       At this point, it's worth mentioning that we've \textbf{bootstrapped
         continuation semantics} from an independently motivated type-lifting
       operation, plus a natural composition rule \ac{sfa}; continuations aren't scary at all!

       The $λ k$ argument is standardly referred to as the \textit{continuation
         argument} in the derivations above. \ac{sfa} allows the continuation
       argument to get \enquote{passed up}.

       We're now tantalizingly close to deriving the right kind of object for
       the sentential meaning (namely, something of type $\type{t}$). Only, what
       we have is something of type $\type{(\fbox{t} → t) → t}$, i.e., a truth
       value in a \enquote{wrapper}. How do we get back the wrapped up value? We
       saturate the $k$ argument with the identity function. We'll call this
       operation $\ml{LOWER}$.\sidenote{Here we've given
           \ml{LOWER} the maximally polymorphic type compatible with the
           function definition; in fact, all we need is
           $(\ml{LOWER} : \type{((t → t) → t) → t})$.}

       \ex
       $\ml{LOWER}$ (def.)\\
       $\ml{LOWER} m ≔ m id$\hfill$\ml{LOWER}: \type{((a → a) → a) → a}$
       \xe

       Applying \ml{LOWER} to the final value in (\ref{ex:step2}) gives us a
       type \type{t} sentential meaning.

       \ex
       $\begin{aligned}[t]
         &\ml{LOWER} (λ k . k (∃x[\ml{Howie} \ml{hug} x]))\\
         &= (λ k . k (∃x[\ml{Howie} \ml{hug} x])) id& \text{by def.}\\
         &= id (∃x[\ml{Howie} \ml{hug} x]) & \text{reduce}\\
         &= ∃x[\ml{Howie} \ml{hug} x] & \text{reduce}
         \end{aligned}$
       \xe

       Let's zoom back out and see how all the pieces fit together by looking at
       the \textit{graph of the derivation}.

       \ex
       \begin{forest}
         [{$\type{t}$}
         [{$\ml{LOWER}$}
         [{$\type{(t → t) → t}$\\$\ml{S}$}
           [{$\type{(e → t) → t}$} [{$\ml{LIFT}$} [{Howie}]]]
           [{$\type{((e → t) → t) → t}$\\$\ml{S}$}
             [{$\type{((e → e → t) → t) → t}$} [{$\ml{LIFT}$} [{hug}]]]
             [{$\type{(e → t) → t}$} [{some woman},roof]]
           ]
         ]
         ]
         ]
       \end{forest}
       \xe

       So far, we've provided an account of how quantificational things compose
       with non-quantificational things, by making use of...

       \begin{itemize}

           \item ...an independently
           motivated type-shifting rule (\ml{LIFT})...

           \item ...a way to apply \ml{LIFT}-ed
           values (\ml{S})...

          \item ...and a way to get an ordinary value back from a
       \ml{LIFT}-ed value (\ml{LOWER}).

           \end{itemize}

           This seems pretty nice, but as I'm sure you've noticed, things are
           quickly going to get pretty cumbersome with more complicated
           sentences, especially with multiple quantifiers. Before we go any
           further, let's introduce some notational conveniences.

\section{Towers}

We've been using the metaphor of a \textit{wrapper} for thinking about what
$\ml{LIFT}$ does to an ordinary semantic value. Let's make this a bit more
transparent by introducing a new \textit{type constructor} for \ml{LIFT}-ed
values.\sidenote{A \textit{type constructor} is just a
   function from a type to a new type -- here, it's a rule for taking any type
   $\ml{a}$ and returning the type of the corresponding \ml{LIFT}-ed value.}

\ex
\type{K_{t} a ≔ (a → t) → t}
\xe


\begin{itemize}

\item Quantificational DPs are therefore of type $\type{{K}_{t} e}$ (inherently).

\item \ml{LIFT} takes something of type \type{a}, and lifts it into
    something of type $\type{K_{t} a}$.

\end{itemize}

Rather than dealing with \textit{flat} expressions of the simply-typed lambda
calculus, which will become increasingly difficult to reason about, we'll follow
\cite{barkerShan2015} in using \textit{tower notation}.\sidenote{To my mind, one
of \citeauthor{barkerShan2015}'s central achievements is simply the introduction
of an accessible notational convention for reasoning about the kinds of lifted
meanings we're using here.}\sidenote{It's important to bear in mind that towers
are just \textit{syntactic sugar} for flat lambda expressions; we should always
be able to translate back from towers to lambda expressions. Towers have no
privileged theoretical status, unlike, e.g., Discourse Representation
Structures, but are merely abbreviations.}

Let's look again at the meaning of a quantificational DP. The $k$ argument which
acts as the \textit{wrapper} is called the \textit{continuation argument}.

\ex Ordinary quantifier meanings:\\
$\eval{some woman} ≔ λ k . ∃x[\ml{woman} x ∧ k x]$
\xe

\ex~ Quantifiers using tower notation:\\
$\eval{some woman} ≔ \semtower{∃x[\ml{woman} x ∧ []]}{x}$
\xe

\ex~ Lifted meanings using tower notation:\\
$\ml{LIFT} (\eval{hug}) = \semtower{[]}{λ xy . y \ml{hug} x}$
\xe

In general:

\ex Tower notation (def.)\\
$\semtower{f []}{x} ≔ λ k . f (k x)$
\xe

We can use tower notation for types too:

\ex Tower types (def.)\\
$\semtower{\type{b}}{\type{a}} ≔ \type{(a → b) → b}$\hfill$≡ \type{K_{b} a}$
\xe

We can now redefine our type constructor $\type{K_{t}}$, and our type-shifting
operations using our new, much more concise, tower notation. These will be our
canonical definitions from now on. We'll also start abbreviating a \ml{LIFT}-ed
value $a$ as
$a^{↑}$ and a \ml{LOWER}-ed value $b$ as $b^{↓}$.

\ex The continuation type constructor \type{K_t} (def.)\\
$\type{K_{t} a} ≔ \type{\semtower{t}{a}}$
\xe

\ex~
\ml{LIFT} (def.)\sidenote{Thinking in terms of towers, \ml{LIFT} takes a value
  $a$ and returns a \enquote{trivial} tower, i.e., a tower with an empty top-story.}\\
$a^{↑} ≔ \semtower{[]}{a}$\hfill$(↑):\type{a → K_{t} a}$
\xe

\ex~
\acf{sfa} (def.)\sidenote{\ac{sfa} takes two scopal values -- one with a
  function on the bottom floor, and the other with an argument on the bottom
  floor -- and combines them by (i) doing function application on the bottom
  floor, and (ii) \textit{sequencing} the scope-takers.}\\
$\semtower{f []}{x} \ml{S} \semtower{g []}{y} ≔
\semtower{f (g [])}{x \ml{A} y}$\hfill$\ml{S}:\type{K_{t} (a → b) → K_{t} a →
  K_{t} b}$
\xe

\ex~
\ml{LOWER} (def.)\footnote{\ml{LOWER} \textit{collapses the tower}, by saturating
  the continuation argument with the identity function.}\\
$\left(\semtower{f []}{p}\right)^{↓} = f p$\hfill$(↓) : \type{K_{t} t → t}$
\xe

In order to see the tower notation in action, let's go through an example
involving multiple quantifiers, and show how continuation semantics derives the
surface scope reading:

\ex
Some knight chased every thief.\hfill$∃ > ∀$
\xe

\begin{fullwidth}
  \begin{multicols}{2}

First, we combine \textit{every thief} with lifted \textit{chase} via \ac{sfa}:

\ex
\begin{forest}
  [{\fbox{$\semtower{∀ x[\ml{thief} x → []]}{λ y . y \ml{chase} x$}}\\\ml{S}}
    [{$\semtower{[]}{λ xy. y \ml{chase} x}$}]
    [{$\semtower{∀x[\ml{thief} x → []]}{x}$}]
  ]
\end{forest}
\xe

\columnbreak

Next, the (boxed) VP value combines with \textit{some knight} via \ac{sfa}:

\ex
\begin{forest}
  [{\fbox{$\semtower{∃y[\ml{knight} y ∧ ∀ x[\ml{thief} x → []]]}{y \ml{chase} x}$}\\\ml{S}}
    [{$\semtower{∃y[\ml{knight} y ∧ []]}{y}$}]
    [{$\semtower{∀ x[\ml{thief} x → []]}{λ y . y \ml{chase} x$}} [{chase every thief},roof]]
  ]
\end{forest}
\xe

\end{multicols}
\end{fullwidth}

Finally, the resulting tower is collapsed via \textit{lower}:

\ex
\begin{forest}
[{\fbox{$∃y[\ml{knight} y ∧ ∀ x[\ml{thief} x → y \ml{chase} x]]$}} [{$\left(\semtower{∃y[\ml{knight} y ∧ ∀ x[\ml{thief} x → []]]}{y \ml{chase} x}\right)^{↓}$}]]
\end{forest}
\xe

Let's zoom out and look at the \textit{graph of the syntactic derivation}
alongside the \textit{graph of the semantic derivation}.

\begin{multicols}{2}

  \begin{forest}
    [{\textsc{merge}}
      [{DP} [{some knight},roof]]
      [{\textsc{merge}}
        [{V\\chase}]
        [{DP} [{every thief},roof]]
      ]
    ]
  \end{forest}

  \columnbreak

  \begin{forest}
    [{\type{t}}
    [{\ml{LOWER}}
    [{\typetower{t}{t}\\\ml{S}}
      [{\typetower{t}{e}\\some knight}]
      [{\typetower{t}{e → t}\\\ml{S}}
        [{\typetower{t}{e → e → t}} [{\ml{LIFT}} [{chase}]]]
        [{\typetower{t}{e}\\every thief}]
      ]
    ]
    ]]
  \end{forest}

  \columnbreak

\end{multicols}

Type-shifting operations are interleaved with syntactic operations. In order to
restrict this system, it seems natural to place an economy constraint on
$\ml{S}$ and $\ml{LIFT}$.\sidenote{This is very tentative. Working out exactly
  how the system outlined here should be restricted in terms of economy would
  make a great student project.}

\ex
Economy condition on type-shifting\\
\textsc{Merge} in the syntax is interpreted as \ml{A/S} in the semantics, whichever
is well-typed. \ml{LIFT} may apply freely to the extent that it is necessary for
the derivation to proceed.
\xe

\section{Scopal ambiguities}

\subsection{Interaction between scope-takers and other operators}

Right now, we have a theory which is very good at deriving the surface scope
reading of a sentence with multiple quantifiers. It can also derive some
ambiguities that arise due to interactions of quantifiers and scopally \textit{immobile}
expressions, such as intensional predicates. Consider, e.g., the interaction
between a universal quantifier and the desire verb \textit{want}.

\ex
Daniele wants to join every group chat.\label{ex:dani}\hfill $\ml{want} > ∀;∀ > \ml{want}$
\xe

(\ref{ex:dani}) is ambiguous: (i) if \textit{every} takes scope below
\textit{want}, it is true if Dani has a desire about joining every group chat,
(ii) if \textit{every} takes scope over \textit{want}, it is true if every group
chat is s.t. Dani has a desire to join in.

We actually already have everything we need in order to account for this. In a
nutshell, the two readings correspond to an application of \ml{LOWER} either
below or above the intensional predicate.\sidenote{For a fully-fledged treatment
of the examples below, we would of course need to systematically replace
$\type{t}$ in our fragment to some intensional type.}

\ex Daniele wants to join every group chat.\hfill$\ml{want} > ∀$
\begin{forest}
  [{...}
    [{Daniele}]
    [{$λ y . y \ml{wants} (∀x[\ml{groupChat} x→ (\ml{pro join }x)])$}
      [{wants}]
      [{$∀x[\ml{groupChat} x→ (\ml{pro join }x)]$}
      [{$\ml{LOWER}$}
      [{$\semtower{∀x[\ml{groupChat} x → []]}{\ml{pro} \ml{join} x}$\\\ml{S}}
        [{PRO$^{↑}$}]
        [{$\semtower{∀x[\ml{groupChat} x → []]}{λ y . y \ml{join} x}$\\\ml{S}}
          [{join$^{↑}$}]
          [{$\semtower{∀x[\ml{groupChat} x → []]}{x}$} [{every group chat},roof]]
        ]
      ]
      ]
      ]
    ]
  ]
\end{forest}
\xe

\ex~
Daniele wants to join every group chat.\hfill $∀ > \ml{want}$
\begin{forest}
  [{$∀x[\ml{groupChat} x → \ml{Dani wants}(\ml{pro join }x)]$}
  [{\ml{LOWER}}
    [{$\semtower{∀x[\ml{groupChat}  \rightarrow []]}{\ml{Dani wants} (\ml{pro} \ml{join} x)}$}
      [{Daniele}]
      [{...}
        [{wants}]
        [{...} [{PRO to join every group chat},roof]]
    ]
  ]
  ]]
\end{forest}
\xe

We can schematize what's going on here by boxing the point of the derivation at
which \ml{LOWER} applies:

\begin{itemize}

    \item $\ml{Daniele} (\ml{wants} \fbox{\ml{pro}^{↑} \ml{S} (\ml{join}^{↑} \ml{S} \ml{everyGroupChat})}^{↓})$

    \item $\fbox{\ml{Daniele}^↑ \ml{S} (\ml{wants}^↑ \ml{S} (\ml{pro}^↑ \ml{S} (\ml{join}^↑ \ml{S} \ml{everyGroupChat})))}^{↓}$

\end{itemize}

If you're more familiar with a treatment of scope-taking in terms of quantifier
raising, then you can think of \ml{LOWER} as being the correlate of the landing
site of QR; semantic composition proceeds via \ml{S} up until we encounter the
landing site, at which point we switch back to \enquote{vanilla} semantic
composition via \ml{A}.

\subsection{Scope rigidity}

As we've seen however, when we're dealing with multiple scopally \textit{mobile}
expressions (such as quantifiers), continuation semantics derives surface scope
readings by default. It is, therefore, well-suited
to languages such as German and Japanese, which have been argued to display
\textit{scope-rigidity} (modulo semantic reconstruction amongst other potential exceptions).

The following example is from Kuroda (1970).

\pex
\a\begingl
\gla Dareka-ga subete-no hon-o yonda.//
\glb someone-{\sc nom} all-{\sc gen} book-{\sc acc} read.//
\glft \enquote{Someone read all the books.}\hfill $∃ > ∀; \text{\xmark} ∀ > ∃$//
\endgl
\a\begingl
\gla Subete-no hon-o dareka-ga yonda.//
\glb all-{\sc gen} book-{\sc acc} someone-{\sc nom} read.//
\glft \enquote{Someone read all the books}\hfill $∀ > ∃; ∃ > ∀$//
\endgl
\xe

There are also, of course, famous environments in English where we observe
apparent scope-rigidity, such as the double-object construction (scope rigidity
in the double-object construction is usually described as \textit{scope freezing}).

\pex
\a Daniele sent a syntactician every picture.\hfill $∃ > ∀; \text{\xmark} ∀ > ∃$
\a Daniele sent a picture to every syntactician.\hfill $∃ > ∀; ∀ > ∃$
\xe

\citet{bobaljikWurmbrand2012} posit a (violable) economy condition in order to
express the preference for surface scope observed in many languages.
\citeauthor{bobaljikWurmbrand2012} make architectural assumptions that we aren't
necessarily committed to here, but the spirit of \ac{ScoT} is very much in line
with a continuation semantics for quantifiers, where surface scope is the
default, and inverse scope will only be achievable via additional application of
the type-shifting rules posited.

\ex
\acf{ScoT}\\
If the order of two elements at LF is $A>B$, then the order at PF is $A>B$.
\xe

We still need to account for the availability of \textit{inverse scope readings}
in English, and other languages however. We'll do this using \textit{multi-story
towers}.

% \subsection{Deriving inverse scope}

% Recall that \(\ml{LIFT}\) is a \textit{polymorphic function} -- it lifts a value
% into a trivial tower:

% \ex
% $a^{↑} ≔ \semtower{[]}{a}$
% \xe

% Since \ml{LIFT} is polymorphic, in principle it can apply to any kind of value
% -- even a tower! Let's flip back to lambda notation to see what happens.

% \ex
% $\eval{everyone} ≔ λ k . ∀x[k x]$
% \xe

% \ex~
% $\eval{everyone}^{↑} = λ l . l (λ k . ∀x[k x])$
% \xe

% \begin{tcolorbox}
%   Question
%   \tcblower
%   What is the \textit{type} of lifted \textit{everyone}?
% \end{tcolorbox}

% Going back to tower notation, lifting a tower adds a trivial third
% story:\sidenote{In fact, via successive application of \ml{LIFT}, we can
%   generate an $n-$story tower.} Following \citet{Charlowc}, when we apply
% \ml{LIFT} to a tower, we'll describe the operation as \textit{external lift}
% (although, it's worth bearing in mind that this is really just our original
% \ml{LIFT} function).


% One important thing to note is that, when we externally lift a tower, the
% quantificational part of the meaning always remains on the same story relative
% to the bottom floor. Intuitively, this reflects the fact that, ultimately,
% \ml{LIFT} alone isn't going to be enough to derive quantifier scope ambiguities.

% \ex
% \(\left(\semtower{∀x[]}{x}\right)^{↑} = \semtower{[]}{\semtower{∀x[]}{x}}\)
% \xe

% \begin{tcolorbox}
% \textbf{Question}
% \tcblower
% Which (if any) of the following bracketings make sense for a three-story tower:

% \begin{multicols}{2}
% \ex
% $\semtower{\left(\semtower{f []}{g []}\right)}{x}$
% \xe
% \columnbreak
% \ex
% $\semtower{f []}{\left(\semtower{g []}{x}\right)}$
% \xe
% \end{multicols}
% \end{tcolorbox}

% The extra ingredient we're going to need, is the ability to sandwhich an empty
% story into the \textit{middle} of our tower, pushing the quantificational part
% of the meaning to the very top. This is \textit{internal lift} ($⇈$).\sidenote{I
% can tell what you're thinking: \enquote{seriously? Another \textit{darn}
%   type-shifter? How many of these are we going to need?!}.
% Don't worry, I got you. Even thought we've defined internal lift here as a
% primitive operation, it actually just follows from our existing machinery.
% Concretely, \textit{internal lift} is really just \textit{lifted} \ml{LIFT} (so
% many lifts!). Lifted \ml{LIFT} applies to its argument via \ml{S}.

% \ex
% $(\ml{LIFT}^{↑}) \ml{S} \semtower{f []}{x} = \semtower{f []}{\semtower{[]}{x}}$
% \xe

% }

% \pex
% \textit{Internal lift} (def.)\\
% \a \((⇈) : \type{K_{t} a → K_{t} (K_{t} a)}\)
% \a \(m^{⇈} ≔ λ k . m (λ x . k x^{↑})\)
% \xe

% It's much easier to see what internal lift is doing by using the tower notation.
% We can also handily compare its effects to those of \textit{external} lift.

% \begin{multicols}{2}
% \ex \textit{Internal lift} (tower ver.)\\
% \(\left(\semtower{f []}{x}}\right)^{⇈} ≔ \semtower{f []}{\semtower{[]}{x}}\)
% \xe
% \columnbreak
% \ex \textit{External lift} (tower ver.)\\
% \(\left({\semtower{f []}{x}}\right)^{↑} ≔ \semtower{[]}{\semtower{f []}{x}}\)
% \xe

% \end{multicols}

% Armed with \textit{internal} and \textit{external} lifting operations, we now
% have everything we need to derive inverse scope. We'll start with a simple
% example (\ref{ex:classic1}).

% The trick is: we \textit{internally} lift the quantifier that is destined to
% take wide scope.

% \ex
% A boy danced with every girl.\hfill $∀ > ∃$\label{ex:classic1}
% \xe

% Before we proceed, we need to generalize \ml{LIFT} and \ac{sfa} to three-story
% towers.\sidenote{
% Before you get worries about expanding our set of primitive operations, notice
% that \textit{3-story lift} is just ordinary lift applied twice. \textit{3-story}
% \ac{sfa} is just \ac{sfa}, but where the bottom story combines via \ml{S} not
% \ml{A}. In fact, we can generalize these operations to $n-$story towers.
% }

% \begin{multicols}{2}
% \ex
% $x^{↑_{2}} ≔ \semtower{[]}{\semtower{[]}{x}}$
% \xe

% \columnbreak

% \ex
% $\semtower{f []}{m} \ml{S}_{2} \semtower{g []}{n} ≔ \semtower{f (g [])}{m \ml{S} n}$
% \xe

% \end{multicols}


% \begin{fullwidth}
%   \begin{multicols}{2}
% \ex Step 1: internally lift \textit{every girl} \\
% \begin{forest}
%   [{$\ml{S}_{2}$}
%     [{$\semtower{[]}{\semtower{[]}{\ml{danceWith}}}$} [{dance-with$^{↑_{2}}$}]]
%     [{$\semtower{∀x[]}{\semtower{[]}{x}}$} [{$⇈$} [{every girl},roof]]]
%   ]
% \end{forest}
% \xe

% \columnbreak

% \ex
% Step 2: \textit{ex}ternally lift \textit{a boy}\\
% \begin{forest}
%   [{$\ml{S}_{2}$}
%     [{$\semtower{[]}{\semtower{∃y[\ml{boy} y ∧ []]}{y}}$} [{a boy$^{↑}$}]]
%     [{$\semtower{∀x[\ml{girl} x → []]}{\semtower{[]}{λ y . y \ml{danceWith} x}}$} [{dance with every girl},roof]]
%   ]
% \end{forest}
% \xe
% \end{multicols}
% \end{fullwidth}

% What we're left with now is a 3-story tower with the universal on the top story
% and the existential on the middle story. We can collapse the tower by first
% collapsing the bottom two stories, and then collapsing the result. In order to
% do this, we'll first define \textit{internal lower}.\sidenote{Let's again
%   address the issue of expanding our set of primitive operations (in what is
%   becoming something of a theme). Internal lower is just lifted lower, applying
%   via \ml{S}. In other words:

%   \ex
%   $m^{⇊} ≡ (↓)^{↑} \ml{S} m$
%   \xe

% }

% \ex
% \textit{Internal lower} (def.)
% $\left(\semtower{f []}{\semtower{g []}{p}}\right)^{⇊} ≔ \semtower{f []}{\left(\semtower{g []}{x}\right)^{↓}}$
% \xe

% Now we can collapse the tower by doing \textit{internal lower}, followed by
% \textit{lower}:

% \ex
% \begin{forest}
%   [{\fbox{$∀x[\ml{girl} x → (∃y[\ml{boy} y ∧ y \ml{danceWith} x])]$}}
%   [{$↓$}
%     [{$\semtower{∀x[\ml{girl} x → []]}{∃ x[\ml{boy} x ∧ y \ml{danceWith} x]}$}
%       [{$⇊$}
%         [{$\semtower{∀x[\ml{girl} x → []]}{\semtower{∃y[\ml{boy} y ∧ []]}{y \ml{danceWith} x}}$} [{a boy danced with every girl},roof]]
%   ]]]]
% \end{forest}
% \xe

% Great! We've shown how to achieve quantifier scope ambiguities using our new
% framework. Let's look at the derivations again side-by-side.

% \begin{fullwidth}
% \begin{multicols}{2}

%   \ex
%   Surface scope (schematic derivation)\\
%   \begin{forest}
%     [{$↓$}
%     [{$\ml{S}$}
%       [{$Q_{1}$}]
%       [{$\ml{S}$}
%         [{$R^{↑}$}]
%         [{$Q_{2}$}]
%       ]
%     ]]
%   \end{forest}
%   \xe

%   \columnbreak

%   \ex
%   Inverse scope (schematic derivation)\\
%   \begin{forest}
%     [{$↓$}
%     [{$⇊$}
%   [{$\ml{S}_{2}$}
%     [{$Q_{1}^{↑}$}]
%     [{$\ml{S}_{2}$}
%       [{$R^{↑_{2}}$}]
%       [{$Q_{2}^{⇈}$}]
%     ]
%   ]]]
%   \end{forest}
%   \xe

% \end{multicols}
% \end{fullwidth}

% There's a couple of interesting things to note here:

% \begin{itemize}

%     \item The inverse scope derivation involves more applications of our
%     type-shifting operations -- this becomes especially clear if we decompose
%     the complex operations
%     $\ml{S}_{2}$, $↑_{2}$, $⇈$, and $⇊$.

%     \item In order to derive an inverse scope reading, what was \textit{crucial}
%     was the availability of \textit{internal lift}; the remaining operations,
%     $\ml{S}_{2}, ↑_{2}, ⇊$ only functioned to massage composition for
%     three-story towers.

% \end{itemize}

% On the latter point, it's tempting to conjecture that in, e.g., German, Japanese
% and other languages which \enquote{wear their LF on their sleeve}, the semantic
% correlate of \textit{scrambling} is \textit{internal lift}, whereas in
% scope-flexible languages such as English, internal lift is a freely available
% operation.\sidenote{To make sense of this, we would of course need to say
%   something more concrete about the relationship between syntax and semantics.
%   For an attempt at marrying continuations to a standard, minimalist syntactic
% component, see my manuscript \textit{Movement as higher-order structure building}.}

% If we adopt some version of the \textit{derivational complexity hypothesis}, we
% also predict that inverse scope readings should take longer to process than
% surface scope readings. This is something Martin may discuss is a couple of
% weeks time.

% It's worth mentioning, incidentally, that although we collapsed the resulting
% three-story tower via internal lower followed by lower, we can also define an
% operation that collapses a three-story tower two an ordinary tower in a
% different way. Let's call it \textit{join}:\sidenote{Join for three-story towers
% corresponds directly to the \textit{join} function associated with the
% continuation monad. For more on continuations from a categorical perspective,
% see the first appendix.}

% \ex \textit{join} (def.)\\
% $m^{μ} ≔ λ k . m (λ c . c k)$\hfill$μ: \type{K_{t} (K_{t} a) → K_{t} a}$
% \xe

% In tower terms, join takes a three-story tower and sequences quantifiers from
% top to bottom:

%  \ex
%   $
%   \left(\semtower{f []}{\semtower{g []}{x}}\right)^{μ} = \semtower{f (g [])}{x}
%   $
%   \xe

% Doing internal lower on a three-story tower followed by lower is equivalent to doing
% join on a three-story tower followed by lower (as an exercise, convince yourself
% of this). However, there's may be a good empirical reason for having internal lower as a
% distinct operation (and since it's just lifted lower, it \enquote{comes for
%   free} in a certain sense).

% \ex
% Daniele wants a boy to dance with every girl.\hfill $∀ > \ml{want} > ∃$\label{ex:dani1}
% \xe

% Arguably, (\ref{ex:dani1}) can be true if for every girl $x$, Daniele has the following
% desire: \textit{a boy dances with $y$}. This is the reading on which
% \textit{every boy} scopes over the intensional verb, and \textit{a boy} scopes
% below it.

% If we have \textit{internal lower}, getting this is easy. We \textit{internally
%   lift} \textit{every girl} and \textit{externally lift} \textit{a boy}. Before
% we reach the intensional verb, we fix the scope of \textit{a boy} by doing
% internal lower. Now we have an ordinary tower, and we can defer fixing the scope
% of \textit{every girl} via \textit{lower} until after the intensional verb.

% If we only have \textit{join} then the scope of \textit{a boy} and \textit{every
% girl} may vary amongst themselves, but they should either both scope below
% \textit{want} or both scope above \textit{want}.



% \section{Scope islands and obligatory evaluation}

% Inspired by research on \textit{delimited control} in computer
% science\sidenote{See, e.g., \cite{danvyFilinski1992} and \cite{wadler1994}.},
% \citet{Charlowc} develops an interesting take on scope islands couched in terms
% of continuations.

% He proposes the following definition:

% \ex
% \textit{Scope islands} (def.)\\
% A \textit{scope island} is a constituent that is subject to \textit{obligatory
%   evaluation}.\\
% \phantom{,}\hfill\citep[p. 90]{Charlowc}
% \xe

% By \textit{obligatory evaluation}, we mean that every continuation argument
% \textit{must} be saturated before semantic computation can proceed. In other
% words, a scope island is a constituent where, if we have something of type
% $\type{K_{t} a}$, we cannot proceed.

% One way of thinking about this, is that the presence of an unsaturated
% continuation argument means that there is some computation that is being
% deferred until later. Scope islands are constituents at which evaluation is
% \textit{forced}. As noted by \citeauthor{Charlowc}, this idea bears an
% intriguing similarity to \citeauthor{chomskyPhase}'s notion of a
% \textit{phase}.\sidenote{Exploring this parallel in greater depth could make for
% an interesting term paper topic.}

% How does this work in practice? A great deal of ink has been spilled arguing
% that, e.g., a finite clause is a scope island.

% \ex
% A boy said $\overbrace{\text{\fbox{that Susan greeted every
%       linguist}}}^{\text{scope island}}$.\hfill$∃>∀; \text{\xmark} ∀ > ∃$
% \xe

% The derivation of the embedded clause proceeds as usual via lift and \ac{sfa}.

% \newpage

% \begin{fullwidth}
% \begin{multicols}{2}
% \ex
% Scope island with an unevaluated type\\
% \begin{forest}
%   [{...}
%     [{...} [{a boy},roof]]
%     [{...}
%       [{said}]
%       [{\xmark $\semtower{∀x[\ml{linguist} x → []]}{\ml{Susan greeted }x}$} [{Susan greeted every linguist},roof]]
%     ]
%   ]
% \end{forest}
% \xe
% \columnbreak
% \ex
% Scope island with an evaluated type\\
% \begin{forest}
%   [{...}
%     [{...} [{a boy},roof]]
%     [{...}
%       [{said}]
%       [{\cmark $∀x[\ml{linguist} x→ \ml{Susan greeted }x]$}
%       [{$↓$} [{$\semtower{∀x[\ml{linguist} x → []]}{\ml{Susan greeted }x}$} [{Susan greeted every linguist},roof]]]]
%     ]
%   ]
% \end{forest}
% \xe
% \end{multicols}
% \end{fullwidth}

% This story leaves a lot of questions unanswered of course:

% \begin{itemize}

%     \item Is this just a recapitulation of a representational constraint on
%     quantifier raising?\sidenote{The answer to this question may ultimate be
%     \textit{yes}, in my view.}

%     \item Can we give a principled story about islands for overt movement using
%     similar mechanisms? What explains the difference between overt movement and
%     scope taking with respect to locality?\footnote{If we want to give a more
%     general account of phases using this mechanism, we need to give an account
%     of overt movement in terms of continuations, too. See my unpublished ms.
%     \textit{Movement as higher-order structure building} for progress in this direction.}

% \end{itemize}

% One interesting property of this theory of scope islands is that it goes
% some way towards explaining why certain expressions, such as indefinites, are able to take
% \textit{exceptional} scope, given certain assumptions about their semantics.

% \citet{Charlowc} shows that, if we conceive of indefinites as inducing
% alternatives, scope islands may be obviated by, in a certain sense, pied-piping
% the scope islands. This strategy won't work for ordinary quantifiers
% however.\sidenote{See \citet{demirok2019} for a parallel story couched in more
%   familiar \citeauthor{heimKratzer1998}-esque machinery.}

% Time permitting, we'll discuss how \citeauthor{Charlowc} accounts for the
% exceptional scope-taking properties of indefinites using continuations, either
% next week, or in our discussion of crossover phenomena.

\section{Next week}

Hopefully you're starting to get a feel for what continuations can accomplish
with very few primitives. Next week, we'll pick up where we left off, with a
discussion of:

\begin{itemize}

  \item Inverse scope.

  \item Generalized con/dis-junction.

  \item \textit{Indexed} continuations and a semantics for determiners.

  \item Continuations and exceptional scope

\end{itemize}

After that, Martin will take over with a re-assessment of QR.

\printbibliography

\begin{appendices}

  \section{Continuations from a categorical perspective}

  The way we've presented continuation semantics here makes crucial use of three
  building blocks:

  \begin{itemize}

      \item A type
  constructor \type{K_t} -- a way of getting from any type \type{a} to an
      enriched type-space characterizing \textit{continuized} values.

      \item A function $\ml{LIFT} (↑)$, i.e., a way of lifting a value $a$ into a
      \textit{trivial} inhabitant of our enriched type-space.

      \item A composition rule \acf{sfa}, i.e., an
  instruction for how to do function application in our enriched type-space.

      \end{itemize}

  There's a rich literature in category theory and (derivatively) functional
  programming on how to characterize this kind of construct, together with
  law-like properties its components should satisfy in order to qualify as
  \enquote{natural}. In fact, as discussed in, e.g.,
  \cite{charlow2018,elliott2019applicatives}, exactly this kind of general
  construct is \textit{implicit} in a great deal of semantic theory, including
  for example theories of pronouns and binding, and theories of focus.

  Formally, the triple $(\type{K_{t}},↑,\ml{S})$ is a special case of an \textit{applicative functor}, a
  highly influential notion in the literature on functional programming
  (\citealt{mcbridePaterson2008}); for applications in linguistic semantics see
  \cite{kiselyov2017}, \cite{charlow2018}, and \cite{elliott2019applicatives}.

  An applicative functor consists of three components: a \textit{type
    constructor} \type{F}, a way of lifting an inhabitant of $\type{a}$ into an
  inhabitant of $\type{F a}$, called $η$, and a way of doing \textit{function
    application} in the enriched type-space $\type{F a}$, called
  $⊛$.\footnote{If you want pronounceable names for these things, $η$ is called
    \textit{pure} in haskell, and $⊛$ is called \textit{ap} (short for
    \textit{application}).} The components of the applicative functor are
  additionally subject to a number of laws. I give the full definition below:

  \pex \textit{Applicative functor} (def.)\\
  An \textit{applicative} functor consists of the following three components
  subject to \textbf{homomorphism} \textbf{identity}, \textbf{interchange}, and
  \textbf{composition} laws:
  \a  $F: \type{type → type}$
  \a $η:\type{a → F a}$
  \a $⊛: \type{F (a → b) → F a → F b}$
  \xe

  \begin{multicols}{2}

    \ex
    \textbf{Homomorphism}\\
    $f^{η} ⊛ x^{η} ≡ (f x)^{η}$
    \xe

    \ex~
    \textbf{Interchange}\\
    $(λ k . k x)^{η} ⊛ m ≣ m ⊛ x^{η}$
    \xe

    \columnbreak

    \ex
    \textbf{Identity}\\
    $id^{η} ⊛ m ≡ m$
    \xe

    \ex~
    \textbf{Composition}\\
    $(∘)^{η} ⊛ u ⊛ v ⊛ w ≡ u ⊛ (v ⊛ w)$
    \xe

    \end{multicols}

        You can verify for yourselves that the triple $(\type{K_{t}},↑,\ml{S})$
    obeys the applicative laws -- we can call it the \textit{continuation applicative}.

    A related, more powerful abstraction from the functional programming
    literature is \textit{monads}. There is a growing body of literature in
    linguistic semantics that explicitly makes use of monads
    (see, e.g., \citealt{shan2002monads,Charlowc,grove2019}, and others). A monad, like an applicative
    functor, is defined as a triple consisting of a type constructor and two
    functions. Monads are strictly speaking more powerful than applicative
    functors; that is to say, if you have a monad you are guaranteed to have an
    applicative functor, but not vice versa.

    \pex
    \textit{Monad} (def.)\\
    A \textit{monad} consists of the following three components, subject to
    \textbf{associativity} and \textbf{identity}.\sidenote{I've defined the monad laws here in terms of
      \ml{map}, which we haven't discussed. In haskell, this corresponds to
      \texttt{fmap}. In category theory, given a functor $F:C → D$, this corresponds
      to the mapping from morphisms in $C$ to morphisms in $D$ supplied by the
      definition of $F$. Here, we'll define $\ml{map}$ as follows: a pair
      $(\type{F},\ml{map})$ is a \textit{functor}:


      P.s. don't worry if you don't know what any of this means! This is only
      here as a tidbit for interested parties.}
    \a $\type{F : \type{type} → \type{type}}$
    \a $η : \type{a → F a}$
    \a $μ: \type{F (F a)
          → F a}$
        \xe

    \begin{multicols}{2}

      \ex
      \textbf{Associativity}\\
      \(μ ∘ (\ml{map} μ) ≡ μ ∘ μ\)
      \xe

      \columnbreak

      \ex~
      \textbf{Identity}\\
      \(μ . (\ml{map} η) ≡ \ml{join} ∘ η ≡ id\)
      \xe

     \end{multicols}

     We can define join for the continuation monad as follows:

     \pex
    \textit{join} (def.)\\
    \a \(μ : \type{K_{t} (K_{t} a) → K_{t} a}\)
    \a \(m^{μ} ≔ λ k . m (λ c . c k)\)
    \xe


    In tower terms, \textit{join} takes a two-level tower and sequences effects
    from the top story down:

    \ex
    $
    \left\semtower{f []}{\semtower{g []}{x}}\right^{μ} = \semtower{f (g [])}{x}
    $
    \xe

    Interestingly, it looks like we can define join just in terms of operations
    from the applicative instance; in other words, the continuations in their
    monadic guide are no more expressive than continuations in their applicative
    guise:\sidenote{Thanks to Julian Grove for discussing this point with me.}

    \ex
    $m^{μ} = m ∘ (↑)$
    \xe

    I leave a demonstration of this fact as an exercise.

    % \section{Shift and reset}

    % \pex
    % \ml{SHIFT} (def.)
    % \a $\ml{SHIFT}: \type{((a → t) → K_{t} t) → K_{t} a}$
    % \a $\ml{SHIFT} m = λ k . (m k)^{↓}$
    % \xe

    % \pex
    % \(\ml{RESET}\) (def.)\\
    % \a \((⇵) : \type{K_{t} t → K_{t} a}\)
    % \a \(a^{⇵} = (a^{↓})^{↑}\)
    % \xe


  \section{{\LaTeX} dojo}

  Here is the macro I use to typeset towers in \LaTeX. Declare this in your
  preamble. You'll need the \texttt{booktabs} and \texttt{xparse} packages too.

  \begin{lstlisting}[language=TeX]
    \NewDocumentCommand\semtower{mm}{
      \begin{tabular}[c]{@{\,}c@{\,}}
        \(#1\)
        \\
        \midrule
        \(#2\)
        \\
      \end{tabular}
    }
  \end{lstlisting}

 A simple two-level tower can now be typeset as follows:

    \begin{lstlisting}[language=TeX]
      $$\semtower{f []}{x}$$
  \end{lstlisting}

  Resulting in:

  $$\semtower{f []}{x}$$

  \end{appendices}

\end{document}
