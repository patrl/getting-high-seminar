\documentclass[nols,twoside,nofonts,nobib,nohyper,showframe]{tufte-handout}

\usepackage{fixltx2e}
\usepackage{tikz-cd}
\usepackage{tcolorbox}
\usepackage{appendix}
\usepackage{listings}
\lstset{language=TeX,
       frame=single,
       basicstyle=\ttfamily,
       captionpos=b,
       tabsize=4,
  }

\begin{acronym}
\acro{sfa}{Scopal Function Application}
\acro{fa}{Function Application}
\acro{wco}{Weak Crossover}
\acro{ScoT}{Scope Transparency}
\acro{vfs}{Variable Free Semantics}
\acro{acd}{Antecedent Contained Deletion}
\acro{qr}{Quantifier Raising}
\acro{doc}{Double Object Construction}
\acro{ccp}{Context Change Potential}
\acro{dmg}{Dynamic Montague Grammar}
\acro{dr}{Discourse Referent}
\acro{qp}{Quantificational Phrase}
\acro{dp}{Determiner Phrase}
\acro{dpp}{Dynamic Predication Principle}
\acro{ds}{Dynamic Semantics}
\acro{gq}{Generalized Quantifier}
\acro{npi}{Negative Polarity Item}
\acro{lf}{Logical Form}
\acro{pm}{Predicate Modification}
\acro{pfa}{Pointwise Function Application}
\acro{cg}{Common Ground}
\end{acronym}

\renewcommand*{\acsfont}[1]{\textsc{#1}}

\makeatletter
% Paragraph indentation and separation for normal text
\renewcommand{\@tufte@reset@par}{%
  \setlength{\RaggedRightParindent}{0pt}%
  \setlength{\JustifyingParindent}{0pt}%
  \setlength{\parindent}{0pt}%
  \setlength{\parskip}{\baselineskip}%
}
\@tufte@reset@par

% Paragraph indentation and separation for marginal text
\renewcommand{\@tufte@margin@par}{%
  \setlength{\RaggedRightParindent}{0pt}%
  \setlength{\JustifyingParindent}{0pt}%
  \setlength{\parindent}{0pt}%
  \setlength{\parskip}{\baselineskip}%
}
\makeatother

\title{Continuation semantics i\thanks{24.979: Topics in
    semantics\\\noindent\textit{Getting high:}\\\noindent\textit{scope, projection, and evaluation order}}}

\author[Patrick D. Elliott and Martin Hackl]{Patrick~D. Elliott \& Martin Hackl}

\addbibresource[location=remote]{/home/patrl/repos/bibliography/elliott_mybib.bib}

\lingset{
  belowexskip=0pt,
  aboveglftskip=0pt,
  belowglpreambleskip=0pt,
  belowpreambleskip=0pt,
  interpartskip=0pt,
  extraglskip=0pt,
  Everyex={\parskip=0pt}
}


% \usepackage{booktabs} % book-quality tables
% \usepackage{units}    % non-stacked fractions and better unit spacing
% \usepackage{lipsum}   % filler text
% \usepackage{fancyvrb} % extended verbatim environments
%   \fvset{fontsize=\normalsize}% default font size for fancy-verbatim environments

% % Standardize command font styles and environments
% \newcommand{\doccmd}[1]{\texttt{\textbackslash#1}}% command name -- adds backslash automatically
% \newcommand{\docopt}[1]{\ensuremath{\langle}\textrm{\textit{#1}}\ensuremath{\rangle}}% optional command argument
% \newcommand{\docarg}[1]{\textrm{\textit{#1}}}% (required) command argument
% \newcommand{\docenv}[1]{\textsf{#1}}% environment name
% \newcommand{\docpkg}[1]{\texttt{#1}}% package name
% \newcommand{\doccls}[1]{\texttt{#1}}% document class name
% \newcommand{\docclsopt}[1]{\texttt{#1}}% document class option name
% \newenvironment{docspec}{\begin{quote}\noindent}{\end{quote}}% command specification environment

\begin{document}

\maketitle% this prints the handout title, author, and date

\section{Roadmap}

\section{Some notation conventions}

Generally speaking, I'll be assuming \cite{heimKratzer1998} as background, but I'll depart from their notation slightly.

Expressions in the meta-language will be typeset in \textsf{sans serif}.

    $$\eval{[\textsubscript{DP} John]} ≔ \underbrace{\ml{John}}_{\text{individual}}$$

    Seeing as its primitive, we'll treat white-space as function
    application, e.g.:

    \ex
    $(λ x . \ml{left} x) \ml{paul} = \ml{left} \ml{paul}$
    \xe

  Function application associates to the \textit{left}:

    \ex
    $(λx . λ y . y \ml{likes} x) \ml{paul} \ml{sophie} ≡ ((λx . λ y . y \ml{likes} x) \ml{paul}) \ml{sophie}$
    \xe

  We'll write \textit{types} in a \texttt{fixed width font}. We have our
    familiar primitive types...

    \ex
    $\mathtt{type} ≔ \type{e} | \type{t} | \type{s} | …$
    \xe

   ...and of course \textit{function types}. Unlike
    \citet{heimKratzer1998}, who use $⟨.⟩$ as the constructor for a function
    type, we'll be using the (more standard (outside of linguistics!)) arrow constructor $(→)$:

    \ex
    $\type{⟨e,t⟩} ≡ \type{e → t}$
    \xe

   The constructor for function types \textit{associates to the right}:

    \ex
    $\type{e → e → t} ≡ \type{e → (e → t)}$
    \xe

\end{itemize}

\section{The Partee triangle}

\ex
$\ml{LIFT} x ≔ λk  . k x$\hfill$\type{\ml{LIFT} : e → (e → t) → t}$
\xe

\ex~
$\ml{IDENT} x ≔ λ y . y = x$\hfill$\ml{IDENT} : \type{e → e → t}$
\xe

\ex~
$\ml{BE} Q ≔ λx . Q (λ y . y = x)$\hfill$\ml{BE} : \type{((e → t) → t) → e → t}$
\xe

\begin{tcolorbox}
\textbf{Commutative diagrams}
\tcblower
The \textit{Partee triangle} is a \textit{commutative diagram}. We say that a diagram \textit{commutes} if, when there are multiple paths
between two points, those paths are equivalent. The equivalence in (\ref{tri1})
is therefore expressed by the triangle.
\end{tcolorbox}


\ex The Partee triangle\sidenote{\cite{partee1986}}\\
\begin{tikzcd}
  \type{e} \arrow[rr, "\ml{LIFT}"] \arrow[dr, "\ml{IDENT}"] & & \type{(e → t) → t}
  \arrow[dl, "\ml{BE}"]\\
  & \type{e → t} &
\end{tikzcd}\label{def:ptri}
\xe

\ex
$\ml{ident} ≡ \ml{BE} ∘ (↑)$\label{tri1}
\xe

\textsc{Observation:} expressions quantificational and non-quantificational DPs
can be coordinated:

\ex
{}[Howie and a woman] entered the club\label{ex:conj1}
\xe

\ml{LIFT} allows something that, by virtue of its quantificational nature, is an
\textit{inherent} scope-taker, to combine with something that
\textit{isn't}:\sidenote{If you're familiar with \cite{parteeRooth} you'll notice
  that the \textit{and} that coordinates quantificational DPs (written here as
  \textit{and}$_{Q}$), is just the result of applying their \textit{generalized
    conjunction} rule. We'll return to generalized conjunction, and the
  connection to coninuations in \S\ref{sec:conj}.}

\ex
\begin{forest}
  [{$λ k . ∃x[\ml{woman} x ∧ k x] ∧ k \ml{Howie}$}
  [{$λk . k \ml{Howie}$} [{\ml{LIFT}} [{\ml{Howie}}]]]
  [{$λ Q' . λ k . ∃x[\ml{woman} x ∧ k x] ∧ Q' k$}
    [{$λQ . λ Q' . λ k . Q k ∧ Q' k$\\and$_{Q}$}]
    [{$λ k . ∃x[\ml{woman} x ∧ k x]$} [{a woman},roof]]
  ]
  ]
\end{forest}
\xe

\subsection{Generalizing the triangle}

Based on the way in which \ml{LIFT} and friends are defined, in theory
we could replace $\type{e}$ with \textit{any} type. Let's give a more general
statement of \ml{LIFT} and friends as polymorphic functions:

\ex
$\ml{LIFT} x ≔ λk  . k x$\hfill$\type{\ml{LIFT} : a → (a → t) → t}$
\xe

\ex~
$\ml{IDENT} x ≔ λ y . y = x$\hfill$\ml{IDENT} : \type{a → a → t}$
\xe

\ex~
$\ml{BE} Q ≔ λx . Q (λ y . y = x)$\hfill$\ml{BE} : \type{((a → t) → t) → a → t}$
\xe

The diagram, of course, still commutes:

\ex The (generalized) Partee triangle\\
\begin{tikzcd}
  \type{a} \arrow[rr, "\ml{LIFT}"] \arrow[dr, "\ml{IDENT}"] & & \type{(a → t) → t}
  \arrow[dl, "\ml{BE}"]\\
  & \type{a → t} &
\end{tikzcd}\label{def:ptri}
\xe

Why might a polymorphic \ml{LIFT} be useful? Recall that we used a typed
instantiation of \ml{LIFT} in order to allow a quantificational thing to combine
with a non-quantificational thing. Polymorphic \ml{LIFT} allows us to type-lift,
e.g., a function that takes multiple arguments:

\ex
\(\ml{LIFT} (λ xy . y \ml{hug} x) = \overbrace{λ k . k (λ
  xy . y \ml{hug} x)}^{\type{((e → e → t) → t) → t}}\)
\xe

One tantalizing possibility is that this allow us to combine a non-quantificational transitive verb with a
quantificational DP.

\ex
\begin{forest}
  [{\xmark}
    [{$\type{((e → e → t) → t) → t}$\\$λ k . k (λ xy . y \ml{hug} x)$} [{\ml{LIFT}} [{$λ xy . y \ml{hug} x$}]]]
    [{$\type{(e → t) → t}$\\$λ k . ∃x[\ml{woman} x ∧ k x]$}]
  ]
  \end{forest}
\xe

Unfortunately, assuming the usual inventory of composition rules (i.e.,
\textit{function application}, \textit{predicate modification}, and
\textit{predicate abstraction}), we're stuck.\sidenote{Other, more exotic
  composition rules such as \textit{restrict} won't help either. Take my word
  for this!} So, \textbf{let's invent a new one}.

Here's the intuition we're going to pursue. Let's look again at the types. One
way of thinking about what \ml{LIFT} as follows: it takes an \type{a}-type thing
and adds a \enquote{wrapper}. Quantificational DPs, on the other hand come
\enquote{pre-wrapped}.

\begin{itemize}

    \item $\ml{LIFT} \eval{hug}: \type{(\fbox{(e → e → t)} → t) → t}$

    \item $\eval{a woman} : \type{(\fbox{e} → t) → t}$

\end{itemize}

If we look at the wrapped-up types, we see a \textit{function} from individuals,
and an individual -- namely, two things that can combine via function application.

What we need to accomplish is the following:

\begin{itemize}

  \item Unwrap lifted \textit{hug}.

  \item Unwrap the \textit{a woman}.

  \item Use function application to combine the unwrapped values.

  \item Finally, wrap the result back up! Think of the quantificational meaning as being
    like a taco -- it isn't really a taco without the wrapper, therefore we
    don't want to throw the wrapper away.

\end{itemize}

    In order to accomplish this, we'll define a new composition rule:
    \acf{sfa}. We're going to define \ac{sfa} in terms of \acf{fa}; we haven't been
      explicit about how \ac{fa} is defined yet, so let's do that now. We'll
      write \ac{fa} as the infix operator
      \ml{A}.\sidenote{\todo[inline]{different ways of making function
          application bidirectional}}

      \pex \acf{fa} (def.)
      \a $f \ml{A} x ≔ f x$\hfill$\ml{A}:\type{(a → b) → a → b}$
      \a $x \ml{A} f ≔ f x$\hfill$\ml{A}:\type{a → (a → b) → b}$
      \xe

      Here, function application is made bidirectional by \textit{overloading}
      -- we've defined forwards and backwards application, and given them the
      same function name.

      \begin{tcolorbox}

       \textbf{\acf{sfa}}

       \tcblower

       We'll write \ac{sfa} as the infix operator \ml{S}. Note that, since
       \ml{A} is overloaded already, and \ml{S} is defined in terms of \ml{A},
       \ml{S} gets overloaded too.

       \ex
       \acf{sfa} (def.)\\
       $m \ml{S} n ≔ λ k . m (λ a . n (λ b . k (a \ml{A} b)))$\\
       \phantom{,}\hfill$\ml{S}: \type{(((a → b) → t) → t) → ((a → t) → t) → (b
         → t) → t}$\\
       \phantom{,}\hfill$\ml{S}: \type{((a → t) → t) → (((a → b) → t) → t) → (b
         → t) → t}$
       \xe

       Now, let's illustrate how \ac{sfa} plus generalized \ml{LIFT} allows to
       compose a quantificational thing with non-quantificational things, using
       a simple example sentence:

       \ex
       Howie hugged some woman.
       \xe

      \end{tcolorbox}


       \ex Step 1: compose \textit{some woman} with \ml{LIFT}-ed \textit{hug}.\\
       \begin{forest}
         [{$\type{((e → t) → t) → t}$\\\fbox{$λ k . ∃x[\ml{woman} x ∧ k (λ y . y \ml{hug} x)]$}}
         [{$\ml{S}$}
           [{$\type{((e → e → t) → t) → t}$\\$λ k . k (λ xy . y \ml{hug} x)$} [{\ml{LIFT}} [{hug}]]]
           [{$\type{(e → t) → t}$\\$λ k . ∃x[\ml{woman} x ∧ k x]$} [{some woman},roof]]
         ]]
       \end{forest}
       \xe

       \ex~ Step 2: compose the resulting VP-denotation with \ml{LIFT}-ed \textit{Howie}\\
       \begin{forest}
         [{$\type{(t → t) → t}$\\\fbox{$λ k . k (∃x[\ml{woman} x ∧ \ml{Howie} \ml{hug} x])$}}
         [{$\ml{S}$}
           [{$\type{(e → t) → t}$\\$λ k . k \ml{Howie}$} [{$\ml{LIFT}$} [{Howie}]]]
           [{$\type{((e → t) → t) → t}$\\$λ k . ∃x[\ml{woman} x ∧ k (λ y . y \ml{hug} x)]$} [{hug some woman},roof]]
         ]]
       \end{forest}\label{ex:step2}
       \xe

       We're now tantalizingly close to deriving the right kind of object for
       the sentential meaning (namely, something of type $\type{t}$). Only, what
       we have is something of type $\type{(\fbox{t} → t) → t}$, i.e., a truth
       value in a \enquote{wrapper}. How do we get back the wrapped up value? We
       saturate the $k$ argument with the identity function. We'll call this
       operation $\ml{LOWER}$.\sidenote{Here we've given
           \ml{LOWER} the maximally polymorphic type compatible with the
           function definition; in fact, all we need is
           $(\ml{LOWER} : \type{((t → t) → t) → t})$.}

       \ex
       $\ml{LOWER}$ (def.)\\
       $\ml{LOWER} m ≔ m id$\hfill$\ml{LOWER}: \type{((a → a) → a) → a}$
       \xe

       Applying \ml{LOWER} to the final value in (\ref{ex:step2}) gives us a
       type \type{t} sentential meaning.

       \ex
       $\begin{aligned}[t]
         &\ml{LOWER} (λ k . k (∃x[\ml{Howie} \ml{hug} x]))\\
         &= (λ k . k (∃x[\ml{Howie} \ml{hug} x])) id\\
         &= id (∃x[\ml{Howie} \ml{hug} x])\\
         &= ∃x[\ml{Howie} \ml{hug} x]
         \end{aligned}$
       \xe

       Let's zoom back out and see how all the pieces fit together by looking at
       the \textit{graph of the derivation}.

       \ex
       \begin{forest}
         [{$\type{t}$}
         [{$\ml{LOWER}$}
         [{$\type{(t → t) → t}$\\$\ml{S}$}
           [{$\type{(e → t) → t}$} [{$\ml{LIFT}$} [{Howie}]]]
           [{$\type{((e → t) → t) → t}$\\$\ml{S}$}
             [{$\type{((e → e → t) → t) → t}$} [{$\ml{LIFT}$} [{hug}]]]
             [{$\type{(e → t) → t}$} [{some woman},roof]]
           ]
         ]
         ]
         ]
       \end{forest}
       \xe

       So far, we've provided an account of how quantificational things compose
       with non-quantificational things, by making use of...

       \begin{itemize}

           \item ...an independently
           motivated type-shifting rule (\ml{LIFT})...

           \item ...a way to apply \ml{LIFT}-ed
           values (\ml{S})...

          \item ...and a way to get an ordinary value back from a
       \ml{LIFT}-ed value (\ml{LOWER}).

           \end{itemize}

           This seems pretty nice, but as I'm sure you've noticed, things are
           quickly going to get pretty cumbersome with more complicated
           sentences, especially with multiple quantifiers. Before we go any
           further, let's introduce some notational conveniences.

\section{Towers}

We've been using the metaphor of a \textit{wrapper} for thinking about what
$\ml{LIFT}$ does to an ordinary semantic value. Let's make this a bit more
transparent by introducing a new \textit{type constructor} for \ml{LIFT}-ed
values.\sidenote{A \textit{type constructor} is just a
   function from a type to a new type -- here, it's a rule for taking any type
   $\ml{a}$ and returning the type of the corresponding \ml{LIFT}-ed value.}

\ex
\type{K_{t} a ≔ (a → t) → t}
\xe


\begin{itemize}

\item Quantificational DPs are therefore of type $\type{{K}_{t} e}$ (inherently).

\item \ml{LIFT} takes something of type \type{a}, and lifts it into
    something of type $\type{K_{t} a}$.

\end{itemize}

Rather than dealing with \textit{flat} expressions of the simply-typed lambda
calculus, which will become increasingly difficult to reason about, we'll follow
\cite{barkerShan2015} in using \textit{tower notation}.\sidenote{To my mind, one
of \citeauthor{barkerShan2015}'s central achievements is simply the introduction
of an accessible notational convention for reasoning about the kinds of lifted
meanings we're using here.}

Let's look again at the meaning of a quantificational DP. The $k$ argument which
acts as the \textit{wrapper} is called the \textit{continuation argument}.

\ex
$\eval{some woman} ≔ λ k . ∃x[\ml{woman} x ∧ k x]$
\xe

\ex
$\eval{some woman} ≔ \semtower{∃x[\ml{woman} x ∧ []]}{x}$
\xe

\ex
$\ml{LIFT} (\eval{hug}) = \semtower{[]}{λ xy . y \ml{hug} x}$
\xe

In general:

\ex
$\semtower{f []}{x} ≔ λ k . f (k x)$
\xe

We can use tower notation for types too:

\ex
$\semtower{\type{b}}{\type{a}} ≔ \type{(a → b) → b}$
\xe

We can now redefine our type constructor $\type{K_{t}}$, and our type-shifting
operations using our new, much more concise, tower notation. These will be our
canonical definitions from now on. We'll also start abbreviating a \ml{LIFT}-ed
value $a$ as
$a^{↑}$ and a \ml{LOWER}-ed value $b$ as $b^{↓}$.

\ex The continuation type constructor \type{K_t} (def.)\\
$\type{K_{t} a} ≔ \type{\semtower{t}{a}}$
\xe

\ex~
\ml{LIFT} (def.)\sidenote{Thinking in terms of towers, \ml{LIFT} takes a value
  $a$ and returns a \enquote{trivial} tower, i.e., a tower with an empty top-story.}\\
$a^{↑} ≔ \semtower{[]}{a}$\hfill$(↑):\type{a → K_{t} a}$
\xe

\ex~
\acf{sfa} (def.)\sidenote{\ac{sfa} takes two scopal values -- one with a
  function on the bottom floor, and the other with an argument on the bottom
  floor -- and combines them by (i) doing function application on the bottom
  floor, and (ii) \textit{sequencing} the scope-takers.}\\
$\semtower{f []}{x} \ml{S} \semtower{g []}{y} ≔
\semtower{f (g [])}{x \ml{A} y}$\hfill$\ml{S}:\type{K_{t} (a → b) → K_{t} a →
  K_{t} b}$
\xe

\ex~
\ml{LOWER} (def.)\footnote{\ml{LOWER} \textit{collapses the tower}, applying
  whatever is on the top story to whatever is on the bottom story.}\\
$\left(\semtower{f []}{p}\right)^{↓}$\hfill$(↓) : \type{K_{t} t → t}$
\xe

In order to see the tower notation in action, let's go through an example
involving multiple quantifiers, and show how continuation semantics derives the
surface scope reading:

\ex
Some knight chased every thief.
\xe

\begin{fullwidth}
  \begin{multicols}{2}

First, we combine \textit{every thief} with lifted \textit{chase} via \ac{sfa}:

\ex
\begin{forest}
  [{\fbox{$\semtower{∀ x[\ml{thief} x → []]}{λ y . y \ml{chase} x$}}\\\ml{S}}
    [{$\semtower{[]}{λ xy. y \ml{chase} x}$}]
    [{$\semtower{∀x[\ml{thief} x → []]}{x}$}]
  ]
\end{forest}
\xe

\columnbreak

Next, the (boxed) VP value combines with \textit{some knight} via \ac{sfa}:

\ex
\begin{forest}
  [{\fbox{$\semtower{∃y[\ml{knight} y ∧ ∀ x[\ml{thief} x → []]]}{y \ml{chase} x}$}\\\ml{S}}
    [{$\semtower{∃y[\ml{knight} y ∧ []]}{y}$}]
    [{$\semtower{∀ x[\ml{thief} x → []]}{λ y . y \ml{chase} x$}} [{chase every thief},roof]]
  ]
\end{forest}
\xe

\end{multicols}
\end{fullwidth}

Finally, the resulting tower is collapsed via \textit{lower}:

\ex
\begin{forest}
[{\fbox{$∃y[\ml{knight} y ∧ ∀ x[\ml{thief} x → y \ml{chase} x]]$}} [{$\left(\semtower{∃y[\ml{knight} y ∧ ∀ x[\ml{thief} x → []]]}{y \ml{chase} x}\right)^{↓}$}]]
\end{forest}
\xe

Let's zoom out and look at the \textit{graph of the syntactic derivation}
alongside the \textit{graph of the semantic derivation}.

\begin{multicols}{2}

  \begin{forest}
    [{\textsc{merge}}
      [{DP} [{some knight},roof]]
      [{\textsc{merge}}
        [{V\\chase}]
        [{DP} [{every thief},roof]]
      ]
    ]
  \end{forest}

  \columnbreak

  \begin{forest}
    [{\type{t}}
    [{\ml{LOWER}}
    [{\typetower{t}{t}\\\ml{S}}
      [{\typetower{t}{e}\\some knight}]
      [{\typetower{t}{e → t}\\\ml{S}}
        [{\typetower{t}{e → e → t}} [{\ml{LIFT}} [{chase}]]]
        [{\typetower{t}{e}\\every thief}]
      ]
    ]
    ]]
  \end{forest}

  \columnbreak

\end{multicols}

\section{Continuations beyond DPs}

\subsection{Generalized (con/dis)junction}\label{sec:conj}

\todo[inline]{generalized conjunction as continuised conjunction.}


\printbibliography

\begin{appendices}

  \section{Continuations from a categorical perspective}

  The triple $(\type{K_{t}},↑,\ml{S})$ is an \textit{applicative functor}, a
  highly influential notion in the literature on functional programming
  (\citealt{mcbridePaterson2008}); for applications in linguistic semantics see
  \cite{kiselyov2017}, \cite{charlow2018}.
  \todo[inline]{Cite my applicatives paper here}

  Technically speaking, an applicative functor is a type constructor (here $\type{K_{t}}$),
  together with two functions $(↑): \type{a → K_{t} a})$ and $(\ml{S}: \type{K_{t} (a → b) →
    K_{t} a → K_{t} b})$ obeying the following laws:

  \todo[inline]{Check the applicative laws}

  \begin{multicols}{2}

    \ex
    \textbf{Homomorphism}\\
    $f^{↑} \ml{S} x^{↑} ≡ (f x)^{↑}$
    \xe

    \ex~
    \textbf{Interchange}\\
    $(λ k . k x)^{↑} \ml{S} m ≣ m \ml{S} x^{↑}$
    \xe

    \columnbreak

    \ex
    \textbf{Identity}\\
    $id^{↑} \ml{S} m ≡ m$
    \xe

    \ex~
    \textbf{Composition}\\
    $(∘)^{↑} \ml{S} u \ml{S} v \ml{S} w ≡ u \ml{S} (v \ml{S} w)$
    \xe

    \end{multicols}

        You can verify for yourselves that the triple $(\type{K_{t}},↑,\ml{S})$
    obeys the applicative laws -- we'll call it the \textit{continuation applicative}.

    A related, more powerful abstraction from the functional programming
    literature is \textit{monads}. There is a growing body of literature in
    linguistic semantics that explicitly makes use of monads
    (\citealt{Charlowc}, \todo{cite others}). A monad, like an applicative
    functor, is defined as a triple consisting of a type constructor and two
    functions. Monads are strictly speaking more powerful than applicative
    functors; that is to say, if you have a monad you are guaranteed to have an
    applicative functor, but not vice versa.

    In, e.g., \citet{barkerShan2015}, and much of the existing literature in
    linguistic semantics making use of continuations, they are presented in
    their applicative guise; in the functional programming literature however
    (and especially in \texttt{haskell}) continuations are more widely used in
    their monadic guise. A monad is a triple $(\type{K_{t}},(↑),μ)$ --
    the crucial addition here is \textit{join} ($μ$).

    \pex
    \textit{join} (def.)\\
    \a \(μ : \type{K_{t} (K_{t} a) → K_{t} a}\)
    \a \(μ m = λ k . m (λ K . K k)\)
    \xe

    \ex
    $μ (λ k . ∃x[k (λ l . ∀ y[l (x \ml{likes} y)])]) = λ j . ∃x[∀ y[j (x \ml{likes} y)])]$
    \xe


    % \section{Shift and reset}

    % \pex
    % \ml{SHIFT} (def.)
    % \a $\ml{SHIFT}: \type{((a → t) → K_{t} t) → K_{t} a}$
    % \a $\ml{SHIFT} m = λ k . (m k)^{↓}$
    % \xe

    % \pex
    % \(\ml{RESET}\) (def.)\\
    % \a \((⇵) : \type{K_{t} t → K_{t} a}\)
    % \a \(a^{⇵} = (a^{↓})^{↑}\)
    % \xe


  \section{{\LaTeX} dojo}

  Here is the macro I use to typeset towers in \LaTeX. Declare this in your
  preamble. You'll need the \texttt{booktabs} and \texttt{xparse} packages too.

  \begin{lstlisting}[language=TeX]
    \NewDocumentCommand\semtower{mm}{
      \begin{tabular}[c]{@{\,}c@{\,}}
        \(#1\)
        \\
        \midrule
        \(#2\)
        \\
      \end{tabular}
    }
  \end{lstlisting}

  A simple two-level tower can now be typeset as follows:

    \begin{lstlisting}[language=TeX]
      $$\semtower{f []}{x}$$
  \end{lstlisting}

  Resulting in:

  $$\semtower{f []}{x}$$

  \end{appendices}

\end{document}
