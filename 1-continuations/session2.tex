\documentclass[nols,twoside,nofonts,nobib,nohyper]{tufte-handout}

\usepackage{fixltx2e}
\usepackage{tikz-cd}
\usepackage{tcolorbox}
\usepackage{appendix}
\usepackage{listings}
\lstset{language=TeX,
       frame=single,
       basicstyle=\ttfamily,
       captionpos=b,
       tabsize=4,
  }

\begin{acronym}
\acro{sfa}{Scopal Function Application}
\acro{fa}{Function Application}
\acro{wco}{Weak Crossover}
\acro{ScoT}{Scope Transparency}
\acro{vfs}{Variable Free Semantics}
\acro{acd}{Antecedent Contained Deletion}
\acro{qr}{Quantifier Raising}
\acro{doc}{Double Object Construction}
\end{acronym}

\renewcommand*{\acsfont}[1]{\textsc{#1}}

\makeatletter
% Paragraph indentation and separation for normal text
\renewcommand{\@tufte@reset@par}{%
  \setlength{\RaggedRightParindent}{0pt}%
  \setlength{\JustifyingParindent}{0pt}%
  \setlength{\parindent}{0pt}%
  \setlength{\parskip}{\baselineskip}%
}
\@tufte@reset@par

% Paragraph indentation and separation for marginal text
\renewcommand{\@tufte@margin@par}{%
  \setlength{\RaggedRightParindent}{0pt}%
  \setlength{\JustifyingParindent}{0pt}%
  \setlength{\parindent}{0pt}%
  \setlength{\parskip}{\baselineskip}%
}
\makeatother

\setcounter{secnumdepth}{3}

\title{Continuation semantics i\thanks{24.979: Topics in
    semantics\\\noindent\textit{Getting high: Scope, projection, and evaluation order}}}

\author[Patrick D. Elliott and Martin Hackl]{Patrick~D. Elliott\sidenote{\texttt{pdell@mit.edu}} \& Martin Hackl\sidenote{\texttt{hackl@mit.edu}}}

\addbibresource[location=remote]{/home/patrl/repos/bibliography/elliott_mybib.bib}

\lingset{
  belowexskip=0pt,
  aboveglftskip=0pt,
  belowglpreambleskip=0pt,
  belowpreambleskip=0pt,
  interpartskip=0pt,
  extraglskip=0pt,
  Everyex={\parskip=0pt}
}


% \usepackage{booktabs} % book-quality tables
% \usepackage{units}    % non-stacked fractions and better unit spacing
% \usepackage{lipsum}   % filler text
% \usepackage{fancyvrb} % extended verbatim environments
%   \fvset{fontsize=\normalsize}% default font size for fancy-verbatim environments

% % Standardize command font styles and environments
% \newcommand{\doccmd}[1]{\texttt{\textbackslash#1}}% command name -- adds backslash automatically
% \newcommand{\docopt}[1]{\ensuremath{\langle}\textrm{\textit{#1}}\ensuremath{\rangle}}% optional command argument
% \newcommand{\docarg}[1]{\textrm{\textit{#1}}}% (required) command argument
% \newcommand{\docenv}[1]{\textsf{#1}}% environment name
% \newcommand{\docpkg}[1]{\texttt{#1}}% package name
% \newcommand{\doccls}[1]{\texttt{#1}}% document class name
% \newcommand{\docclsopt}[1]{\texttt{#1}}% document class option name
% \newenvironment{docspec}{\begin{quote}\noindent}{\end{quote}}% command specification environment

\begin{document}

\maketitle% this prints the handout title, author, and date

\section{A note on syntax}

So far, I've been a little shy about saying explicitly what we're assuming here
about syntax, and what we're assuming about the syntax-semantics mapping.

I'll assume a derivational theory, according to which structures are built-up
via successive application of \textsc{Merge}.\sidenote{I'll often use
  \enquote{syntactic structure speak} when talking about trees. This is
  harmless, since they can always be interpreted as the graph of a syntactic
  derivation, especially since trees encode both structure and order.}

\ex
\begin{forest}
  [{\textsc{Merge}}
    [{..}]
    [{\textsc{Merge}}
      [{..}]
      [{\textsc{Merge}}
        [{...}]
        [{...}]
      ]
    ]
  ]
\end{forest}
\xe

I'll furthermore adopt the hypothesis that the syntactic derivation proceeds in
lockstep with the semantic computation. This conjecture, which goes back to
MONTAGUE\todo{cite montague} is often described as \textit{direct
  compositionality}.\sidenote{Although direct compositionality is often strongly
associated with Combinatory Categorial Grammar, it's in principle independent.
See, e.g., \citet{kobele2006} for an explicit formalization of a directly compositional minimalist grammar.}

Minimally, the formatives must be \textit{tuples} consisting of phonological
features and semantic features: (\texttt{phon}, \texttt{sem}, ...). Semantic
features could be cashed out as model theoretic objects, or perhaps as
expressions of the simply typed lambda calculus.

Heretically, I'll assume that part of what \textsc{Merge} does is concatenate
phonological features. This is because \textsc{Merge} is just an instruction for
combining formatives. On the semantic side, it typically does function application.

\ex
$(𝕩,x) ∗ (𝕪,y) ≔ ([𝕩 𝕪], x \ml{A} y)$
\xe

It can also do concatenation of phonological features, plus \ac{sfa} of semantic
values (whence the left-to-right bias of \ml{S}).

\ex
$(𝕩,x) ∗ (𝕪,y) ≔ ([𝕩 𝕪], x \ml{S} y)$
\xe

I've define \ml{Lift} as a purely semantic operation -- this is to be taken as
shorthand for an operation on a formative that only effects the semantic value:

\ex
$(𝕩,x)^{↑} ≔ (𝕩,x^{↑})$
\xe

This constitutes the basics of the system laid out in \cite{elliott2019movement}.

\subsection{Deriving inverse scope}

Recall that \(\ml{LIFT}\) is a \textit{polymorphic function} -- it lifts a value
into a trivial tower:

\ex
$a^{↑} ≔ \semtower{[]}{a}$
\xe

Since \ml{LIFT} is polymorphic, in principle it can apply to any kind of value
-- even a tower! Let's flip back to lambda notation to see what happens.

\ex
$\eval{everyone} ≔ λ k . ∀x[k x]$
\xe

\ex~
$\eval{everyone}^{↑} = λ l . l (λ k . ∀x[k x])$
\xe

\begin{tcolorbox}
  Question
  \tcblower
  What is the \textit{type} of lifted \textit{everyone}?
\end{tcolorbox}

Going back to tower notation, lifting a tower adds a trivial third
story:\sidenote{In fact, via successive application of \ml{LIFT}, we can
  generate an $n-$story tower.} Following \citet{Charlowc}, when we apply
\ml{LIFT} to a tower, we'll describe the operation as \textit{external lift}
(although, it's worth bearing in mind that this is really just our original
\ml{LIFT} function).


One important thing to note is that, when we externally lift a tower, the
quantificational part of the meaning always remains on the same story relative
to the bottom floor. Intuitively, this reflects the fact that, ultimately,
\ml{LIFT} alone isn't going to be enough to derive quantifier scope ambiguities.

\ex
\(\left(\semtower{∀x[]}{x}\right)^{↑} = \semtower{[]}{\semtower{∀x[]}{x}}\)
\xe

\begin{tcolorbox}
\textbf{Question}
\tcblower
Which (if any) of the following bracketings make sense for a three-story tower:

\begin{multicols}{2}
\ex
$\semtower{\left(\semtower{f []}{g []}\right)}{x}$
\xe
\columnbreak
\ex
$\semtower{f []}{\left(\semtower{g []}{x}\right)}$
\xe
\end{multicols}
\end{tcolorbox}

The extra ingredient we're going to need, is the ability to sandwhich an empty
story into the \textit{middle} of our tower, pushing the quantificational part
of the meaning to the very top. This is \textit{internal lift} ($⇈$).\sidenote{I
can tell what you're thinking: \enquote{seriously? Another \textit{darn}
  type-shifter? How many of these are we going to need?!}.
Don't worry, I got you. Even thought we've defined internal lift here as a
primitive operation, it actually just follows from our existing machinery.
Concretely, \textit{internal lift} is really just \textit{lifted} \ml{LIFT} (so
many lifts!). Lifted \ml{LIFT} applies to its argument via \ml{S}.

\ex
$(\ml{LIFT}^{↑}) \ml{S} \semtower{f []}{x} = \semtower{f []}{\semtower{[]}{x}}$
\xe

}

\pex
\textit{Internal lift} (def.)\\
\a \((⇈) : \type{K_{t} a → K_{t} (K_{t} a)}\)
\a \(m^{⇈} ≔ λ k . m (λ x . k x^{↑})\)
\xe

It's much easier to see what internal lift is doing by using the tower notation.
We can also handily compare its effects to those of \textit{external} lift.

\begin{multicols}{2}
\ex \textit{Internal lift} (tower ver.)\\
\(\left(\semtower{f []}{x}}\right)^{⇈} ≔ \semtower{f []}{\semtower{[]}{x}}\)
\xe
\columnbreak
\ex \textit{External lift} (tower ver.)\\
\(\left({\semtower{f []}{x}}\right)^{↑} ≔ \semtower{[]}{\semtower{f []}{x}}\)
\xe

\end{multicols}

Armed with \textit{internal} and \textit{external} lifting operations, we now
have everything we need to derive inverse scope. We'll start with a simple
example (\ref{ex:classic1}).

The trick is: we \textit{internally} lift the quantifier that is destined to
take wide scope.

\ex
A boy danced with every girl.\hfill $∀ > ∃$\label{ex:classic1}
\xe

Before we proceed, we need to generalize \ml{LIFT} and \ac{sfa} to three-story
towers.\sidenote{
Before you get worries about expanding our set of primitive operations, notice
that \textit{3-story lift} is just ordinary lift applied twice. \textit{3-story}
\ac{sfa} is just \ac{sfa}, but where the bottom story combines via \ml{S} not
\ml{A}. In fact, we can generalize these operations to $n-$story towers.
}

\begin{multicols}{2}
\ex
$x^{↑_{2}} ≔ \semtower{[]}{\semtower{[]}{x}}$
\xe

\columnbreak

\ex
$\semtower{f []}{m} \ml{S}_{2} \semtower{g []}{n} ≔ \semtower{f (g [])}{m \ml{S} n}$
\xe

\end{multicols}


\begin{fullwidth}
  \begin{multicols}{2}
\ex Step 1: internally lift \textit{every girl} \\
\begin{forest}
  [{$\ml{S}_{2}$}
    [{$\semtower{[]}{\semtower{[]}{\ml{danceWith}}}$} [{dance-with$^{↑_{2}}$}]]
    [{$\semtower{∀x[]}{\semtower{[]}{x}}$} [{$⇈$} [{every girl},roof]]]
  ]
\end{forest}
\xe

\columnbreak

\ex
Step 2: \textit{ex}ternally lift \textit{a boy}\\
\begin{forest}
  [{$\ml{S}_{2}$}
    [{$\semtower{[]}{\semtower{∃y[\ml{boy} y ∧ []]}{y}}$} [{a boy$^{↑}$}]]
    [{$\semtower{∀x[\ml{girl} x → []]}{\semtower{[]}{λ y . y \ml{danceWith} x}}$} [{dance with every girl},roof]]
  ]
\end{forest}
\xe
\end{multicols}
\end{fullwidth}

What we're left with now is a 3-story tower with the universal on the top story
and the existential on the middle story. We can collapse the tower by first
collapsing the bottom two stories, and then collapsing the result. In order to
do this, we'll first define \textit{internal lower}.\sidenote{Let's again
  address the issue of expanding our set of primitive operations (in what is
  becoming something of a theme). Internal lower is just lifted lower, applying
  via \ml{S}. In other words:

  \ex
  $m^{⇊} ≡ (↓)^{↑} \ml{S} m$
  \xe

}

\ex
\textit{Internal lower} (def.)
$\left(\semtower{f []}{\semtower{g []}{p}}\right)^{⇊} ≔ \semtower{f []}{\left(\semtower{g []}{x}\right)^{↓}}$
\xe

Now we can collapse the tower by doing \textit{internal lower}, followed by
\textit{lower}:

\ex
\begin{forest}
  [{\fbox{$∀x[\ml{girl} x → (∃y[\ml{boy} y ∧ y \ml{danceWith} x])]$}}
  [{$↓$}
    [{$\semtower{∀x[\ml{girl} x → []]}{∃ x[\ml{boy} x ∧ y \ml{danceWith} x]}$}
      [{$⇊$}
        [{$\semtower{∀x[\ml{girl} x → []]}{\semtower{∃y[\ml{boy} y ∧ []]}{y \ml{danceWith} x}}$} [{a boy danced with every girl},roof]]
  ]]]]
\end{forest}
\xe

Great! We've shown how to achieve quantifier scope ambiguities using our new
framework. Let's look at the derivations again side-by-side.

\begin{fullwidth}
\begin{multicols}{2}

  \ex
  Surface scope (schematic derivation)\\
  \begin{forest}
    [{$↓$}
    [{$\ml{S}$}
      [{$Q_{1}$}]
      [{$\ml{S}$}
        [{$R^{↑}$}]
        [{$Q_{2}$}]
      ]
    ]]
  \end{forest}
  \xe

  \columnbreak

  \ex
  Inverse scope (schematic derivation)\\
  \begin{forest}
    [{$↓$}
    [{$⇊$}
  [{$\ml{S}_{2}$}
    [{$Q_{1}^{↑}$}]
    [{$\ml{S}_{2}$}
      [{$R^{↑_{2}}$}]
      [{$Q_{2}^{⇈}$}]
    ]
  ]]]
  \end{forest}
  \xe

\end{multicols}
\end{fullwidth}

There's a couple of interesting things to note here:

\begin{itemize}

    \item The inverse scope derivation involves more applications of our
    type-shifting operations -- this becomes especially clear if we decompose
    the complex operations
    $\ml{S}_{2}$, $↑_{2}$, $⇈$, and $⇊$.

    \item In order to derive an inverse scope reading, what was \textit{crucial}
    was the availability of \textit{internal lift}; the remaining operations,
    $\ml{S}_{2}, ↑_{2}, ⇊$ only functioned to massage composition for
    three-story towers.

\end{itemize}

On the latter point, it's tempting to conjecture that in, e.g., German, Japanese
and other languages which \enquote{wear their LF on their sleeve}, the semantic
correlate of \textit{scrambling} is \textit{internal lift}, whereas in
scope-flexible languages such as English, internal lift is a freely available
operation.\sidenote{To make sense of this, we would of course need to say
  something more concrete about the relationship between syntax and semantics.
  For an attempt at marrying continuations to a standard, minimalist syntactic
component, see my manuscript \textit{Movement as higher-order structure building}.}

If we adopt some version of the \textit{derivational complexity hypothesis}, we
also predict that inverse scope readings should take longer to process than
surface scope readings. This is something Martin may discuss is a couple of
weeks time.

It's worth mentioning, incidentally, that although we collapsed the resulting
three-story tower via internal lower followed by lower, we can also define an
operation that collapses a three-story tower two an ordinary tower in a
different way. Let's call it \textit{join}:\sidenote{Join for three-story towers
corresponds directly to the \textit{join} function associated with the
continuation monad. For more on continuations from a categorical perspective,
see the first appendix.}

\ex \textit{join} (def.)\\
$m^{μ} ≔ λ k . m (λ c . c k)$\hfill$μ: \type{K_{t} (K_{t} a) → K_{t} a}$
\xe

In tower terms, join takes a three-story tower and sequences quantifiers from
top to bottom:

 \ex
  $
  \left(\semtower{f []}{\semtower{g []}{x}}\right)^{μ} = \semtower{f (g [])}{x}
  $
  \xe

Doing internal lower on a three-story tower followed by lower is equivalent to doing
join on a three-story tower followed by lower (as an exercise, convince yourself
of this). However, there's may be a good empirical reason for having internal lower as a
distinct operation (and since it's just lifted lower, it \enquote{comes for
  free} in a certain sense).

\ex
Daniele wants a boy to dance with every girl.\hfill $∀ > \ml{want} > ∃$\label{ex:dani1}
\xe

Arguably, (\ref{ex:dani1}) can be true if for every girl $x$, Daniele has the following
desire: \textit{a boy dances with $y$}. This is the reading on which
\textit{every boy} scopes over the intensional verb, and \textit{a boy} scopes
below it.

If we have \textit{internal lower}, getting this is easy. We \textit{internally
  lift} \textit{every girl} and \textit{externally lift} \textit{a boy}. Before
we reach the intensional verb, we fix the scope of \textit{a boy} by doing
internal lower. Now we have an ordinary tower, and we can defer fixing the scope
of \textit{every girl} via \textit{lower} until after the intensional verb.

If we only have \textit{join} then the scope of \textit{a boy} and \textit{every
girl} may vary amongst themselves, but they should either both scope below
\textit{want} or both scope above \textit{want}.

\section{Split scope}

In the first p-set, I asked you how to think about analyzing split scope of
non-upward-monotone quantifiers:

\ex
The company need fire no employees.\hfill $¬ > □ > ∃$\\
\xe

With continuation semantics, we can understand this data as providing support
for the idea that expressions can denote three-story towers (something not
excluded by, e.g., \citealt{heimKratzer1998} in any case).

\ex
$\eval{no employees} ≔ λ k . ¬ k (λ l . ∃x[\ml{employee} x ∧ l x])$
\xe



\section{Scope islands and obligatory evaluation}

Inspired by research on \textit{delimited control} in computer
science\sidenote{See, e.g., \cite{danvyFilinski1992} and \cite{wadler1994}.},
\citet{Charlowc} develops an interesting take on scope islands couched in terms
of continuations.

He proposes the following definition:

\ex
\textit{Scope islands} (def.)\\
A \textit{scope island} is a constituent that is subject to \textit{obligatory
  evaluation}.\\
\phantom{,}\hfill\citep[p. 90]{Charlowc}
\xe

By \textit{obligatory evaluation}, we mean that every continuation argument
\textit{must} be saturated before semantic computation can proceed. In other
words, a scope island is a constituent where, if we have something of type
$\type{K_{t} a}$, we cannot proceed.

One way of thinking about this, is that the presence of an unsaturated
continuation argument means that there is some computation that is being
deferred until later. Scope islands are constituents at which evaluation is
\textit{forced}. As noted by \citeauthor{Charlowc}, this idea bears an
intriguing similarity to \citeauthor{chomskyPhase}'s notion of a
\textit{phase}.\sidenote{Exploring this parallel in greater depth could make for
an interesting term paper topic.}

How does this work in practice? A great deal of ink has been spilled arguing
that, e.g., a finite clause is a scope island.

\ex
A boy said $\overbrace{\text{\fbox{that Susan greeted every
      linguist}}}^{\text{scope island}}$.\hfill$∃>∀; \text{\xmark} ∀ > ∃$
\xe

The derivation of the embedded clause proceeds as usual via lift and \ac{sfa}.

\newpage

\begin{fullwidth}
\begin{multicols}{2}
\ex
Scope island with an unevaluated type\\
\begin{forest}
  [{...}
    [{...} [{a boy},roof]]
    [{...}
      [{said}]
      [{\xmark $\semtower{∀x[\ml{linguist} x → []]}{\ml{Susan greeted }x}$} [{Susan greeted every linguist},roof]]
    ]
  ]
\end{forest}
\xe
\columnbreak
\ex
Scope island with an evaluated type\\
\begin{forest}
  [{...}
    [{...} [{a boy},roof]]
    [{...}
      [{said}]
      [{\cmark $∀x[\ml{linguist} x→ \ml{Susan greeted }x]$}
      [{$↓$} [{$\semtower{∀x[\ml{linguist} x → []]}{\ml{Susan greeted }x}$} [{Susan greeted every linguist},roof]]]]
    ]
  ]
\end{forest}
\xe
\end{multicols}
\end{fullwidth}

This story leaves a lot of questions unanswered of course:

\begin{itemize}

    \item Is this just a recapitulation of a representational constraint on
    quantifier raising?\sidenote{The answer to this question may ultimate be
    \textit{yes}, in my view.}

    \item Can we give a principled story about islands for overt movement using
    similar mechanisms? What explains the difference between overt movement and
    scope taking with respect to locality?\footnote{If we want to give a more
    general account of phases using this mechanism, we need to give an account
    of overt movement in terms of continuations, too. See my unpublished ms.
    \textit{Movement as higher-order structure building} for progress in this direction.}

\end{itemize}

\section{Generalized coordination}

Unlike other expressions we've seen so far, we can characterize \textit{and} as
something that takes two continuized values as arguments.

\ex
\(m \ml{and} n ≔ λ k . m k ∧ n k\)
\xe


The intuition here is as follows: \textit{and} wants as its arguments things
that are \textit{guaranteed} to give back truth values at some future stage of computation.

\newpage

\begin{fullwidth}
  \begin{multicols}{2}
\ex
\begin{forest}
  [{$\ml{John} \ml{left} ∧ \ml{Mary} \ml{left}$\\$\ml{A}$}
  [{$λ k . k \ml{John}∧ k \ml{Mary}$\\\ml{A}}
    [{$λ k . k \ml{John}$\\John$^{↑}$}]
    [{$λ nk . n k ∧ k \ml{Mary}$\\$\ml{A}$}
      [{$λ mnk . n k ∧ m k$\\and}]
      [{$λ k .k \ml{Mary}$\\Mary$^{↑}$}]
    ]
  ]
    [{left}]
  ]
\end{forest}
\xe
\columnbreak
\ex
\begin{forest}
  [{$\ml{John sneezed and coughed}$\\$\ml{A}$}
    [{$λ k . k \ml{John}$\\John$^{↑}$}]
    [{$λ k . k \ml{sneezed} ∧ k \ml{coughed}$\\\ml{A}}
      [{$λ k . k (\ml{sneezed})$\\sneezed$^{↑}$}]
      [{\ml{A}}
        [{$λ mnk . n k ∧ m k$\\and}]
        [{$λ k . k (\ml{coughed})$\\coughed$^{↑}$}]
      ]
    ]
  ]
  \end{forest}
\xe
\end{multicols}
\end{fullwidth}

Conjunction exhibits \enquote{scope} ambiguities:

\ex
You're not allowed to dance and sing.\hfill $¬ > ◇ > ∧;∧ > ¬ > ◇$
\xe

We can account for the wide/narrow scope ambiguity as a matter of where we
\ml{LOWER}.

Let's first compute the semantic value of the prejacent of the modal:

\ex
\begin{forest}
  [{$λ k . k (\ml{you dance}) ∧ k (\ml{you sing})$\\$\ml{S}$}
    [{$λ k . k \ml{you}$\\you$^{↑}$}]
    [{$λ k . k \ml{dance} ∧ k \ml{sing}$\\$\ml{A}$}
       [{$λ k . k \ml{dance}$\\dance$^↑$}]
       [{$\ml{A}$}
         [{$λ mnk . n k ∧ m k$\\and}]
         [{$λ k . k \ml{sing}$} [{sing$^{↑}$},roof]]
       ]
    ]
  ]
\end{forest}
\xe

If we \ml{LOWER} immediately we're just going to get a proposition, which
\ml{allowed} will take as its argument, deriving the narrow scope
reading.\sidenote{Instead of composing via \ml{S} and lowering, we could equivalently compose lifted
  \textit{you} with its complement via \ml{A}, since it's a subject.}

The \enquote{wide scope} reading is more interesting. We can simply defer
lowering, and compose the prejacent with lifted \textit{allowed} via \ml{S}.

\ex
\begin{forest}
  [{$(¬ (◇ (\ml{you sing}))) ∧ (¬ (◇ (\ml{you dance})))$}
  [{\ml{LOWER}}
  [{$λ k . k (¬ (◇ (\ml{you sing}))) ∧ k (¬ (◇ (\ml{you dance})))$\\$\ml{S}$}
    [{$λ k . k (λ p . ¬ p)$\\not$^{↑}$}]
    [{$\ml{S}$}
      [{$λ k . k (λ p . ◇ p)$\\allowed$^{↑}$}]
      [{$λ k . k (\ml{you dance}) ∧ k (\ml{you sing})$} [{you dance and sing},roof]]
    ]
  ]]]
\end{forest}
\xe

We make the nice prediction that \enquote{wide scope} readings of conjunction
should be subject to scope islands:

\ex
John isn't allowed to claim [that you sing and dance].\hfill \xmark $∧ > ¬ > ◇$
\xe

\printbibliography


\end{document}
