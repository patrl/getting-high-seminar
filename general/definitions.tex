\documentclass[nols,twoside,nofonts,nobib,nohyper]{tufte-handout}

\usepackage{fixltx2e}
\usepackage{tikz-cd}
\usepackage{tcolorbox}
\usepackage{appendix}
\usepackage{listings}
\lstset{language=TeX,
       frame=single,
       basicstyle=\ttfamily,
       captionpos=b,
       tabsize=4,
  }

\begin{acronym}
\acro{sfa}{Scopal Function Application}
\acro{fa}{Function Application}
\acro{wco}{Weak Crossover}
\acro{ScoT}{Scope Transparency}
\acro{vfs}{Variable Free Semantics}
\acro{acd}{Antecedent Contained Deletion}
\acro{qr}{Quantifier Raising}
\acro{doc}{Double Object Construction}
\acro{ccp}{Context Change Potential}
\acro{dmg}{Dynamic Montague Grammar}
\acro{dr}{Discourse Referent}
\acro{qp}{Quantificational Phrase}
\acro{dp}{Determiner Phrase}
\acro{dpp}{Dynamic Predication Principle}
\acro{ds}{Dynamic Semantics}
\acro{gq}{Generalized Quantifier}
\acro{npi}{Negative Polarity Item}
\acro{lf}{Logical Form}
\acro{pm}{Predicate Modification}
\acro{pfa}{Pointwise Function Application}
\acro{cg}{Common Ground}
\end{acronym}

\renewcommand*{\acsfont}[1]{\textsc{#1}}

\makeatletter
% Paragraph indentation and separation for normal text
\renewcommand{\@tufte@reset@par}{%
  \setlength{\RaggedRightParindent}{0pt}%
  \setlength{\JustifyingParindent}{0pt}%
  \setlength{\parindent}{0pt}%
  \setlength{\parskip}{\baselineskip}%
}
\@tufte@reset@par

% Paragraph indentation and separation for marginal text
\renewcommand{\@tufte@margin@par}{%
  \setlength{\RaggedRightParindent}{0pt}%
  \setlength{\JustifyingParindent}{0pt}%
  \setlength{\parindent}{0pt}%
  \setlength{\parskip}{\baselineskip}%
}
\makeatother

\setcounter{secnumdepth}{3}

\title{Definitions\thanks{24.979: Topics in
    semantics\\\noindent\textit{Getting high: Scope, projection, and evaluation order}}}

\author[Patrick D. Elliott and Martin Hackl]{Patrick~D. Elliott\sidenote{\texttt{pdell@mit.edu}} \& Martin Hackl\sidenote{\texttt{hackl@mit.edu}}}

\addbibresource[location=remote]{/home/patrl/repos/bibliography/elliott_mybib.bib}

\lingset{
  belowexskip=0pt,
  aboveglftskip=0pt,
  belowglpreambleskip=0pt,
  belowpreambleskip=0pt,
  interpartskip=0pt,
  extraglskip=0pt,
  Everyex={\parskip=0pt}
}


% \usepackage{booktabs} % book-quality tables
% \usepackage{units}    % non-stacked fractions and better unit spacing
% \usepackage{lipsum}   % filler text
% \usepackage{fancyvrb} % extended verbatim environments
%   \fvset{fontsize=\normalsize}% default font size for fancy-verbatim environments

% % Standardize command font styles and environments
% \newcommand{\doccmd}[1]{\texttt{\textbackslash#1}}% command name -- adds backslash automatically
% \newcommand{\docopt}[1]{\ensuremath{\langle}\textrm{\textit{#1}}\ensuremath{\rangle}}% optional command argument
% \newcommand{\docarg}[1]{\textrm{\textit{#1}}}% (required) command argument
% \newcommand{\docenv}[1]{\textsf{#1}}% environment name
% \newcommand{\docpkg}[1]{\texttt{#1}}% package name
% \newcommand{\doccls}[1]{\texttt{#1}}% document class name
% \newcommand{\docclsopt}[1]{\texttt{#1}}% document class option name
% \newenvironment{docspec}{\begin{quote}\noindent}{\end{quote}}% command specification environment

\begin{document}

\maketitle% this prints the handout title, author, and date

\section{Tower notation}

\begin{multicols}{2}
\ex Tower values (def.)\\
$\semtower{f []}{x} ≔ λ k . f (k x)$
\xe

\columnbreak

\ex Tower types (def.)\\
$\semtower{\type{b}}{\type{a}} ≔ \type{(a → b) → b}$
\xe

\end{multicols}

N.b. we also use $\type{K_{b} a}$ as an abbreviation for types of the form $\type{(a → b) → b}$.

\section{Composition rules}

\subsection{Bidirectional \acf{fa}}

\pex \acf{fa} (def.)
      \a $f \ml{A} x ≔ f x$\hfill$\ml{A}:\type{(a → b) → a → b}$
      \a $x \ml{A} f ≔ f x$\hfill$\ml{A}:\type{a → (a → b) → b}$
      \xe

\subsection{\ml{LIFT}}

\ml{LIFT} (a generalization of \textit{Montague lift}) lifts a value into a
trivially continuized value.

\begin{multicols}{2}

\ex
\ml{LIFT} (def.)\\
$a^{↑} ≔ λ k . k a$\hfill$(↑):\type{a → K_{t} a}$
\xe

\columnbreak

\ex
\ml{LIFT} (tower ver.)\\
$a^{↑} ≔ \semtower{[]}{a}$
\xe

\end{multicols}

Since \ml{LIFT} is polymorphic, we can use it to lift continuized values -- we
call this \textit{external lift} (although it's really just \ml{LIFT}).

\subsection{\acf{sfa}}

\begin{fullwidth}
\begin{multicols}{2}
\ex
\acf{sfa} (def.)\\
$m \ml{S} n ≔ λ k . m (λ a . n (λ b . k (a \ml{A} b)))$\\
\xe

\columnbreak

\ex~
\acf{sfa} (tower ver.)\\
$\semtower{f []}{x} \ml{S} \semtower{g []}{y} ≔
\semtower{f (g [])}{x \ml{A} y}$
\xe

\end{multicols}
\end{fullwidth}

\subsection{\ml{LOWER}}

\begin{multicols}{2}

   \ex
       $\ml{LOWER}$ (def.)\\
       $m^{↓} ≔ m id$\hfill$\ml{LOWER}: \type{((a → a) → a) → a}$
       \xe


  \columnbreak

\ex~
\ml{LOWER} (tower ver.)\\
$\left(\semtower{f []}{p}\right)^{↓} = f p$\hfill$(↓) : \type{K_{t} t → t}$
\xe

\end{multicols}



\printbibliography


\end{document}
