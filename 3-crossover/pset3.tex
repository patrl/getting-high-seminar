\documentclass[nols,twoside,nofonts,nobib,nohyper]{tufte-handout}

\usepackage{fixltx2e}
\usepackage{tikz-cd}
\usepackage{tcolorbox}
\usepackage{appendix}
\usepackage{listings}
\lstset{language=TeX,
       frame=single,
       basicstyle=\ttfamily,
       captionpos=b,
       tabsize=4,
  }

\begin{acronym}
\acro{sfa}{Scopal Function Application}
\acro{fa}{Function Application}
\acro{wco}{Weak Crossover}
\acro{ScoT}{Scope Transparency}
\acro{vfs}{Variable Free Semantics}
\acro{acd}{Antecedent Contained Deletion}
\acro{qr}{Quantifier Raising}
\acro{doc}{Double Object Construction}
\end{acronym}

\renewcommand*{\acsfont}[1]{\textsc{#1}}

\makeatletter
% Paragraph indentation and separation for normal text
\renewcommand{\@tufte@reset@par}{%
  \setlength{\RaggedRightParindent}{0pt}%
  \setlength{\JustifyingParindent}{0pt}%
  \setlength{\parindent}{0pt}%
  \setlength{\parskip}{\baselineskip}%
}
\@tufte@reset@par

% Paragraph indentation and separation for marginal text
\renewcommand{\@tufte@margin@par}{%
  \setlength{\RaggedRightParindent}{0pt}%
  \setlength{\JustifyingParindent}{0pt}%
  \setlength{\parindent}{0pt}%
  \setlength{\parskip}{\baselineskip}%
}
\makeatother

\setcounter{secnumdepth}{3}

\title{P-set 3\thanks{24.979: Topics in
    semantics\\\noindent\textit{Getting high: Scope, projection, and evaluation order}}}

\author[Patrick D. Elliott and Martin Hackl]{Patrick~D. Elliott\sidenote{\texttt{pdell@mit.edu}} \& Martin Hackl\sidenote{\texttt{hackl@mit.edu}}}

\addbibresource[location=remote]{/home/patrl/repos/bibliography/elliott_mybib.bib}

\lingset{
  belowexskip=0pt,
  aboveglftskip=0pt,
  belowglpreambleskip=0pt,
  belowpreambleskip=0pt,
  interpartskip=0pt,
  extraglskip=0pt,
  Everyex={\parskip=0pt}
}


% \usepackage{booktabs} % book-quality tables
% \usepackage{units}    % non-stacked fractions and better unit spacing
% \usepackage{lipsum}   % filler text
% \usepackage{fancyvrb} % extended verbatim environments
%   \fvset{fontsize=\normalsize}% default font size for fancy-verbatim environments

% % Standardize command font styles and environments
% \newcommand{\doccmd}[1]{\texttt{\textbackslash#1}}% command name -- adds backslash automatically
% \newcommand{\docopt}[1]{\ensuremath{\langle}\textrm{\textit{#1}}\ensuremath{\rangle}}% optional command argument
% \newcommand{\docarg}[1]{\textrm{\textit{#1}}}% (required) command argument
% \newcommand{\docenv}[1]{\textsf{#1}}% environment name
% \newcommand{\docpkg}[1]{\texttt{#1}}% package name
% \newcommand{\doccls}[1]{\texttt{#1}}% document class name
% \newcommand{\docclsopt}[1]{\texttt{#1}}% document class option name
% \newenvironment{docspec}{\begin{quote}\noindent}{\end{quote}}% command specification environment

\begin{document}

\maketitle% this prints the handout title, author, and date

\section{Continuation semantics with assignments}

In class I laid out one way of incorporating the \enquote{standard} theory of
pronouns into continuation semantics, in such a way that
\citeauthor{barkerShan2015}'s theory of \ac{wco} is preserved.

We assumed that pronouns (i) contribute an individual locally, (ii) \textit{expect} a
    proposition, (iii) and \textit{return} an assignment sensitive proposition.

\ex
Pronoun (def.)\\
$\ml{pro}ₙ ≔ \semtower{λ g . []}{gₙ}$\hfill$\ml{pro}_{n}:\type{\tower{g → t}{t}{e}}$
\xe

\begin{tcolorbox}
\textbf{Exercise 1: warming up}
\tcblower
Compute the meaning of the following sentence, assuming the meaning for pronouns
outlined above. Lower the result. What do you get? N.b. assume that the pronoun
is \textit{free}.

\ex
Jo likes him$_{1}$.
\xe

\vspace{1\baselineskip}

Remember that, in order to accommodate tower types where the expected and return
types differ, \ac{sfa} has a more general typing (adjacent types must match, and
cancel out):

\ex
$\ml{S} : \type{\tower{r}{i}{a \rightarrow b} \rightarrow \tower{i}{e}{a} \rightarrow \tower{r}{e}{b}\righta}$
\xe
\vspace{1\baselineskip}
Lower also has a more general typing:

\ex
$\downarrow : \type{\tower{a}{t}{t} \rightarrow a}$
\xe
\vspace{1\baselineskip}
The definitions of these operations don't change.

\end{tcolorbox}

On this account, quantifiers must be type-shifted into binders via an operation
we called \textit{abstract}:\sidenote{This is the categorematic counterpart of
  \citeauthor{heimKratzer1998}'s \textit{Predicate Abstraction} rule.}

\ex
Abstract (def.)\\
$Λ_{n} m ≔ λ k . λ g . m (λ x . k x g^{[n → x]})$\hfill$Λₙ : \type{\semtower{t}{e} → \semtower{g → t}{e}}$
\xe

\begin{tcolorbox}
\textbf{Exercise 2: getting hotter}
\tcblower
Compute the meaning of the following sentence by abstract-shifting the binder:

\ex
The teacher gave every student$^{1}$'s mother her$_{1}$ report.
\xe

\vspace{\baselineskip}

Assume that the \ac{doc} has an \textit{ascending} structure, i.e.:

\ex
\begin{forest}
  [{VP}
  [{V'}
    [{give}]
    [{DP}
       [{DP} [{every student},roof]]
       [{D'}
         [{'s}]
         [{mother}]
       ]
    ]
  ]
    [{DP} [{her$_{1}$ report},roof]]
  ]
\end{forest}
\xe

See \cite{jankeNeeleman2012} for recent arguments that this structure
\textit{must} be available, at least some of the time, for the English verb phrase.

\end{tcolorbox}

\begin{tcolorbox}
\textbf{Exercise 3: Handling multiple pronouns}\\
\tcblower
Try to compute and lower the meaning of the following sentence; explain what
goes wrong (both pronouns are free).

\ex
He$_{1}$ likes her$_{2}$.
\xe

\end{tcolorbox}

\section{Going monadic}

In the answer to the previous question, you'll observe that there's a problem
with the way in which we turn pronouns into scope-takers.\sidenote{Keny
  Chatain inspired this set of exercises by pointing out this deficiency.} Recall pronouns in the
standard theory has the following meaning:

\ex Pronouns (standard def.)\\
$\ml{pro}ₙ ≔ λ g . gₙ$\hfill$\ml{pro}ₙ : \type{g → e}$
\xe

There's a different way to shift standard pronouns into scope-takers, via a
function we'll call $⋆$ (star).\sidenote{Star is the \textit{bind} of a
  \texttt{Reader} monad. In his dissertation, \citet{Charlowc} shows that for a
  given monad $m$ the bind of that monad can be used to shift an inhabitant of
  $m$ into a scope-taker. Here, we implicitly make use of the same idea.}

\ex Star (def.)\\
$\star m  ≔ λ k . λ g . k (m g) g$\hfill$⋆ : \type{(g → a) → \semtower{g \rightarrow b}{a}}$
\xe

Star-shifting a pronoun gives us the following entry:

\ex
Star-shifted pronoun\\
$⋆ \ml{pro}_{n} = λ k . λ g . k gₙ g$\hfill$(⋆ \ml{pro}ₙ):\type{\semtower{g \rightarrow t}{e}}$
\xe

We can represent this meaning as a tower in
the following way:

\ex
Star-shifted pronoun (tower version)\\
$\ml{pro}_{n} ≔ \semtower{λ g . ([] g)}{g_{n}}$
\xe

In order to accommodate the result of adopting this entry for the pronoun, we'll
need a slightly different entry for lower:

\ex
Lower (revised version)\\
$m^{̣↓} ≔ m (λ pg . p)$\hfill$\downarrow : \type{\semtower{g \rightarrow t}{t}} \rightarrow g \rightarrow t$
\xe

\begin{tcolorbox}
\textbf{Exercise 4}
\tcblower
Convince yourself that our basic theory of variable binding remains intact.
Compute the meaning of the following sentence, assuming the star-shifted def for
the pronoun and the revised version of lower. We can keep our old version of
abstract.

\ex
Every boy$^{2}$ loves his$_{2}$ mother.
\xe

\vspace{\baselineskip}

Now demonstrate how the star-shifted entry for the pronoun handles multiple
pronouns in the following example:

\ex
He$₁$ likes her$₂$.
\xe
\end{tcolorbox}

\printbibliography

\end{document}
