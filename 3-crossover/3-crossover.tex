\documentclass[nols,twoside,nofonts,nobib,nohyper]{tufte-handout}

\usepackage{fixltx2e}
\usepackage{tikz-cd}
\usepackage{tcolorbox}
\usepackage{appendix}
\usepackage{listings}
\lstset{language=TeX,
       frame=single,
       basicstyle=\ttfamily,
       captionpos=b,
       tabsize=4,
  }

\begin{acronym}
\acro{sfa}{Scopal Function Application}
\acro{fa}{Function Application}
\acro{wco}{Weak Crossover}
\acro{ScoT}{Scope Transparency}
\acro{vfs}{Variable Free Semantics}
\acro{acd}{Antecedent Contained Deletion}
\acro{qr}{Quantifier Raising}
\acro{doc}{Double Object Construction}
\acro{ccp}{Context Change Potential}
\acro{dmg}{Dynamic Montague Grammar}
\acro{dr}{Discourse Referent}
\acro{qp}{Quantificational Phrase}
\acro{dp}{Determiner Phrase}
\acro{dpp}{Dynamic Predication Principle}
\acro{ds}{Dynamic Semantics}
\acro{gq}{Generalized Quantifier}
\acro{npi}{Negative Polarity Item}
\acro{lf}{Logical Form}
\acro{pm}{Predicate Modification}
\acro{pfa}{Pointwise Function Application}
\acro{cg}{Common Ground}
\end{acronym}

\renewcommand*{\acsfont}[1]{\textsc{#1}}

\makeatletter
% Paragraph indentation and separation for normal text
\renewcommand{\@tufte@reset@par}{%
  \setlength{\RaggedRightParindent}{0pt}%
  \setlength{\JustifyingParindent}{0pt}%
  \setlength{\parindent}{0pt}%
  \setlength{\parskip}{\baselineskip}%
}
\@tufte@reset@par

% Paragraph indentation and separation for marginal text
\renewcommand{\@tufte@margin@par}{%
  \setlength{\RaggedRightParindent}{0pt}%
  \setlength{\JustifyingParindent}{0pt}%
  \setlength{\parindent}{0pt}%
  \setlength{\parskip}{\baselineskip}%
}
\makeatother

\setcounter{secnumdepth}{3}

\title{Crossover i\thanks{24.979: Topics in
    semantics\\\noindent\textit{Getting high: Scope, projection, and evaluation order}}}

\author[Patrick D. Elliott and Martin Hackl]{Patrick~D. Elliott\sidenote{\texttt{pdell@mit.edu}} \& Martin Hackl\sidenote{\texttt{hackl@mit.edu}}}

\addbibresource[location=remote]{/home/patrl/repos/bibliography/elliott_mybib.bib}

\lingset{
  belowexskip=0pt,
  aboveglftskip=0pt,
  belowglpreambleskip=0pt,
  belowpreambleskip=0pt,
  interpartskip=0pt,
  extraglskip=0pt,
  Everyex={\parskip=0pt}
}


% \usepackage{booktabs} % book-quality tables
% \usepackage{units}    % non-stacked fractions and better unit spacing
% \usepackage{lipsum}   % filler text
% \usepackage{fancyvrb} % extended verbatim environments
%   \fvset{fontsize=\normalsize}% default font size for fancy-verbatim environments

% % Standardize command font styles and environments
% \newcommand{\doccmd}[1]{\texttt{\textbackslash#1}}% command name -- adds backslash automatically
% \newcommand{\docopt}[1]{\ensuremath{\langle}\textrm{\textit{#1}}\ensuremath{\rangle}}% optional command argument
% \newcommand{\docarg}[1]{\textrm{\textit{#1}}}% (required) command argument
% \newcommand{\docenv}[1]{\textsf{#1}}% environment name
% \newcommand{\docpkg}[1]{\texttt{#1}}% package name
% \newcommand{\doccls}[1]{\texttt{#1}}% document class name
% \newcommand{\docclsopt}[1]{\texttt{#1}}% document class option name
% \newenvironment{docspec}{\begin{quote}\noindent}{\end{quote}}% command specification environment

\begin{document}

\maketitle% this prints the handout title, author, and date

\begin{tcolorbox}
  \textbf{General notice}
  \tcblower
  \begin{itemize}
      \item This will, as far as we're aware, be our final meeting in-person for
      the remainder of the semester. We plan to hold subsequent meetings via
      \texttt{zoom}. Details will follow.

      \item Next week's class, is \textit{canceled}, and the following week is
      spring break. The next class will therefore be on \textbf{Thursday April 2}.

      \item We're going to do everything we can to make sure that this class
      continues to run as smoothly as possible.

      \item We're
      available for remote meetings during normal working hours. Take advantage of this!
  \end{itemize}
\end{tcolorbox}

\begin{tcolorbox}
  \textbf{Homework}
  \tcblower
  \begin{itemize}
      \item \textsc{Registered students}: please send us a \textit{project proposal}
      (less than two pages long) by Thursday March 26. This should ideally
      include a brief summary of what you plan to present in class (either a
      paper, or your own work).
      \item \textsc{Everyone}:

      \begin{itemize}

          \item Read Gennaro Chierchia's 2019 paper \enquote{Origins of weak
          crossover: when dynamic semantics meets event semantics}
          (\textit{Natural Language Semantics}). Send me at least one question
          by the end of Spring break.

          \item There will be a third problem set, posted on Friday.

      \end{itemize}
  \end{itemize}
\end{tcolorbox}

\section{Setting the stage}

\begin{itemize}

    \item Last time Martin gave an overview of the movement
    approach to quantifier scope, and some of the other analytical approaches
    available to us (e.g., the $ϵ$-calculus).

    \item Given the extensive and rich literature, quantifier raising is
    a powerful tool for analyzing phenomena such as \ac{acd}.

    \item \textit{Continuation semantics} is a much less mature framework. There
    haven't yet been serious attempts to model, e.g., \ac{acd}, but this should
    at least be attempted.\sidenote{There are also over-generation issues. Much like
    \ac{qr}, \textit{scope islands} seriously constrain scope-taking
    possiblities (and unlike \ac{qr}, continuations allow for a purely
    \textit{denotational} theory of scope islands). On other hand, our toy account
    of split scope demonstrated that continuations are so powerful, that
    unattested readings can be difficult to block.}.

    \item There are some conceptual advantages to continuation semantics,
    however -- it sidesteps non-compositional complications involving, e.g., \textit{trace
    conversion}\sidenote{\cite{sauerland2004trace,FoxJohnson16}, etc.}, allowing
    for expressions to take scope via a generalization of
    \citeauthor{partee1986}'s \textsc{lift}.

    \item Furthermore \ac{sfa} -- the composition rule essential for composing
    continuized values -- has a \textit{built in left-to-right bias}.

    \item So far we've only applied this to the (poorly understood) surface
    scope bias for sentences with multiple QPs.

    \item This week we'll be getting round to (arguably) the jewel in the crown
    of the continuations literature -- a semantic account of \textit{crossover phenomena}.

    \item I'll begin by giving a brief overview of the phenomenon, before
    discussing pronominal binding in \acf{vfs}.

    \item I'll show how the \citeauthor{barkerShan2015} account of crossover
    leverages distinctive properties of continuation semantics, in a
    \textit{variable free} setting.

    \item In the next section, I'll show how we can translate
    \citeauthor{barkerShan2015}'s account into the \enquote{standard} setting,
    where pronouns denote variables.

\end{itemize}

\section{The phenomenon}

\subsection{Weak crossover and overt movement}

% The strong crosso

% \ex
% Who$^{x}$ $t_{x}$ likes himself$_{x}$
% \xe

% \ex~
% Who$^{x}$ does he$_{x}$ like $t_{x}$?
% \xe

The simplest form of the \ac{wco} paradigm\sidenote{The term \enquote{crossover}
  was originally coined by Paul Postal.
} is illustrated below:

\pex
\a Who$^{x}$ $t_{x}$ likes his$_{x}$ mother?
\a\ljudge{*}Who$^{x}$ does his mother like $t_{x}$?
\xe

At first blush, it looks like the \textit{wh-}quantifier can only bind a
pronominal if the base-position of the \textit{wh} c-commands the pronoun.

Why is this a problem? It is fairly standard to assume that scope feeds binding;
in fact, according in semantics 101, it's often assumed that scope is
\textit{necessary} for binding -- moving the \textit{wh} creates an abstraction
index.

The following LF should be perfectly legitimate from the perspective of the
semantics:

\ex
who $1$ [his$_{1}$ mother likes $t_{1}$]?
\xe

Since both traces and pronouns are interpreted as variables, there is no reason
why the representation above shouldn't result in a sensible reading.

This constraint on variable binding
extends beyond configurations involving overt \textit{wh}-movement to those
involving quantificational scope.\sidenote{This was first observed by Chomsky
  (1976)
}

\pex
\a Everyone$^{x}$ loves his$_{x}$ mother.
\a\ljudge{*}His$_{x}$ mother loves everyone$^{x}$.
\xe


A special case of \ac{wco} is \textit{strong crossover} -- in a strong crossover
configuration, the bound pronoun c-commands the base position of the binder.

\pex
\a\ljudge{*}Who$^{x}$ did he$_{x}$ say Mary saw $t_{x}$?
\a Who$^{x}$ said Mary saw him$_{x}$.
\xe

\pex~
\a\ljudge{*}He$_{x}$ wants to see everyone$^{x}$?
\a Everyone$^{x}$ wants to see him$_{x}$.
\xe

\subsection{A- vs. A'-dependencies}

Unlike A'-movement, A-movement \textit{bleeds} \ac{wco}.

This is illustrated for A-movement (raising) of a QP...

\ex
Everyone$^{x}$ seems to his$_{x}$ mother to be a genius.
\xe

...and for A-movement, followed by A'-movement, of a \textit{wh-}expression. Crucially,
the dependency spanning the bound variable is an A-dependency:

\ex
Who$^{x}$ seems to his$_{x}$ mother to be a genius.
\xe

We'll have something to say about this later on.

\subsection{\ac{wco} is about scope, not c-command}

It has been known for some time that variable binding \textit{doesn't} require
c-command (contra received wisdom):

In the following examples, the base-position of the binder doesn't c-command the
bound variable, but binding is still possible (Ruys 1992 calls this the
\textit{transitivity property} of variable binding).

\ex
{}[Every boy$^{x}$'s mother] loves him$_{x}$.
\xe

\ex~
{}[[Every boy$^{x}$'s mother]'s husband] loves him$_{x}$.
\xe

\ex~
{}[Which boy$^{x}$'s mother] loves him$_{x}$.
\xe

\ex~
{}[[Which boy$^{x}$'s mother]'s husband] loves him$_{x}$.
\xe

Note that this paradigm could in itself be tricky for quantifier raising theories of
scope (the \textit{wh} pied-piping cases fare even worse), especially if DP is
a phase.

Continuation semantics, on the other hand, straightforwardly predicts scope and
hence binding out of DP, as we'll see later.

Binding out of DP correlates with inverse linking readings:

\ex
{}[Someone in [every city]$^{x}$] hates it$_{x}$.\hfill
\cmark $\forall > \exists$; \xmark $\exists > \forall$
\xe

Scope is harder to distinguish between two \textit{wh-}expressions:\sidenote{In
  fact, if the \textit{wh-}expressions are just existential quantifiers, they
  should be scopally commutative.

  One might imagine that the binder must be the \textit{sort key} (in
  \citeauthor{Kuno}'s \citeyear{Kuno} sense) under a pair-list reading of the
  question. I've argued however in other work (see \citealt{elliott-nesting})
  that what I call \textit{nested wh-questions} following
  \cite{heim_questions_seminar}, lack a pair-list reading.

  These kinds of examples deserve more careful consideration.
}

\ex
{}[Which picture of [which boy]$^{x}$] pleased him$_{x}$.
\xe

Note that \ac{wco} effects obtain if the the pronoun precedes the base-position
of the DP containing
the binder:

\ex
\ljudge{*}His$_{x}$ father loves [every boy$^{x}$'s mother].
\xe

\ex~
\ljudge{*}{}[Whose$^{x}$ father] does his$_{x}$ mother hate?
\xe

It seems that crossover obtains if a pronoun occurs to the \textit{left} of the
base-position of its binder (modulo A-movement).

\section{Weakest crossover}

Lasnik \& Stowell observe that crossover is obviated in configurations such as
the following:

\ex
Who did you stay with [$Op_{PG}$ before his wife had spoken to \_\_]?
\xe

\section{Crossover phenomena in continuation semantics}

\begin{fullwidth}
\begin{tcolorbox}
  \textbf{A refresher}
  \tcblower
  \begin{multicols}{2}

\ex Tower notation (def.)\\
$\semtower{f []}{x} ≔ λ k . f (k x)$
\xe

\ex Tower types (def.)\\
$\semtower{\type{b}}{\type{a}} ≔ \type{(a → b) → b}$
\xe

\ex Type constructor \type{K_t} (def.)\\
$\type{K_{t} a} ≔ \type{\semtower{t}{a}}$
\xe


\columnbreak

\ex~
\textit{lift} (def.)\\
$a^{↑} ≔ \semtower{[]}{a}$\hfill$(↑):\type{a → K_{t} a}$
\xe

\ex \textit{Internal lift} (tower ver.)\\
\(\left(\semtower{f []}{x}}\right)^{⇈} ≔ \semtower{f []}{\semtower{[]}{x}}\)
\xe

\ex~
\textit{lower} (def.)\\
$\left(\semtower{f []}{p}\right)^{↓} = f p$\hfill$(↓) : \type{K_{t} t → t}$
\xe

\ex
\textit{Internal lower} (def.)\\
$\left(\semtower{f []}{\semtower{g []}{p}}\right)^{⇊} ≔ \semtower{f []}{\left(\semtower{g []}{p}\right)^{↓}}$
\xe


\end{multicols}

\ex~
\acf{sfa} (def.)\\
$\semtower{f []}{x} \ml{S} \semtower{g []}{y} ≔
\semtower{f (g [])}{x \ml{A} y}$\hfill$\ml{S}:\type{K_{t} (a → b) → K_{t} a →
  K_{t} b}$
\xe


\end{tcolorbox}
\end{fullwidth}


Remember: the default in continuation semantics is \textit{left-to-right
  sequencing of scopal effects}.

\textit{Lower} accounts for scopal ambiguities with scopally immobile
expressions, such as intensional verbs etc.

We need \textit{internal lift} and \textit{$n$-story towers} in order to obviate
the \textit{left-to-right bias}, and account for inverse scope (in non
scopally-rigid languages).\sidenote{See \cite{barker2002} for a different
  approach couched in continuation semantics, which posits both a rightwards
  version of \ac{sfa} and a leftwards version.}


For a reminder of how this works, let's derive an inverse scope reading:

\ex
A boy danced with every girl.\hfill$\forall > \exists$
\xe

\begin{fullwidth}
  \begin{multicols}{2}
\ex Step 1: internally lift \textit{every girl} \\
\begin{forest}
  [{\fbox{$\semtower{∀x[\ml{girl} x → []]}{\semtower{[]}{λ y . y \ml{danceWith} x}}$}\\$\ml{S}_{2}$}
    [{$\semtower{[]}{\semtower{[]}{\ml{danceWith}}}$} [{dance-with$^{↑_{2}}$}]]
    [{$\semtower{∀x[]}{\semtower{[]}{x}}$} [{$⇈$} [{every girl},roof]]]
  ]
\end{forest}
\xe

\columnbreak

\ex
Step 2: \textit{ex}ternally lift \textit{a boy}\\
\begin{forest}
  [{\fbox{$\semtower{∀x[\ml{girl} x → []]}{\semtower{∃y[\ml{boy} y ∧ []]}{y \ml{danceWith} x}}$}\\$\ml{S}_{2}$}
    [{$\semtower{[]}{\semtower{∃y[\ml{boy} y ∧ []]}{y}}$} [{a boy$^{↑}$}]]
    [{$\semtower{∀x[\ml{girl} x → []]}{\semtower{[]}{λ y . y \ml{danceWith} x}}$} [{dance with every girl},roof]]
  ]
\end{forest}
\xe
\end{multicols}
\end{fullwidth}

Now we can collapse the tower by doing \textit{internal lower}, followed by
\textit{lower}:

\ex
\begin{forest}
  [{\fbox{$∀x[\ml{girl} x → (∃y[\ml{boy} y ∧ y \ml{danceWith} x])]$}}
  [{$↓$}
    [{$\semtower{∀x[\ml{girl} x → []]}{∃ x[\ml{boy} x ∧ y \ml{danceWith} x]}$}
      [{$⇊$}
        [{$\semtower{∀x[\ml{girl} x → []]}{\semtower{∃y[\ml{boy} y ∧ []]}{y \ml{danceWith} x}}$} [{a boy danced with every girl},roof]]
  ]]]]
\end{forest}
\xe



% \begin{multicols}{2}
% \ex \textit{Internal lift} (tower ver.)\\
% \(\left(\semtower{f []}{x}}\right)^{⇈} ≔ \semtower{f []}{\semtower{[]}{x}}\)
% \xe
% \columnbreak
% \ex \textit{External lift} (tower ver.)\\
% \(\left({\semtower{f []}{x}}\right)^{↑} ≔ \semtower{[]}{\semtower{f []}{x}}\)
% \xe

% \end{multicols}


\subsection{Variable free foundations}

Before we develop a story for crossover phenomena in continuation semantics, we
need a story about pronominal binding.

\citeauthor{barkerShan2015} adopt a version of Polly Jacobson's \acf{vfs} -- in
this section, we'll lay out the foundations.

The fundamental idea is that a sentence with a free pronoun denotes an open proposition.

\ex
$\eval{Jo likes him} \coloneq λ x . \ml{j likes }x$
\xe

How do we derive this compositionally?

\citet{jacobson1999} develops a theory of pronominals according to which they
denote the identity function on individuals -- this theory is known as \ac{vfs}:\sidenote{Pronouns also come with
  phi features, which must be interpeted -- we'll mostly abstract away from this
complication in what follows, but the most straightforward implementation is to
treat pronouns with phi features as denoting partial (i.e., presuppositional)
identity functions.

\ex
$\eval{her} ≔ λ x:\ml{identifies-fem} x . x$
\xe

}

\ex
$\ml{pro}_{Polly}  ≔ λ x . x$\hfill$\type{e → e}$
\xe

Pronominal meanings can compose with ordinary meanings via a type-shifter
(analogous to \textit{lift}) and a composition rule (analagous to
\ac{sfa}).\sidenote{This presentation of \ac{vfs} departs significantly from
  Jacobson and is based on \cite{charlow2018,charlow2019vfs}.}

\ex Pure (def.)\\
$a^{ρ} ≔ λ x . a$\hfill$ρ : \type{a \rightarrow e \rightarrow a}$
\xe

\ex Ap (def.)\sidenote{
Since \textit{ap} is defined in terms of bi-directional function application
($\ml{A}$), we have a forwards ap and a backwards ap, depending on whether the
function argument is on the left or the right. I give the type signatures of
both functions here.
}\\
$m ⊛ n ≔ λ x . (m x) \ml{A} (n x)$\hfill$\circledast : \type{(e → (a → b)) → (e → a) → e → b}$\\
\phantom{,}\hfill$\circledast : \type{(e → a) → (e → (a \rightarrow b)) → e → b}$
\xe

Composition of a sentence with a pronoun may now proceed via ap and pure --
non-pronominal meanings get pure-shifted, and composition proceeds via ap.

\ex
\begin{forest}
  [{$λ x . \ml{j likes} x$\\$⊛$}
    [{$λ x . \ml{j}$\\Jo$^{ρ}$}]
    [{$λ x . λ y . y \ml{likes} z$\\$⊛$}
      [{$λ x . λ zy . y \ml{likes} z$\\likes$^{ρ}$}]
      [{$λ x . x$\\$\ml{pro}_{Polly}$}]
    ]
  ]
\end{forest}
\xe

\begin{tcolorbox}
  An aside on \textbf{applicative functors}
  \tcblower
  As we've alluded to, there's a family resemblance between:

  \begin{itemize}

      \item The \textit{lift} of continuation semantics, and the \textit{pure}
      of \ac{vfs}

      \item \ac{sfa} from continuation semantics and the \textit{ap} of \ac{vfs}.

  \end{itemize}

  This is because both continuation semantics and \ac{vfs} implicitly use
  \textit{applicative functors} (\citealt{mcbridePaterson2008}) -- a notion from functional programming and
  category theory for characterizing composition in an enriched semantic domain.

  An applicative functor simply consists of a type-constructor, characterizing
  the enriched value-space, and two operations.
 
\end{tcolorbox}

\subsection{Going Scopal}

How do we get pronouns to interact with scope-takers in our current setting?
\citeauthor{barkerShan2015}'s solution is to treat pronouns \textit{themselves}
as scope-takers:\sidenote[][-10\baselineskip]{\citet{charlow2019vfs} provides a different way of
  incorporating \ac{vfs} and continuation semantics by composing applicative
  functors.

}

\begin{multicols}{2}

\ex Pronouns in continuation semantics\\
$\ml{pro}_{BS}  ≔ λ k . λ x . k x$
\xe

\multicols

\ex Pronouns (tower version)\\
$\ml{pro} ≔ \semtower{λ x . []}{x}$
\xe

\end{multicols}

The $\ml{pro}_{BS}$ denotation preserves the intuition of \ac{vfs} that pronouns
should be treated as identity functions, but the $λ x$ part \textit{takes scope}.\sidenote[][-10\baselineskip]{How do we derive the $BS$ pronoun denotation from the
  $Polly$ pronoun denotation? Explaining how requires borrowing a useful
  notion from functional programming/category theory.

  First, note that $\type{((→) e)}$ characterizes an \textit{enriched value
    space} -- essentially, the value space assumed in \ac{vfs}. Informally,
  meanings with an extra outer $λ x$ argument. $\type{((→) e)}$ is a \textit{functor},
  which means that we can characterize a general way of applying ordinary functions
  of type $\type{a → b}$ to values of type $\type{e → a}$. We'll call
  this mapping $\ml{map}$.

  \ex
  $\ml{map} m ≔ λ k . λ x . k (m x)$\\
  $\ml{map} : \type{(e \rightarrow a) \rightarrow (a \rightarrow b) \rightarrow e \rightarrow b}$
  \xe

  Applying $\ml{map}$ to $\ml{pro}_{Polly}$ gives back...$\ml{pro}_{BS}$!

}

Our current definition of \ac{sfa} is too rigidly typed to handle pronominal
scope takers, however. To see why, consider the type of $\ml{pro}_{BS}$:

\ex
$\ml{pro}_{BS} : \type{(e → t) → e → t}$
\xe

\ac{sfa} is designed to handle scope-takers of type
$\type{(a \rightarrow b) \rightarrow b}$, whereas a pronoun is a scope-taker of
type $\type{(a → b) → c}$:

\begin{itemize}
\item It \textit{expects} at a constituent of type
    $\type{t}$...

\item ...and returns something of type $\type{e → t}$.
\end{itemize}

It turns out, however, that we can
give our existing definition of \ac{sfa} a more general type in order to
accommodate pronominal scope-takers:\sidenote{If you had a go at the second
  problem set, and read the extra material from the second handout of the
  semester, then this idea should be familiar. In fact, generalizing our
  existing machinery to scope-takers of type $\type{(a → b) → c}$ receives
  independent motivation from DP-internal composition. We'll come back to this
  when we discuss scope out of DP and inverse linking later on.}

\ex
$m \ml{S} n ≔ λ k . m (λ x . n (λ y . k (x \ml{A} y)))$
\xe

\ex~
$\ml{S}: \type{(((a → b) → \textcolor{red}{r₁}) → r₂) → ((a → r₃) → \textcolor{red}{r₁}) → (b → r₃) → r₂}$
\xe

Just so long as the scope type of the left input and the return type of the
right input match, they cancel out.

We can model this idea of a generalized scope-taker using the type constructor
$\type{K_{r}^{i}}$:\sidenote{This ultimately goes back to \cite{wadler1994}.}

\ex
$\type{K_{r}^{i} a ≔ (a → i) → r}$
\xe

\citet{barkerShan2015} further generalize tower-type notation in order to accommodate
scope takers in which the expected and return types differ.\sidenote{See also \cite{shan2002}.}

\ex Tripartite tower types (def.)\\
$\tower{\type{r}}{\type{i}}{\type{a}} ≔ \type{(a → i) → r}$
\xe

We can think of our existing tower notation as an abbreviation for a tripartite
tower type, where the intermediate and final result types happen to be the same:

\ex Bipartite towers as abbreviations for tripartite towers\\
$\type{\semtower{r}{a} ≔ \tower{r}{r}{a}}$
\xe

We now have everything we need to accommodate pronominal scope-takers into our
compositional regime:

\ex
Jo likes him.
\xe

\begin{multicols}{2}

\ex
\begin{forest}
  [{$\type{\tower{e → t}{t}{t}}$\\$\ml{S}$}
    [{$\type{\tower{e → t}{e →t}{e}}$\\Jo$^{↑}$}]
    [{$\type{\tower{e → t}{t}{e → t}}$\\$\ml{S}$}
    [{$\type{\tower{e →t}{e → t}{e → e → t}}$\\likes$^{↑}$}]
      [{$\type{\tower{e → t}{t}{e}}$\\him}]
    ]
  ]
\end{forest}
\xe

\columnbreak

\ex
\begin{forest}
  [{\fbox{$\semtower{λ x . []}{\ml{j} \ml{likes} x}$}}
    [{$\semtower{[]}{\ml{j}}$}]
    [{$\semtower{λ x . []}{λ y . y \ml{likes} x}$}
      [{$\semtower{[]}{λxy . y \ml{likes} x}$}]
      [{$\semtower{λ x . []}{x}$}]
    ]
  ]
\end{forest}
\xe

\end{multicols}

Now that we have tripartite towers, we can also adopt a more general typing for
\textit{lower}, which simply requires that the inner type and the scope type are
both $\type{t}$.

\ex Lower (revised type)\\
$↓: \type{\tower{a}{t}{t} → \type{a}}$
\xe

The definition remains the same -- namely, when we \textit{lower} a continuized
value, we saturate the continuation argument with the identity function.

Observe that \textit{lowering} the result of scoping out $\ml{pro}_{BS}$ gives
back a \ac{vfs}-style sentential meaning.

\ex
$\left(\semtower{λ x . []}{\ml{j likes }x}\right)^{↓} = λ x . \ml{j likes }x$
\xe

\section{Achieving variable binding}

\citet{barkerShan2015} achieve \textit{binding} in their framework by
type-shifting the binder.

\ex Bind (def.)\\
$m^{B} ≔ λ k . m (λ x . k x x)$\hfill$B : \type{((a → b) → c) → (a → a → b) → c}$
\xe

Bind pulls out the inner value from a continuized meaning, returns a new
continuation with an extra argument saturated by the inner value. The tower
version is perhaps more intuitive:

\ex
Bind (tower ver.)\\
$\left(\semtower{f []}{x}\right) = \semtower{f ([] x)}{x}$\hfill$B: \type{\tower{c}{b}{a} \rightarrow \tower{c}{a → b}{a}}$
\xe

We'll illustrate with a quantifier, such as \textit{every boy}:

\ex
$\left(\semtower{∀x[\ml{boy} x → []]}{x}\right)^{B} = \semtower{∀ x[\ml{boy} x → ([] x)]}{x}$.
\xe

$B$-shifted \textit{every boy} expects an \textit{open
  proposition}; pronominals create open propositions.

We now have everything we need in order to account for a simple case of variable binding.

\ex
Every boy$^{x}$ loves his$_{x}$ mother.
\xe

\begin{fullwidth}
  \begin{multicols}{2}

\ex
Step 1: scope out pronoun\\
\begin{forest}
  [{\fbox{$\semtower{λ z . []}{λ y . y \ml{loves} ιx[x \ml{mother} z]}$}\\\ml{S}}
    [{loves$^↑$}]
    [{$\semtower{λ z . []}{ιx[x \ml{mother} z]}$\\\ml{S}}
      [{$\semtower{λ z . []}{z}$\\$\ml{pro}_{BS}$}]
      [{$λ z . ι x[x \ml{mother} z]$\\\ml{A}}
         [{'s}]
         [{mother}]
      ]
    ]
  ]
\end{forest}
\xe

\ex~
Step 2: Compose bind-shifted subject\\
\begin{forest}
  [{\fbox{$\semtower{∀ y[\ml{boy} y → ((λ z . []) y)]}{y \ml{loves} ιx[x \ml{mother} z]}$}\\$\ml{S}$}
    [{$\semtower{∀ y[\ml{boy} y → ([] y)]}{y}$\\every boy$^B$}]
    [{$\semtower{λ z . []}{λ y . y \ml{loves} ιx[x \ml{mother} z]}$} [{loves his mother},roof]]
  ]
\end{forest}
\xe


\columnbreak

\ex
Step 3: Lower the result\\
\begin{forest}
  [{\fbox{$∀y[\ml{boy} y → (y \ml{loves} ιx[x \ml{mother} y])]$}}
  [{(reduce)}
  [{$∀y[\ml{boy} y → ((λ z . y \ml{loves} ιx[x \ml{mother} z]) y)]$}
  [{$↓$}
  [{$\semtower{∀ y[\ml{boy} y → ((λ z . []) y)]}{y \ml{loves} ιx[x \ml{mother} z]}$}
    [{every boy loves his mother},roof]
  ]
  ]
  ]]]
\end{forest}
\xe

\ex~
Types:\\
\begin{forest}
  [{$\type{t}$}
    [{$↓$}]
    [{$\type{\semtower{t}{t}}$}
      [{$\type{\tower{t}{e → t}{e}}$\\everyone$ᴮ$}]
      [{$\type{\tower{e \rightarrow t}{t}{e → t}}$}
        [{$\type{\semtower{e \rightarrow t}{e → e → t}}$\\loves$^{\uparrow}$}]
        [{$\type{\tower{e \rightarrow t}{t}{e}}$} [{his mother},roof]]
      ]
    ]
  ]
\end{forest}
\xe

\end{multicols}

\end{fullwidth}

\begin{tcolorbox}
  \textbf{Intuition}\\
  \tcblower
  Pronouns \textit{expect a proposition and return an open proposition},
  whereas bind-shifted quantifiers \textit{expect an open proposition and return
  a proposition}.
\end{tcolorbox}

\subsection{Getting the basic facts}

Due to the \textit{left-to-right bias} of \ac{sfa} \ac{wco}-violating readings
can never be fully lowered, assuming our basic inventory of combinators (we put
higher-order combinators such as internal lift to one side for the time being).

\begin{tcolorbox}
\textbf{Assumption}\\
A sentence is deemed felicitous only if computing its meaning results in a value
of a lowerable type.
\end{tcolorbox}

To illustrate, let's try to compute the meaning of a \ac{wco} violating
sentence, and see how far we get:

\ex
\ljudge{*}His$_{x}$ mother loves every boy$^{x}$.
\xe


First, we compute the value of the VP, first bind-shifting the quantifier:

  \begin{multicols}{2}
\ex Composition\\
\begin{forest}
  [{\fbox{$\semtower{∀x[\ml{boy} x → ([] x)]}{λ y . y \ml{loves} x}$}\\$\ml{S}$}
    [{loves$^{\uparrow}$}]
    [{every boy$^{B}$}]
  ]
\end{forest}
\xe

\columnbreak

\ex Types:\\
\begin{forest}
  [{$\type{\tower{t}{e \rightarrow t}{e \rightarrow t}}$}
    [{$\type{\semtower{t}{e \rightarrow e \rightarrow t}}$}]
    [{$\type{\tower{t}{e \rightarrow t}{e}}$}]
  ]
\end{forest}
\xe

\end{multicols}

Next, let's try to compose the pronoun. Since the effect of the pronoun (the
$λ z$) gets processed \textit{before} the effect of the bind-shifted quantifier,
binding is not and \emph{can not} be achieved.

\begin{fullwidth}
\begin{multicols}{2}
  \ex
  \begin{forest}
    [{\fbox{$\semtower{λ z . ∀x[\ml{boy} x → ([] x)]}{ιy[y \ml{mother} z] \ml{loves} x}$}}
      [{$\semtower{λ z . []}{ιy[y \ml{mother} z]}$} [{his mother},roof]]
      [{$\semtower{∀x[\ml{boy} x → ([] x)]}{λ y . y \ml{loves} x}$} [{loves every boy},roof]]
    ]
  \end{forest}
  \xe
  \columnbreak
  \ex
  \begin{forest}
    [{$\type{\semtower{e \rightarrow t}{t}}$}
      [{$\type{\tower{e \rightarrow t}{t}{e}}$}]
      [{$\type{\tower{t}{e \rightarrow t}{e \rightarrow t}}$}]
    ]
  \end{forest}
  \xe
\end{multicols}
\end{fullwidth}

Furthermore, the resulting meaning is of an \textit{unlowerable type} -- it
expects an open proposition, and returns an open proposition. This is the basic account of crossover in \citet{barkerShan2015}.

\subsection{Binding out of DP}

It's worth noting that, since continuation semantics can straightforwardly account for scope out of DP,
it can account for binding out of DP.

Bona fide scope out of DP is independently motivated in any case:

\ex
{}[[No boy]$^{x}$'s mother] gave him$_{x}$ anything to read.
\xe

Consider a simple example of binding out of DP:

\ex
Every boy$^{x}$'s mother loves him$_{x}$.
\xe

\begin{fullwidth}
\begin{multicols}{2}

\ex
Step 1: VP composition:\\
\begin{forest}
  [{\fbox{$\semtower{λ z . []}{λ y . y \ml{loves} z}$}}
    [{loves$^{\uparrow}$}]
    [{$\ml{pro}_{BS}$}]
  ]
\end{forest}
\xe

\columnbreak

\ex
Step 2: Subject composition:\\
\begin{forest}
[{\fbox{$\semtower{∀x[\ml{boy} x → ([] x)]}{ιy[y \ml{mother} x]}$}}
   [{$\semtower{∀x[\ml{boy} x → ([] x)]}{x}$} [{every boy$^{B}$},roof]]
   [{$\semtower{[]}{λ x . ιy[y \ml{mother} x]}$}
   [{$↑$}
   [{...}
     [{'s}]
     [{mother}]
   ]
]]]
\end{forest}
\xe

\end{multicols}

\ex~
Step 3: compose and lower result\\
\begin{forest}
  [{\fbox{$∀x[\ml{boy} x → (ιy[y \ml{mother} x] \ml{loves} x)]$}}
  [{$↓$}
  [{$\semtower{∀x[\ml{boy} x → ((λ z . []) x)]}{ιy[y \ml{mother} x] \ml{loves} z}$}
    [{$\semtower{∀x[\ml{boy} x → ([] x)]}{ιy[y \ml{mother} x]}$} [{every boy's mother},roof]]
    [{$\semtower{λ z . []}{λ y . y \ml{loves} z}$} [{loves him},roof]]
  ]]]
\end{forest}
\xe

\end{fullwidth}


\subsection{Can inverse scope obviate crossover?}

Since continuation semantics has a mechanism for obviating the left-to-right
bias -- namely, \textit{internal lift}, which allows QPs to take inverse scope
-- we have to ensure that internal lift doesn't accidentally allow us to obviate crossover.

Let's therefore try to bind-shift a crossover, and then internally lift it, to
try to derive the unattested bound reading for \textit{his sister loves every
  boy}:

\ex
\ljudge{*}His$_{x}$ sister loves every boy$^{x}$.
\xe

\begin{fullwidth}
  \begin{multicols}{2}

\ex
Step 1: compose the VP meaning\\
\begin{forest}
  [{\fbox{$\semtower{∀x[\ml{boy} x → ([] x)]}{\semtower{[]}{λ y . y \ml{loves} x}}$}\\\ml{S}$_{2}$}
    [{loves$^{\uparrow_{2}}$}]
    [{$\semtower{∀x[\ml{boy} x → ([] x)]}{\semtower{[]}{x}}$\\every boy$^{\intLift ∘ B}$}]
  ]
\end{forest}
\xe

\columnbreak

\ex
Step 2: compose the subject\\
\begin{forest}
  [{\fbox{$\semtower{∀x[\ml{boy} x → ([] x)]}{\semtower{λ z . []}{ιy[y \ml{sister} z] \ml{loves} x}}$}}
  [{$\semtower{[]}{\semtower{λ z . []}{ιy[y \ml{sister} z]}}$}
    [{his sister$^{↑}$},roof]
  ]
    [{$\semtower{∀x[\ml{boy} x → ([] x)]}{\semtower{[]}{λ y . y \ml{loves} x}}$} [{loves every boy},roof]]
  ]
\end{forest}
\xe

\end{multicols}
\end{fullwidth}

The meaning we've computed looks fairly reasonable. Let's consider its type:

\ex
$\left(\semtower{∀x[\ml{boy} x → ([] x)]}{\semtower{λ z . []}{ιy[y \ml{sister} z] \ml{loves} x}}\right):\type{\tower{t}{e → t}{\tower{e → t}{t}{t}}}$
\xe

We can internally lower once, since on the bottom two stories, the expected type
matches the inner type, and we get:

\ex
$\type{\tower{t}{e \rightarrow t}{e \rightarrow t}}$
\xe

There \textit{is} actually a sensible way to lower the result \textit{again}, in
such a way as to achieve binding, but in order to do this we'd need minimally a lower of type $\type{(e \rightarrow t) \rightarrow e \rightarrow t}$.

At the cost of being unable to treat lower simply as a polymorphic identity
function, \citeauthor{barkerShan2015} conjecture that the grammar simply doesn't
make a lower of the relevant type available -- lower is rigidly typed, such that
it only applies to propositions.

This is at the core of \citeauthor{barkerShan2015}'s account of \ac{wco} -- the
payoff is a quantifier can only bind a pronoun on the same tower story.

It's worth noting at this point that \citeauthor{barkerShan2015}'s makes an
accurate prediction -- \ac{wco} is about \textit{scope}, not c-command -- in
continuation semantics, a quantifier can bind a pronoun just in case it's
processed first.

This captures the fact that possessors can bind out of the DP:

A problem with continuation semantics and \ac{vfs}: no obvious account of inverse
linking.

\section{Variables strike back}

Does an explanation of crossover using continuations require a
    commitment to \ac{vfs}?

Here I'll show that it doesn't. Their account is fully compatible with the
\enquote{standard picture}.

\subsection{The \enquote{standard} picture}

\ac{vfs} has been argued to have some conceptual advantages, but it's far more common to treat pronouns as \textit{variables}.

According to the standard picture (e.g., \citealt{heimKratzer1998}), pronouns
are indexed and acquire their value relative to the evaluation assignment (the
interpretation function is parameterized to an assignment).

\ex
$\eval[g]{her$_1$}  ≔ g₁$
\xe

In the following I'll try to see if we can retain the essence of the
\citeauthor{barkerShan2015} account of crossover in this more standard picture.

The first move I'll main is to extensionalize the standard picture, i.e., for
us, pronouns will be \textit{functions from assignments to individuals}:

\ex Pronouns (first ver.)\\
$\ml{pro}ₙ ≔ λ g . gₙ$\hfill$\ml{pro}ₙ : \type{g → e}$
\xe

Now, we can lift pronouns into scope takers in the same way as
\citeauthor{barkerShan2015} lift the \ac{vfs} pronoun into a
scope-taker:\sidenote{Implicitly, we're using the $\ml{map}$ of type
  $\type{g a → (a → b) → g b}$.}

\ex
Pronouns (second ver.)\\
$\ml{pro}ₙ ≔ \semtower{λ g . []}{gₙ}$\hfill$\ml{pro}_{n}:\type{\tower{g → t}{t}{e}}$
\xe

If we try to compute the meaning of a simple sentence such as \textit{Jo loves him}, and lower the result, with this pronominal meaning, it should be obvious
that what we get is the classical meaning extensionalized.

\ex
$\eval{Jo loves him$₁$} = λ g . \ml{j} \ml{loves} g₁$
\xe

How do we shift a QP into a binder? Intuitively, this should involve taking
something that expects (and returns) a proposition, and returns something that
expects (and returns) an \textit{assignment sensitive} proposition.\sidenote{One
advantage of going extensional is therefore a fully categorematic treatment of
abstraction; there is no need for a syncategorematic rule, such as
\citeauthor{heimKratzer1998}'s \textsc{Predicate Abstraction}.

An extensional account of assignment sensitivity provides a
semantic account of binding reconstruction, although I won't go into the details
here. It also has been argued to
be necessary to resolve issues in the theory of \textsc{acd} by, e.g., Kennedy (2014).
}

\ex
Abstract (def.)\\
$Λ_{n} m ≔ λ k . λ g . m (λ x . k x g^{[n → x]})$\hfill$Λₙ : \type{\semtower{t}{e} → \semtower{g → t}{e}}$
\xe

Abstract takes a QP, pulls out its inner value, and returns a scope-taker that (i)
expects an \textit{assignment-sensitive proposition}, feeds in a modified
assignment, and then re-abstracts over it.

In tower form:

\ex
$\Lambda_{n} \left(\semtower{f []}{x}\right) = \semtower{λ g . f ([] g^{[n → x]})}{x}$
\xe

Now we have everything we need to achieve binding. The computation is pretty
much isomorphic to what we had in \ac{vfs}.

\newpage

\ex
Every boy loves his mother.
\xe

\begin{fullwidth}
\begin{multicols}{2}
\ex Composition\\
\begin{forest}
  [{\fbox{$λg . ∀y[\ml{boy} y → y \ml{loves} ιz[z \ml{mother} y]]$}}
  [{$↓$}
  [{$\semtower{λ g . ∀y[\ml{boy} y → []]}{y \ml{loves} ιz[z \ml{mother} g₁^{[1 → y]}]}$}
  [{(reduce)}
  [{$\semtower{λ g . ∀y[\ml{boy} y → ([λ g . []] g^{[1 → y]})]}{y \ml{loves} ιz[z \ml{mother} g₁]}$}
    [{$\semtower{λ g . ∀y[\ml{boy} y → ([] g^{[1 → y]})]}{y}$} [{every boy$^{Λ₁}$},roof]]
    [{$\semtower{λ g . []}{λ y . y \ml{loves} ιz[z \ml{mother} g₁]}$} [{loves his$₁$ mother},roof]]
  ]]]]]
\end{forest}
\xe
\columnbreak
\ex Types\\
\begin{forest}
  [{$\type{g → t}$}
  [{$\type{\tower{g \rightarrow t}{t}{t}}$}
    [{$\type{\semtower{g \rightarrow t}{e}}$}]
    [{$\type{\tower{g \rightarrow t}{t}{e \rightarrow t}}$}]
  ]]
\end{forest}
\xe
\end{multicols}
\end{fullwidth}

Let's also check that we don't accidentally feed binding via internal lift.
Assuming that \textit{lower} is rigidly typed to truth values, lowering the result of this is impossible.

\newpage

\ex
His$_{x}$ mother loves every boy$^{x}$.
\xe

\begin{fullwidth}

  \begin{multicols}{2}

\ex~ Composition\\
\begin{forest}
  [{\xmark}
  [{$\semtower{λ g . ∀y[\ml{boy} y → ([] g^{[1 →
            y]})]}{λ g . ιz[z \ml{mother} g₁] \ml{loves} y}$}
  [{internal lower}
  [{$\semtower{λ g . ∀y[\ml{boy} y → ([] g^{[1 →
            y]})]}{\semtower{λ g . []}{ιz[z \ml{mother} g₁] \ml{loves} y}}$}
    [{$\semtower{[]}{\semtower{λ g . []}{ιz[z \ml{mother} g₁]}}$} [{his mother$^{↑}$}]]
    [{$\semtower{λ g . ∀y[\ml{boy} y → ([] g^{[1 →
            y]})]}{\semtower{[]}{λz . z \ml{loves} y}}$}
      [{$\semtower{[]}{\semtower{[]}{λ yz . z \ml{loves} y}}$\\loves$^{\uparrow_{2}}$}]
      [{$\semtower{λ g . ∀y[\ml{boy} y → ([] g^{[1 →
            y]})]}{\semtower{[]}{y}}$} [{every boy$^{\intLift \circ Λ_{1}}$}]]
    ]
  ]]]]
\end{forest}
\xe

\columnbreak

\ex~ Types\\
\begin{forest}
  [{\xmark}
  [{$\type{\semtower{g \rightarrow t}{g \rightarrow t}}$}
  [{internal lower}
  [{$\type{\semtower{g → t}{\tower{g \rightarrow t}{t}{t}}}$}
    [{$\type{\semtower{g → t}{\tower{g \rightarrow t}{t}{e}}}$}]
    [{$\type{\semtower{g \rightarrow t}{\semtower{t}{e \rightarrow t}}}$}
      [{$\type{\semtower{g \rightarrow t}{\semtower{t}{e \rightarrow e \rightarrow t}}}$}]
      [{$\type{\semtower{g \rightarrow t}{\semtower{t}{e}}}$}]
    ]
  ]]]]
\end{forest}
\xe

\end{multicols}

\end{fullwidth}

The primary intuition of \citeauthor{barkerShan2015}'s account can therefore be
maintained.

\subsection{A-movement}

As we've seen, overt A-movement \textit{bleeds} \ac{wco} whereas overt
A'-movement \textit{feeds} \ac{wco}.

Concretely, overt A-movement bleeds \ac{wco} only when it feeds scope --
(\ref{ex:a}) only has a reading on which \textit{every boy} takes scope over the
raising predicate \textit{likely}.

\ex
Every boy seems to his mother to be happy.\\
\phantom{,}\hfill\cmark $∀ > \ml{likely}$; \xmark $\ml{likely} > ∀$\label{ex:a}
\xe

How do we account for the exceptionality of A-movement wrt
crossover?\sidenote{As far as I can see, \citet{barkerShan2015} don't have much
  to say about this.}

What I'd like to suggest here is that A-traces are really a distinct lexical
item, and denote, essentially, a \ac{vfs} style pronoun.

\ex
$\ml{trace}_{A} ≔ \semtower{λ x . []}{x}$\hfill$\type{\tower{e → t}{t}{e}}$
\xe

A-traces are scoped out, and lowered -- A-raised DPs are base-generated in their
surface position.

Let's see how this derives A-movement bleeding crossover for (\ref{ex:a}).

\newpage

\ex
Every boy seems to his mother to be happy.
\xe

\ex Step 1: composition of matrix VP\\
\begin{forest}
  [{\fbox{$\semtower{λ g . []}{λ x . \ml{seems-to} (ιz[z \ml{mother} g₁]) (\ml{happy} x)}$}}
  [{(internal lower)}
  [{$\semtower{λ g . []}{\semtower{λ x . []}{\ml{seems-to} (ιz[z \ml{mother} g₁]) (\ml{happy} x)}}$}
    [{$\semtower{λ g . []}{\semtower{[]}{ιz[z \ml{mother} g₁]}}$} [{his mother$^{\intLift}$},roof]]
    [{$\semtower{[]}{\semtower{λ x . []}{λ z . \ml{seems-to} z (\ml{happy} x)}}$}
    [{$↑$}
    [{$\semtower{λ x . []}{λ z . \ml{seems-to} z (\ml{happy} x)}$}
      [{$\semtower{[]}{λ p . λ z . \ml{seems-to} z p}$\\seems-to$^{↑}$}]
      [{$\semtower{λ x . []}{\ml{happy} x}$}
        [{$\semtower{λ x . []}{x}$\\$t_{A}$}]
        [{$\semtower{[]}{λ x . \ml{happy} x}$}
          [{to be happy$^↑$},roof]
        ]
      ]
    ]
  ]]]]]
\end{forest}
\xe

\newpage

\ex~
Step 2: Compose \enquote{A-moved} DP\\
\begin{forest}
  [{\fbox{$λg . ∀x[\ml{boy} x → (\ml{seems-to} (ιz[z \ml{mother} g₁^{[1 → x]}]) (\ml{happy} x))]$}}
  [{$↓$}
  [{$\semtower{λ g . ∀x[\ml{boy} x → ((λ g . []) g^{[1 → x]})]}{\ml{seems-to} (ιz[z \ml{mother} g₁]) (\ml{happy} x)}$}
    [{$\semtower{λ g . ∀x[\ml{boy} x → ([] g^{[1 → x]})]}{x}$} [{every boy$^{Λ₁}$},roof]]
    [{$\semtower{λ g . []}{λ x . \ml{seems-to} (ιz[z \ml{mother} g₁]) (\ml{happy} x)}$} [{seems to his mother $t_{A}$ to be happy},roof]]
  ]]]
\end{forest}
\xe

This approach has a virtue -- it explains why A-moved expressions can scope
higher than their surface position (A-movement doesn't fix scope).

This is illustrated by the following example.

\ex
Some boys wants every girl$ˣ$ to seem to her$ₓ$ mother to be happy.\hfill$∀ > ∃$
\xe

How do we account for the fact that A'-movement \textit{doesn't} bleed
crossover? We can assume that A'-movers are genuinely interpreted as
scope-takers, in their base position; on the syntactic side, the phonological
features of the expression are displaced.\sidenote{See my manuscript
  \cite{elliott2019movement} for a theory of overt movement to this effect.}
In general, we expect A-movement to feed A'-movement – if we think of
A'-movement as scope-taking with a phonological reflex.

\ex
Which boy seemed to be happy?
\xe

Interestingly, it also derives the ban on \textit{improper movement} as a matter
of the semantics (A-movement may not feed A'-movement). Why? This is because an
$t_{A}$ trace is a lexical item which must saturate an argument position.

\ex
\ljudge{*}John seems that is intelligent.
\xe

\section{Inverse linking in continuation semantics}

How can we get the inverse scope reading for the following, while maintaining
the assumption that DP is a scope island?

\ex
A boy from every city is attending.\label{ex:inverse}
\xe

Evidence that DP is (in some sense) a scope island, comes from Larson's
generalization.\sidenote{See \cite{Sauerlanda} for critical discussion, and
  \cite{Charlow} for a response.}

\ex
Two politicians spy on someone from every city.\\
\phantom{,}\hfill\cmark $\forall > \exists > 2$\\
\phantom{,}\hfill\cmark $2 > \forall > exists$\\
\phantom{,}\hfill\xmark $\forall > 2 > exists$
\xe


We'll assume a completely standard semantics for determiners as Generalized
Quantifiers. The semantics for \textit{every} is given below.

It's a function
from a predicate to a continuized individual.

\ex
$\eval{every} = λ r . λ c . ∀ x[r x → c x]$\hfill$\type{(e → t) → \semtower{t}{e}}$
\xe

How does \textit{every} compose with its restrictor? Well, since its restrictor
is a syntactically simplex nominal, composition can proceed via function application.

\ex
Step 1: Compose \textit{every} and its restrictor
\begin{forest}
  [{$λ c . ∀ x[\ml{city} x → c x]$\\$\ml{A}$}
    [{$λ r . λ c . ∀ x[r x → c x]$}]
    [{$λ y . \ml{city} y$}]
  ]
\end{forest}
\xe

Since \textit{every city} is a scope taker, composition of the PP and containing NP is mediated via \textit{lift} and \ac{sfa}.\footnote{As well as \ac{sfa}, the derivation in (\ref{ex:step2}) makes use of an additional composition rule: the scopal counterpart of \textit{predicate modification}, written here as $\ml{S}_∧$.}

\ex
Step 2: composition of NP\\
\begin{forest}
  [{$λ c . ∀y[\ml{city} y → c (λ x . \ml{boy} x ∧ x \ml{from} y)]$\\$\ml{S}_∧$}
    [{$λ c . c (λ x . \ml{boy} x)$\\boy$^↑}]
    [{$\ml{S}$}
      [{$λ c . c (λ yx . x \ml{from} y)$\\from$^↑$}]
      [{$λ c . ∀ y[\ml{city} y → c y]} [{every city},roof]]
    ]
  ]
\end{forest}\label{ex:step2}
\xe

Finally, we need to compose the indefinite determiner with its restrictor. We assume that, much like \textit{every}, \textit{a} is a GQ. Its denotation is given below:

The restrictor of the indefinite is itself associated with a scopal
\textit{side-effect}, as reflected by its type. The indefinite is therefore unable to
compose with its complement via \ml{A} or \ml{S}. In order to resolve the type
mismatch, we must first lift the determiner, at which point composition can
proceed via \ml{S}.

\ex
\begin{forest}
  [{\fbox{$λ c . ∀ y[\ml{city} y → c (λ s . ∃x[\ml{boy} x ∧ x \ml{from} y ∧ s x])]$}\\$\ml{S}$}
    [{$λ c . c (λ r . λs . ∃x[r x → s x])$\\a$^{↑}$}]
    [{$λ c . ∀y[\ml{city} y → c (λ x . \ml{boy} x ∧ x \ml{from} y)]$} [{boy from every city},roof]]
  ]
\end{forest}
\xe

At this stage in the derivation, the DP denotes an individual with two layers of scopal side-effects -- the higher corresponding to the universal, and the lower corresponding the existential.

\ex
$\semtower{∀y[\ml{city} y → []]}{\semtower{∃x[\ml{boy} x ∧ x \ml{from} y ∧ []]}{x}}$
\xe

\citet{barkerShan2015} (see also \citealt{Charlowc}) typically use
\textit{internal lower} to collapse a three-story tower. There is,
however, another way of collapsing a three-story tower \textit{implicit} in our
existing machinery.

We're now going to define a new operation for lowering a value of $m$ of type
$\type{K_t (K_t e)}$, called \textit{join} (written $μ$). Joining $m$ is simply an
instruction to \textit{compose} $m$ with lift:\sidenote{Why does this work? Let's start by de-sugaring the type of $m$: $\type{K_t (K_t e)}$ -- this is an
abbreviation for the following type:

\ex $m: \type{(((e → t) → t) → t) → t}$\xe

This type is amenable to a distinct sugaring
in terms of $\type{K_t}$ -- namely $m: \type{K_t (e → t) → t}$.

Now, consider the type of lift: $\type{a → K_t a}$. Since lift is polymorphic,
$\type{a}$ could be any type. Let's instantiate it as $\type{e → t}$ -- the
corresponding lift function is of type $\type{(e → t) → K_t (e → t)}$.

Note that
the output of lift on $\type{e → t}$ is the same as the input to our re-sugaring
of $\type{K_t (K_t e)}$ into $\type{K_t (e → t) → t}$. This means we can do
\textit{function composition}. The result of composing $m$ and lift should be a function of
type $\type{(e → t) → t}$ (i.e. $\type{K_t e}$).}

\ex
Join (def.)\\
$m^{μ} ≔ m ∘ (↑)$
\xe

We can now take the meaning we've computed for \textit{a boy from every city}
and lower it into an ordinary tower via \textit{join} -- as you can see, join
takes a three-story tower, and sequences scopal side-effects from top-to-bottom:

\ex
\begin{forest}
  [{\fbox{$λ s . ∀y[\ml{city} y → (∃x[\ml{boy} x ∧ x \ml{from} y ∧ s x])]$}}
  [{$μ$}
    [{$λ c . ∀ y[\ml{city} y → c (λ s . ∃x[\ml{boy} x ∧ x \ml{from} y ∧ s x])]$}]
  ]
  ]
\end{forest}
\xe

\ex
\begin{forest}
  [{$\type{\semtower{t}{e}}$}
  [{$μ$}
  [{$\type{\semtower{t}{\semtower{t}{e}}}$}
    [{$\type{\semtower{t}{(e → t) → \semtower{t}{e}}}$\\a$^{↑}$}]
    [{$\type{\semtower{t}{(e → t)}}$\\$\ml{S}_{∧}$}
      [{$\type{\semtower{t}{(e → t)}}$\\boy$^{↑}$}]
      [{$\type{\semtower{t}{(e → t)}}$\\$\ml{S}$}
        [{$\type{\semtower{t}{(e → e → t)}}$\\from$^{↑}$}]
        [{$\type{\semtower{t}{e}}$\\$\ml{A}$}
          [{$\type{(e → t) → \semtower{t}{e}}$\\every}]
          [{$\type{e → t}$\\city}]
        ]
      ]
    ]
  ]
  ]
  ]
\end{forest}
\xe

In order to capture Larson's generalization, we conjecture that DPs are a kind
of scope island, distinct to the \textit{evaluation islands} discussed by
\citet{Charlowc}:

\ex
DP type rigidity (def.)\\
A DP must denote a value of type $\type{K_t e}$ before the derivation can proceed.
\xe

From this constraint, it follows straightforwardly that, if a quantificational
determiner has a GQ in its complement, then the two must scope together.

I leave it as an exercise to combine our account of binding in the standard picture with the account of inverse linking outlined here.

\printbibliography


\end{document}
