\documentclass[nols,twoside,nofonts,nobib,nohyper]{tufte-handout}

\usepackage{fixltx2e}
\usepackage{tikz-cd}
\usepackage{tcolorbox}
\usepackage{appendix}
\usepackage{listings}
\lstset{language=TeX,
       frame=single,
       basicstyle=\ttfamily,
       captionpos=b,
       tabsize=4,
  }

\begin{acronym}
\acro{sfa}{Scopal Function Application}
\acro{fa}{Function Application}
\acro{wco}{Weak Crossover}
\acro{ScoT}{Scope Transparency}
\acro{vfs}{Variable Free Semantics}
\acro{acd}{Antecedent Contained Deletion}
\acro{qr}{Quantifier Raising}
\acro{doc}{Double Object Construction}
\end{acronym}

\renewcommand*{\acsfont}[1]{\textsc{#1}}

\makeatletter
% Paragraph indentation and separation for normal text
\renewcommand{\@tufte@reset@par}{%
  \setlength{\RaggedRightParindent}{0pt}%
  \setlength{\JustifyingParindent}{0pt}%
  \setlength{\parindent}{0pt}%
  \setlength{\parskip}{\baselineskip}%
}
\@tufte@reset@par

% Paragraph indentation and separation for marginal text
\renewcommand{\@tufte@margin@par}{%
  \setlength{\RaggedRightParindent}{0pt}%
  \setlength{\JustifyingParindent}{0pt}%
  \setlength{\parindent}{0pt}%
  \setlength{\parskip}{\baselineskip}%
}
\makeatother

\setcounter{secnumdepth}{3}

\title{Continuation semantics ii\thanks{24.979: Topics in
    semantics\\\noindent\textit{Getting high: Scope, projection, and evaluation order}}}

\author[Patrick D. Elliott and Martin Hackl]{Patrick~D. Elliott\sidenote{\texttt{pdell@mit.edu}} \& Martin Hackl\sidenote{\texttt{hackl@mit.edu}}}

\addbibresource[location=remote]{/home/patrl/repos/bibliography/elliott_mybib.bib}

\lingset{
  belowexskip=0pt,
  aboveglftskip=0pt,
  belowglpreambleskip=0pt,
  belowpreambleskip=0pt,
  interpartskip=0pt,
  extraglskip=0pt,
  Everyex={\parskip=0pt}
}


% \usepackage{booktabs} % book-quality tables
% \usepackage{units}    % non-stacked fractions and better unit spacing
% \usepackage{lipsum}   % filler text
% \usepackage{fancyvrb} % extended verbatim environments
%   \fvset{fontsize=\normalsize}% default font size for fancy-verbatim environments

% % Standardize command font styles and environments
% \newcommand{\doccmd}[1]{\texttt{\textbackslash#1}}% command name -- adds backslash automatically
% \newcommand{\docopt}[1]{\ensuremath{\langle}\textrm{\textit{#1}}\ensuremath{\rangle}}% optional command argument
% \newcommand{\docarg}[1]{\textrm{\textit{#1}}}% (required) command argument
% \newcommand{\docenv}[1]{\textsf{#1}}% environment name
% \newcommand{\docpkg}[1]{\texttt{#1}}% package name
% \newcommand{\doccls}[1]{\texttt{#1}}% document class name
% \newcommand{\docclsopt}[1]{\texttt{#1}}% document class option name
% \newenvironment{docspec}{\begin{quote}\noindent}{\end{quote}}% command specification environment

\begin{document}

\maketitle% this prints the handout title, author, and date

\begin{tcolorbox}
Homework
\tcblower
\begin{itemize}
\item Finish reading \cite{barkerShan2015} chapters 1 and 4, if you haven't already.
\item Read \cite{barkerShan2015} chapter 7.
\item Do p-set 2 (to be posted on Stellar later today)!
\end{itemize}
\end{tcolorbox}

\section{Roadmap}

This week, I'll finish introducing \textit{continuation semantics}. We'll
minimally try to cover:

\begin{itemize}

    \item The syntax-semantics interface (more explicitly, this time).

    \item Inverse scope via multi-story towers.

    \item Split scope.

    \item Scope islands as \textit{evaluation islands} + remarks on scope economy.

    \item Generalized con/dis-junction + \enquote{split scope} readings.

\end{itemize}

With potentially two extensions:

\begin{itemize}

    \item DP-internal composition and indexed continuations.

    \item Exceptionally-scoping indefinites via continuations (\citealt{Charlowc}).

\end{itemize}

\section{A note on syntax}

So far, I've been a little shy about saying explicitly what we're assuming here
about syntax, and what we're assuming about the syntax-semantics mapping.

I'll assume a derivational theory, according to which structures are built-up
via successive application of \textsc{Merge}.\sidenote{I'll often use
  \enquote{syntactic structure speak} when talking about trees. This is
  harmless, since they can always be interpreted as the graph of a syntactic
  derivation, especially since trees encode both structure and order.}

\ex
\begin{forest}
  [{\textsc{Merge}}
    [{..}]
    [{\textsc{Merge}}
      [{..}]
      [{\textsc{Merge}}
        [{...}]
        [{...}]
      ]
    ]
  ]
\end{forest}
\xe

I'll furthermore adopt the hypothesis that the syntactic derivation proceeds in
lockstep with the semantic computation. This conjecture, which goes back at
least to
\citet{montague1973}, is often described as \textit{direct
  compositionality}.\sidenote{Although direct compositionality is often
  associated with frameworks such as variable free semantics and Combinatory Categorial Grammar, it's in principle independent.
See, e.g., \citet{kobele2006} for an explicit formalization of a directly compositional minimalist grammar.}

Minimally, the formatives must be \textit{tuples} consisting of phonological
features and semantic features: (\texttt{phon}, \texttt{sem}, ...). Semantic
features could be cashed out as model theoretic objects, or perhaps as
expressions of the simply typed lambda calculus.

I'll assume that part of what \textsc{Merge} does is concatenate
phonological features. This is because \textsc{Merge} is just an instruction for
combining formatives. On the semantic side, it typically does function
application.\sidenote{It follows that \textsc{Merge} is a \textit{non-symmetric
    relation}, departing from, e.g., \citet{chomsky1995}, but consistent with
  \citet{stabler1997} and related work.}

\ex
$(𝕩,x) ∗ (𝕪,y) ≔ ([𝕩 𝕪], x \ml{A} y)$
\xe

It can also do concatenation of phonological features, plus \ac{sfa} of semantic
values (whence the left-to-right bias of \ml{S}).

\ex
$(𝕩,x) ∗ (𝕪,y) ≔ ([𝕩 𝕪], x \ml{S} y)$
\xe

I've define \ml{LIFT} as a purely semantic operation -- this is to be taken as
shorthand for an operation on a formative that only effects the semantic value:

\ex
$(𝕩,x)^{↑} ≔ (𝕩,x^{↑})$
\xe

This constitutes the basics of the system laid out in
\cite{elliott2019movement}. See \citeauthor{elliott2019movement} for an
elaboration of how to supplement this system with a feature
calculus, but for today's purposes we won't need any additional assumptions.

\subsection{Deriving inverse scope}

Last type we got as far as being able to derive \textit{surface scope} readings,
as well as scopal ambiguities arising via interactions with scopally-immobile
expressions.

We achieved this latter coverage by allowing \ml{LOWER} to fix the scope of a
quantifier at different points in the derivation.

We still don't have any way of accounting for scopal interactions between
multiple scopally-mobile expressions (i.e., quantifiers). Since every major
theory of quantifier scope is tailored to achieve this, we have a major problem
on our hands!

Fortunately, we already have all of the primitive operations we need in order to
achieve inverse scope readings.

Recall that \(\ml{LIFT}\) is a \textit{polymorphic function} -- it lifts a value
into a trivial tower:

\ex
$a^{↑} ≔ \semtower{[]}{a}$
\xe

Since \ml{LIFT} is polymorphic, in principle it can apply to any kind of value
-- even a tower! Let's flip back to lambda notation to see what happens.

\ex
$\eval{everyone} ≔ λ k . ∀x[k x]$\hfill $\type{(e → t) → t}$
\xe

\ex~
$\eval{everyone}^{↑} = λ l . l (λ k . ∀x[k x])$\hfill$\type{(((e → t) → t) → t) → t}$
\xe

Going back to tower notation, lifting a tower adds a trivial third
story:\sidenote{In fact, via successive application of \ml{LIFT}, we can
  generate an $n-$story tower.} Following \citet{Charlowc}, when we apply
\ml{LIFT} to a tower, we'll describe the operation as \textit{external lift}
(although, it's worth bearing in mind that this is really just our original
\ml{LIFT} function).

The third story of the tower corresponds to whatever takes scope over the outer
continuation variable (here, $l$), and the second story of the tower corresponds
to whatever takes scope over the inner continuation variable (here, $k$).

\ex
\(\left(\semtower{∀x[]}{x}\right)^{↑} = \semtower{[]}{\semtower{∀x[]}{x}}\)
\xe

One important thing to note is that, when we externally lift a tower, the
quantificational part of the meaning always remains on the same story relative
to the bottom story. Intuitively, this reflects the fact that, ultimately,
\ml{LIFT} alone isn't going to be enough to derive quantifier scope ambiguities.

\begin{tcolorbox}
\textbf{Question}
\tcblower
Which (if any) of the following bracketings make sense for a three-story tower:

\begin{multicols}{2}
\ex
$\semtower{\left(\semtower{f []}{g []}\right)}{x}$
\xe
\columnbreak
\ex
$\semtower{f []}{\left(\semtower{g []}{x}\right)}$
\xe
\end{multicols}
\end{tcolorbox}

The extra ingredient we're going to need in order to derive inverse scope, is the ability to sandwhich an empty
story into the \textit{middle} of our tower, pushing the quantificational part
of the meaning to the very top.

This is \textit{internal lift} ($⇈$).\sidenote{I
can tell what you're thinking: \enquote{seriously? Another \textit{darn}
  type-shifter? How many of these are we going to need?!}.
Don't worry, I got you. Even thought we've defined internal lift here as a
primitive operation, it actually just follows from our existing machinery.
Concretely, \textit{internal lift} is really just \textit{lifted} \ml{LIFT} (so
many lifts!). Lifted \ml{LIFT} applies to its argument via \ml{S}.

\ex
$\semtower{[]}{↑} \ml{S} \semtower{f []}{x} = \semtower{f []}{\semtower{[]}{x}}$
\xe

}

\pex
\textit{Internal lift} (def.)\\
\a \((⇈) : \type{K_{t} a → K_{t} (K_{t} a)}\)
\a \(m^{⇈} ≔ λ k . m (λ x . k x^{↑})\)
\xe

It's much easier to see what internal lift is doing by using the tower notation.
We can also handily compare its effects to those of \textit{external} lift.

\begin{multicols}{2}
\ex \textit{Internal lift} (tower ver.)\\
\(\left(\semtower{f []}{x}}\right)^{⇈} ≔ \semtower{f []}{\semtower{[]}{x}}\)
\xe
\columnbreak
\ex \textit{External lift} (tower ver.)\\
\(\left({\semtower{f []}{x}}\right)^{↑} ≔ \semtower{[]}{\semtower{f []}{x}}\)
\xe

\end{multicols}

Armed with \textit{internal} and \textit{external} lifting operations, we now
have everything we need to derive inverse scope. We'll start with a simple
example (\ref{ex:classic1}).

The trick is: we \textit{internally} lift the quantifier that is destined to
take wide scope.

\ex
A boy danced with every girl.\hfill $∀ > ∃$\label{ex:classic1}
\xe

Before we proceed, we need to generalize \ml{LIFT} and \ac{sfa} to three-story
towers.\sidenote{
Before you get worries about expanding our set of primitive operations, notice
that \textit{3-story lift} is just ordinary lift applied twice. \textit{3-story}
\ac{sfa} is just \ac{sfa}, but where the bottom story combines via \ml{S} not
\ml{A}. In fact, we can generalize these operations to $n-$story towers.
}

\begin{multicols}{2}
\ex
$x^{↑_{2}} ≔ \semtower{[]}{\semtower{[]}{x}}$
\xe

\columnbreak

\ex
$\semtower{f []}{m} \ml{S}_{2} \semtower{g []}{n} ≔ \semtower{f (g [])}{m \ml{S} n}$
\xe

\end{multicols}


\begin{fullwidth}
  \begin{multicols}{2}
\ex Step 1: internally lift \textit{every girl} \\
\begin{forest}
  [{\fbox{$\semtower{∀x[\ml{girl} x → []]}{\semtower{[]}{λ y . y \ml{danceWith} x}}$}\\$\ml{S}_{2}$}
    [{$\semtower{[]}{\semtower{[]}{\ml{danceWith}}}$} [{dance-with$^{↑_{2}}$}]]
    [{$\semtower{∀x[]}{\semtower{[]}{x}}$} [{$⇈$} [{every girl},roof]]]
  ]
\end{forest}
\xe

\columnbreak

\ex
Step 2: \textit{ex}ternally lift \textit{a boy}\\
\begin{forest}
  [{\fbox{$\semtower{∀x[\ml{girl} x → []]}{\semtower{∃y[\ml{boy} y ∧ []]}{y \ml{danceWith} x}}$}\\$\ml{S}_{2}$}
    [{$\semtower{[]}{\semtower{∃y[\ml{boy} y ∧ []]}{y}}$} [{a boy$^{↑}$}]]
    [{$\semtower{∀x[\ml{girl} x → []]}{\semtower{[]}{λ y . y \ml{danceWith} x}}$} [{dance with every girl},roof]]
  ]
\end{forest}
\xe
\end{multicols}
\end{fullwidth}

What we're left with now is a 3-story tower with the universal on the top story
and the existential on the middle story. We can collapse the tower by first
collapsing the bottom two stories, and then collapsing the result. In order to
do this, we'll first define \textit{internal lower}.\sidenote{Let's again
  address the issue of expanding our set of primitive operations (in what is
  becoming something of a theme). Internal lower is just lifted lower, applying
  via \ml{S}. In other words:

  \ex
  $m^{⇊} ≡ (↓)^{↑} \ml{S} m$
  \xe

}

\pex \textit{Internal lower} (def)
\a $(⇊): \type{K_{t} (K_{t} a) → K_{t} a}$
\a $m^{⇊} ≔ λ k . m (λ n . k n^{↓})$
\xe

\ex
\textit{Internal lower} (def.)
$\left(\semtower{f []}{\semtower{g []}{p}}\right)^{⇊} ≔ \semtower{f []}{\left(\semtower{g []}{p}\right)^{↓}}$
\xe

Now we can collapse the tower by doing \textit{internal lower}, followed by
\textit{lower}:

\ex
\begin{forest}
  [{\fbox{$∀x[\ml{girl} x → (∃y[\ml{boy} y ∧ y \ml{danceWith} x])]$}}
  [{$↓$}
    [{$\semtower{∀x[\ml{girl} x → []]}{∃ x[\ml{boy} x ∧ y \ml{danceWith} x]}$}
      [{$⇊$}
        [{$\semtower{∀x[\ml{girl} x → []]}{\semtower{∃y[\ml{boy} y ∧ []]}{y \ml{danceWith} x}}$} [{a boy danced with every girl},roof]]
  ]]]]
\end{forest}
\xe

Great! We've shown how to achieve quantifier scope ambiguities using our new
framework. Let's look at the derivations again side-by-side.

\begin{fullwidth}
\begin{multicols}{2}

  \ex
  Surface scope (schematic derivation)\\
  \begin{forest}
    [{$↓$}
    [{$\ml{S}$}
      [{$Q_{1}$}]
      [{$\ml{S}$}
        [{$R^{↑}$}]
        [{$Q_{2}$}]
      ]
    ]]
  \end{forest}
  \xe

  \columnbreak

  \ex
  Inverse scope (schematic derivation)\\
  \begin{forest}
    [{$↓$}
    [{$⇊$}
  [{$\ml{S}_{2}$}
    [{$Q_{1}^{↑}$}]
    [{$\ml{S}_{2}$}
      [{$R^{↑_{2}}$}]
      [{$Q_{2}^{⇈}$}]
    ]
  ]]]
  \end{forest}
  \xe

\end{multicols}
\end{fullwidth}

There's a couple of interesting things to note here:

\begin{itemize}

    \item The inverse scope derivation involves more applications of our
    type-shifting operations -- this becomes especially clear if we decompose
    the complex operations
    $\ml{S}_{2}$, $↑_{2}$, $⇈$, and $⇊$.

    \item In order to derive an inverse scope reading, what was \textit{crucial}
    was the availability of \textit{internal lift}; the remaining operations,
    $\ml{S}_{2}, ↑_{2}, ⇊$ only functioned to massage composition for
    three-story towers.

\end{itemize}

On the latter point, it's tempting to conjecture that in, e.g., German, Japanese
and other languages which \enquote{wear their LF on their sleeve}, the semantic
correlate of \textit{scrambling} is \textit{internal lift}, whereas in
scope-flexible languages such as English, internal lift is a freely available
operation.\sidenote{To make sense of this, we would of course need to say
  something more concrete about overt movement.
  For an attempt at marrying continuations to a standard, minimalist syntactic
component, see my manuscript \textit{Movement as higher-order structure building}.}

If we adopt some version of the \textit{derivational complexity hypothesis}\sidenote{I.e., that derivational complexity correlates
  with processing difficulty.}, we
also predict that inverse scope readings should take longer to process than
surface scope readings.

It's worth mentioning, incidentally, that although we collapsed the resulting
three-story tower via internal lower followed by lower, we can also define an
operation that collapses a three-story tower two an ordinary tower in a
different way. Let's call it \textit{join}:\sidenote{Join for three-story towers
corresponds directly to the \textit{join} function associated with the
continuation monad. For more on continuations from a categorical perspective,
see the appendix to the first handout.}

\ex \textit{join} (def.)\\
$m^{μ} ≔ λ k . m (λ c . c k)$\hfill$μ: \type{K_{t} (K_{t} a) → K_{t} a}$
\xe

In tower terms, join takes a three-story tower and sequences quantifiers from
top to bottom:

 \ex
  $
  \left(\semtower{f []}{\semtower{g []}{x}}\right)^{μ} = \semtower{f (g [])}{x}
  $
  \xe

Doing internal lower on a three-story tower followed by lower is equivalent to doing
join on a three-story tower followed by lower (as an exercise, convince yourself
of this). However, there's may be a good empirical reason for having internal lower as a
distinct operation (and since it's just lifted lower, it \enquote{comes for
  free} in a certain sense).

\ex
Daniele wants a boy to dance with every girl.\hfill $∀ > \ml{want} > ∃$\label{ex:dani1}
\xe

Arguably, (\ref{ex:dani1}) can be true if for every girl $x$, Daniele has the following
desire: \textit{a boy dances with $y$}. This is the reading on which
\textit{every boy} scopes over the intensional verb, and \textit{a boy} scopes
below it.

If we have \textit{internal lower}, getting this is easy. We \textit{internally
  lift} \textit{every girl} and \textit{externally lift} \textit{a boy}. Before
we reach the intensional verb, we fix the scope of \textit{a boy} by doing
internal lower. Now we have an ordinary tower, and we can defer fixing the scope
of \textit{every girl} via \textit{lower} until after the intensional
verb.

\begin{fullwidth}
  \begin{multicols}{2}
    \ex Step 1: scope \textit{every girl} over \textit{a boy}\\
    \begin{forest}
      [{\fbox{$\semtower{∀x[\ml{girl} x → []]}{\semtower{∃y[\ml{boy} y ∧ []]}{y \ml{{dance-with} x}}}$}\\$\ml{S}_2$}
        [{...} [{$↑$} [{a boy},roof]]]
        [{$\ml{S}_2$}
          [{dance with$^{↑_2}$}]
          [{...} [{$⇈$} [{every girl},roof]]]
        ]
      ]
    \end{forest}
    \xe
    \columnbreak
    \ex Step 2: internally lower below \textit{want}
    \begin{forest}
      [{\fbox{$\semtower{∀x[\ml{girl} x → []]}{∃y[\ml{boy} y ∧ y \ml{dance-with} x]}$}}
      [{$⇊$}
        [{$\semtower{∀x[\ml{girl} x → []]}{\semtower{∃y[\ml{boy} y ∧ []]}{y \ml{{dance-with} x}}}$}]
      ]
      ]
    \end{forest}
    \xe
  \end{multicols}
  \end{fullwidth}

    \ex
    Step 3: scope \textit{every girl} over \textit{want}\\
    \begin{forest}
      [{\fbox{$\semtower{∀x[\ml{girl} x → []]}{\ml{daniele} \ml{want} (∃y[\ml{boy} y ∧ y \ml{dance-with} x])}$}\\$\ml{S}$}
        [{Daniele$^↑$}]
        [{$\ml{S}$}
          [{wants$^↑$}]
          [{$\semtower{∀x[\ml{girl} x → []]}{∃y[\ml{boy} y ∧ y \ml{dance-with} x]}$} [{...},roof]]
        ]
      ]
    \end{forest}
    \xe

    \ex~
    Step 4: lower\\
    \begin{forest}
      [{\fbox{$∀x\left[\begin{aligned}[c]
            &\ml{girl} x\\
            &→ \left(\ml{daniele} \ml{want} \left(∃y\left[\begin{aligned}[c]
                &\ml{boy} y\\
                &∧ y \ml{dance-with} x
                \end{aligned}\right]\right)\right)
            \end{aligned}\right]$}}
      [{$↓$}
       [{$\semtower{∀x[\ml{girl} x → []]}{\ml{daniele} \ml{want} (∃y[\ml{boy} y ∧ y \ml{dance-with} x])}$}]
      ]
      ]
    \end{forest}
    \xe

If we only have \textit{join} then the scope of \textit{a boy} and \textit{every
girl} may vary amongst themselves, but they should either both scope below
\textit{want} or both scope above \textit{want}.}

Note that we haven't given a concrete treatment of intensionality here, but, mechanically,
we can simply replace every occurrence of $\type{t}$ with type $\type{s → t}$.
Intensional verbs take type $\type{s →
  t}$ complements. Here are some sample lexical entries:

\pex
\a
$\eval{every girl} ≔ λ k . λ w . ∀x[\ml{girl}_{w} x → k x]$\hfill$\type{(e → (s → t)) → s → t}$
\a
$\eval{dance with} ≔ λ yx . λ w . y \ml{dance-with}_{w} x$\hfill$\type{e → s → t}$
\a
$\eval{want} ≔ λ p . λ x . λ w . x \ml{want}_{w} p$\hfill$\type{(s → t) → e → s → t}$
\xe

Instead of giving back a return type $\type{t}$, in an intensional setting our continuation type
constructor is defined in terms of a return type $\type{s → t}$:

\ex
$\type{K_{s → t} a≔ (a → (s → t)) → s → t}$
\xe

You can verify for yourselves that the core features of the system remain
unaffected, i.e, the definitions of lift and \ml{S} can remain the same.

\section{Split scope}

In the first p-set, I asked you how to think about analyzing split scope of
non-upward-monotone quantifiers:

\ex
The company need fire no employees.\\
\textit{It's not the case that the company \textit{needs} to hire
  employees.}\hfill $¬ > □ > ∃$
\xe

With continuation semantics, we can understand this data as providing support
for the idea that expressions can denote three-story towers (something not
excluded by, e.g., \citealt{heimKratzer1998} in any case).

\ex
$\eval{no employees} ≔ λ k . ¬ k (λ l . ∃x[\ml{employee} x ∧ l x])$\hfill$\type{K_{t} (K_{t} a)}$
\xe

Tower version:

\ex
$\eval{no employees} ≔ \semtower{¬ []}{\semtower{∃x[\ml{employee} x ∧ []]}{x}}$
\xe

We get the split scope reading by doing internal lower first below the modal,
and then external lower above the modal.

\ex
\begin{forest}
  [{\fbox{$¬ (□ (∃x[\ml{company} x ∧ \ml{the-company fire }x]))$}}
  [{$↓$}
  [{$\semtower{¬ []}{□ (∃x[\ml{company} x ∧ \ml{the-company fire }x])}$\\\ml{S}$}
    [{need$^{↑}$}]
    [{$\semtower{¬ []}{∃x[\ml{company} x ∧ \ml{the-company fire }x]}$} [{$⇊$} [{$\ml{S}_{2}$}
      [{the company$^{↑_{2}}$}]
      [{$\ml{S}_{2}$}
        [{fire$^{↑_{2}}$}]
        [{$\semtower{¬ []}{\semtower{∃x[\ml{employee} x ∧ []]}{x}}$}]
      ]
    ]
  ]]]]]
\end{forest}
\xe

One interesting property of split scope readings is that it is essential that we
be able to \textit{lower} a 3-story tower in two distinct steps --
\textit{internal lower} followed by \textit{lower}.

Despite being very elegant, I'm not sure whether this is a completely satisfactory account of split scope
readings -- it's been observed, for example, that split scope always seems to
involve narrow scope of an existential (see, e.g., \citealt{abelsMarti2010}). This
doesn't fall out from the analysis outlined here.\sidenote{An investigation of split scope from the perspective
  of continuations could be a good project for this class -- see, e.g.,
  \cite{bumford2017} for relevant discussion.}

\section{Scope islands and obligatory evaluation}

Quantifiers can't take scope arbitrarily high -- rather, their scope is roofed
by certain constituents. As is well known, the environments that are
\textit{islands for scope taking} don't necessarily correspond to the
environments that are islands for overt movement (see \citealt{may}).

We still need a theory of scope islands in order to restrict the power of
continuation semantics. It turns out that there is a very natural notion
available to us, similar to \citeauthor{chomsky1995}'s notion of a \textit{phase}.

Inspired by research on \textit{delimited control} in computer
science\sidenote{See, e.g., \cite{danvyFilinski1992} and \cite{wadler1994}.},
\citet{Charlowc} develops an interesting take on scope islands couched in terms
of continuations.

He proposes the following definition:

\ex
\textit{Scope islands} (def.)\\
A \textit{scope island} is a constituent that is subject to \textit{obligatory
  evaluation}.\\
\phantom{,}\hfill\citep[p. 90]{Charlowc}
\xe

By \textit{obligatory evaluation}, we mean that every continuation argument
\textit{must} be saturated before semantic computation can proceed. In other
words, a scope island is a constituent where, if we have something of type
$\type{K_{t} a}$, we must lower it before we can proceed.

Let's be more precise:

\pex
\a A constituent X is \textit{evaluated} if it has an evaluated type $\type{a}$.
\a A type $\type{a}$ is \textit{evaluated} if $\type{a} \neq \type{K_{t} b}$.
\xe

One way of thinking about this, is that the presence of an unsaturated
continuation argument means that there is some computation that is being
deferred until later.

Scope islands are constituents at which evaluation is
\textit{forced}. As noted by \citeauthor{Charlowc}, this idea bears an
intriguing similarity to \citeauthor{chomskyPhase}'s notion of a
\textit{phase}.\sidenote{Exploring this parallel in greater depth could make for
an interesting term paper topic.}

How does this work in practice? A great deal of ink has been spilled arguing
that, e.g., a finite clause is a scope island.

\ex
A boy said $\overbrace{\text{\fbox{that Susan greeted every
      linguist}}}^{\text{scope island}}$.\hfill$∃>∀; \text{\xmark} ∀ > ∃$
\xe

The derivation of the embedded clause proceeds as usual via lift and \ac{sfa}.

\begin{fullwidth}
\begin{multicols}{2}
\ex
Scope island with an unevaluated type\\
\begin{forest}
  [{...}
    [{...} [{a boy},roof]]
    [{...}
      [{said}]
      [{\xmark $\semtower{∀x[\ml{linguist} x → []]}{\ml{Susan greeted }x}$} [{Susan greeted every linguist},roof]]
    ]
  ]
\end{forest}
\xe
\columnbreak
\ex
Scope island with an evaluated type\\
\begin{forest}
  [{...}
    [{...} [{a boy},roof]]
    [{...}
      [{said}]
      [{\cmark $∀x[\ml{linguist} x→ \ml{Susan greeted }x]$}
      [{$↓$} [{$\semtower{∀x[\ml{linguist} x → []]}{\ml{Susan greeted }x}$} [{Susan greeted every linguist},roof]]]]
    ]
  ]
\end{forest}
\xe
\end{multicols}
\end{fullwidth}

This story leaves a lot of questions unanswered of course:

\begin{itemize}

    \item Is this just a recapitulation of a representational constraint on
    quantifier raising?\sidenote{The answer to this question may ultimate be
    \textit{yes}, in my view.}

    \item Are finite CPs the only scope island? What about DPs?\sidenote{In my
    view, yes, the default assumption should be that DPs are scope islands. See
    \cite{sauerland2005dp} and \cite{charlow2010} for relevant discussion.}

    \item Can we give a principled story about islands for overt movement using
    similar mechanisms? What explains the difference between overt movement and
    scope taking with respect to locality?\footnote{If we want to give a more
    general account of phases using this mechanism, we need to give an account
    of overt movement in terms of continuations, too. See my unpublished ms.
    \textit{Movement as higher-order structure building} for progress in this direction.}

\end{itemize}

\subsection{A brief remark on Scope Economy}

As famously discovered by \cite{fox1995}, scope-shifting operations are subject
to an economy condition.

\ex
Economy condition on scope shifting (\textit{Scope Economy}) (def.)\\
OP can apply only if it affects semantic interpretation (i.e., only if
inverse-scope and surface-scope are semantically distinct).\sidenote{Here, OP
  stands for \textit{scope-shifting operation}, i.e., quantifier raising.}\\
\phantom{,}\hfill(\citealt[p.,21]{fox2000})
\xe

Scope economy has an impressively wide empirical coverage, including
interactions between scope and ellipsis. Consider, e.g., the following contrasts
(from \citealt{fox1995}):

\ex
Some linguist likes every philosopher.\hfill $∃ > ∀; ∀ > ∃$
\xe

\ex~
Some linguist likes every philosopher,\\
and some mathematician does too.\hfill
$∃ > ∀; ∀ > ∃$
\xe

\ex~
Some linguist likes every philosopher,\\
and Mary does too.\hfill
$∃ > ∀; $\xmark $∀ > ∃$
\xe

\ex~
Some linguist likes every philosopher,\\
and every mathematician does too.\hfill
$∃ > ∀; $\xmark $∀ > ∃$\label{ex:mary}
\xe

Scope economy gives us an explanation of this paradigm, just in case we assume
that an elliptical sentence and its antecedent must involve \textit{parallel
  scopal relations}.

In (\ref{ex:mary}), for example, QR of \textit{every philosopher} over \textit{every
  mathematician} is blocked by scope economy, since the relative scope of two
universal quantifiers doesn't affect truth-conditions.

\ex
\ljudge{*}every philosopher [$λ x$ every mathematician does \fbox{like $t_{x}$}]
\xe

Since the antecedent must involve a parallel scopal relation, inverse scope is blocked.

Here, I just want to point out that our continuation semantic framework is
actually an extremely natural fit for scope economy:

\ex
Economy condition on scope shifting (continuations ver.)\\
A derivation $D$ is ruled out if there is a simpler derivation $D'$ that gives rise to the same interpretation.
\xe

Just so long as the economy condition is evaluated at scope islands (as
argued for by \citealt{fox1995}), we capture the basic observations, since scope
\textit{every philosopher} over \textit{every mathematician}, involves,
minimally, an additional internal lift of \textit{every philosopher}.

\begin{fullwidth}
\begin{multicols}{2}

  \ex
  Surface scope (schematic derivation)\\
  \begin{forest}
    [{$↓$}
    [{$\ml{S}$}
      [{$Q_{1}$}]
      [{$\ml{S}$}
        [{$R^{↑}$}]
        [{$Q_{2}$}]
      ]
    ]]
  \end{forest}
  \xe

  \columnbreak

  \ex
  Inverse scope (schematic derivation)\\
  \begin{forest}
    [{$↓$}
    [{$⇊$}
  [{$\ml{S}_{2}$}
    [{$Q_{1}^{↑}$}]
    [{$\ml{S}_{2}$}
      [{$R^{↑_{2}}$}]
      [{$Q_{2}^{⇈}$}]
    ]
  ]]]
  \end{forest}
  \xe

\end{multicols}
\end{fullwidth}


It remains to be seen whether an economy condition framed in terms of
the complexity of a continuation-semantic derivation makes
any distinct predictions to an economy condition framed in terms of
QR.\sidenote{Figuring this out would be a great student project, I think.}

\section{Generalized con/dis-junction}

The flexibility of \textit{and} and \textit{or} has been discussed at length by,
e.g., \citet{parteeRooth}, \citet{winter_flexibility_2001}, among
others.\sidenote{We saw an example of this when we motivated \ml{LIFT}.}

\ex
Lan and some woman arrived.
\xe

\ex~
Howie sneezed and/or coughed.
\xe

\ex~
Lan kissed and/or hugged Irene.
\xe

Unlike other expressions we've seen so far, we can characterize \textit{and} and
\textit{or} as expressions that takes two continuized values as arguments.\sidenote{Note that
  since the continuation argument $k$ occurs more than once in the function
  body, we can no longer abbreviate the flat lambda-expression using a tower.}

\pex
\a \(m \ml{and} n ≔ λ k . m k ∧ n k\)\hfill$\ml{and} : \type{K_{t} a → K_{t} a → K_{t} a}$
\a \(m \ml{or} n ≔ λ k . m k ∨ n k\)\hfill$\ml{or} : \type{K_{t} a → K_{t} a → K_{t} a}$
\xe


The intuition here is as follows: \textit{and} wants as its arguments things
that are \textit{guaranteed} to give back truth values at some future stage of computation.

This accounts for the basic cases discussed by \citet{parteeRooth}, with a
single (polymorphic) entry for \textit{and}, as illustrated below:\sidenote{
Unlike other lexical entries we've seen so far, \textit{and} is
\textit{lexically specified} as seeking continuized arguments. As such, if a
value isn't already typed as an instantiation of \type{K_t a}, it must be
lifted.

In the derivations below, you can observe that, unlike the cases we've
encountered so far, \textit{and} composes with its arguments via \ac{fa} rather
than \ac{sfa}.
}

\ex Lan and some woman arrived.\\
\begin{forest}
  [{$\ml{Lan} \ml{arrived} ∧ ∃x[\ml{woman} x ∧ \ml{arrived} x]$\\$\ml{A}$}
  [{$λ k . k \ml{Lan}∧ ∃x[\ml{woman} x ∧ k x]$\\\ml{A}}
    [{$λ k . k \ml{Lan}$\\Lan$^{↑}$}]
    [{$λ nk . n k ∧ ∃x[\ml{woman} x ∧ k x]$\\$\ml{A}$}
      [{$λ mnk . n k ∧ m k$\\and}]
      [{$λ k .∃x[\ml{woman} x ∧ k x]$\\some woman}]
    ]
  ]
    [{arrived}]
  ]
\end{forest}
\xe

\ex~ Howie sneezed and coughed.\\
\begin{forest}
  [{$\ml{Howie sneezed and coughed}$\\$\ml{A}$}
    [{$λ k . k \ml{Howie}$\\Howie$^{↑}$}]
    [{$λ k . k \ml{sneezed} ∧ k \ml{coughed}$\\\ml{A}}
      [{$λ k . k (\ml{sneezed})$\\sneezed$^{↑}$}]
      [{\ml{A}}
        [{$λ mnk . n k ∧ m k$\\and}]
        [{$λ k . k (\ml{coughed})$\\coughed$^{↑}$}]
      ]
    ]
  ]
  \end{forest}
\xe

\subsection{The scope of con/dis-junction}

In a lot of the cases we've seen so far, teasing apart the power of QR from the
power of continuations hasn't been at all trivial. In this section, we'll see an
example of a case where it's clearer that continuations can help us in ways that
QR can't.

Both conjunction and disjunction exhibit \enquote{scope} ambiguities. This is
illustrated below for conjunction:

\pex
You're not allowed to dance and sing.
\a \textit{You're not allowed to dance and you're not allowed to sing}\\
\phantom{,}\hfill$∧ > ¬ > ◇$
\a \textit{You're not allowed to both dance and sing (at the same time)}\\
\phantom{,}\hfill$¬ > ◇ > ∧$
\xe

We can account for the wide/narrow scope ambiguity as a matter of where we
\ml{LOWER}.

Let's first compute the semantic value of the prejacent of the modal:

\ex
\begin{forest}
  [{$λ k . k (\ml{you dance}) ∧ k (\ml{you sing})$\\$\ml{S}$}
    [{$λ k . k \ml{you}$\\you$^{↑}$}]
    [{$λ k . k \ml{dance} ∧ k \ml{sing}$\\$\ml{A}$}
       [{$λ k . k \ml{dance}$\\dance$^↑$}]
       [{$\ml{A}$}
         [{$λ mnk . n k ∧ m k$\\and}]
         [{$λ k . k \ml{sing}$} [{sing$^{↑}$},roof]]
       ]
    ]
  ]
\end{forest}
\xe

If we \ml{LOWER} immediately we're just going to get a proposition, which
\ml{allowed} will take as its argument, deriving the narrow scope
reading.\sidenote{Instead of composing via \ml{S} and lowering, we could equivalently compose lifted
  \textit{you} with its complement via \ml{A}, since it's a subject.}

The \enquote{wide scope} reading is more interesting. We can simply defer
lowering, and compose the prejacent with lifted \textit{allowed} via \ml{S}.

\ex
\begin{forest}
  [{$(¬ (◇ (\ml{you sing}))) ∧ (¬ (◇ (\ml{you dance})))$}
  [{\ml{LOWER}}
  [{$λ k . k (¬ (◇ (\ml{you sing}))) ∧ k (¬ (◇ (\ml{you dance})))$\\$\ml{S}$}
    [{$λ k . k (λ p . ¬ p)$\\not$^{↑}$}]
    [{$\ml{S}$}
      [{$λ k . k (λ p . ◇ p)$\\allowed$^{↑}$}]
      [{$λ k . k (\ml{you dance}) ∧ k (\ml{you sing})$} [{you dance and sing},roof]]
    ]
  ]]]
\end{forest}
\xe

We make the nice prediction that \enquote{wide scope} readings of conjunction
should be subject to scope islands:

\ex
John isn't allowed to claim [that you sing and dance].\hfill \xmark $∧ > ¬ > ◇$
\xe

\subsection{Split scope with conjunction}

\citet{hirsch2017} claims that conjunction reduction is necessary in order to
derive the \enquote{split scope} reading of the following sentence:

\ex
John refused to visit any city in Europe and any city in
Asia.\label{ex:hirsch1}\\
\textit{John refused to visit any city in Europe, and he refused to visit any
  city in Asia.}\hfill$∧ > \ml{refuse} > \ml{any}$
\xe


\citeauthor{hirsch2017}'s analysis:

\ex
John \bracketStr[vP]{refused to visit any city in Europe} and\\
\bracketStr[vP]{refused to visit any city in Asia}
\xe

Here, I'm going to argue that we can get the split scope reading of
(\ref{ex:hirsch1}) using just the machinery we've already introduced.
Concretely:
\begin{itemize}
    \item Our entry for \textit{and}.
    \item The free availability of \textit{external lift}.
\end{itemize}

First of all, each conjunct is going to be \textit{externally lifted}.

\ex
$\eval{any city in Europe}^{↑} = λ l . l (λ k . ∃x[\ml{city-europe} x ∧ k x])$
\xe

\textit{and}, in our view, is polymorphic -- it's looking for two arguments of
type $\type{K_{t} a}$. This means it can compose the externally lifted DPs,
returning the following meaning:

\ex
$λ l . l \left(\semtower{∃x[\ml{europe-city} x ∧ []]}{x}\right) ∧ l \left(\semtower{∃x[\ml{asia-city} x ∧ []]}{x}\right):\type{K_{t} (K_{t} a)}$
\xe

Remember that $\ml{S}_{2}$ is just like $\ml{S}}$, only it does \ml{S} of
the wrapped values. Composition can proceed via $↑_{2}$ and $\ml{S}_{2}$ up to
the prejacent of \textit{refuse}.

To get the scope of the quantifiers right, we do \textit{internal lower} just
before we compose with \textit{refuse}.

\ex
\begin{forest}
  [{$λ l . l (\ml{refuse }∃x[\ml{europe-city} x ∧ \ml{pro visit }x]) ∧ l (\ml{refuse} ∃x[\ml{europe-city} x ∧ \ml{pro visit }x])$\\\ml{S}}
    [{refuse$^{↑}$}]
    [{$λ l . l (∃x[\ml{europe-city} x ∧ \ml{pro visit }x]) ∧ l (∃x[\ml{europe-city} x ∧ \ml{pro visit }x])$}
    [{$⇊$}
    [{$λ l . l \left(\semtower{∃x[\ml{europe-city} x ∧ []]}{\ml{pro visit} x}\right) ∧ l \left(\semtower{∃x[\ml{asia-city} x ∧ []]}{\ml{pro visit }x}\right)$} [{PRO visit any city in Europe\\and any city in Asia},roof]]
  ]]]
\end{forest}
\xe

Lowering the result gives us...the split-scope reading:

\ex
\begin{forest}
  [{$\ml{refuse }∃x[\ml{europe-city} x ∧ \ml{pro visit }x ∧ \ml{refuse }∃x[\ml{asia-city} x ∧ \ml{pro visit }x]$}
  [{$\ml{LOWER}$}
    [{\fbox{$λ l . l (\ml{refuse }∃x[\ml{europe-city} x ∧ \ml{pro visit }x]) ∧ l (\ml{refuse} ∃x[\ml{europe-city} x ∧ \ml{pro visit }x])$}}]
  ]
  ]
\end{forest}
\xe

So, continuations can get split-scope readings of conjunction straightforwardly.
This is a notable result.\sidenote{To my knowledge, this is a novel observation.

\citeauthor{hirsch2017} doesn't consider an analysis of this data in terms of
continuations, but does entertain a similar analysis involving lifting the
quantificational conjuncts, alongside QR with higher-type traces.
\citet{hirsch2017} rejects this analysis on the basis that it allows for the
quantifiers to be syntactically above the intensional verb, while semantically
reconstructing below it -- a configuration which has been argued to be ruled
out. Note that an analysis in terms of continuations doesn't face this objection
-- at no point in the analysis did we need to invoke covert movement.
}

Note that this theory of split-scope coordination explains the lack of
split-scope reading in the following example, from \citet{parteeRooth}.

\ex
John hopes [that some company will hire a maid and a cook].\\
\xmark \textit{John hopes that some company will hire a maid,\\
and John hopes that
some company will hire a cook.}
\xe

This is simply because the embedded finite clause is a scope island, and
therefore the scope of \textit{and} is trapped.

\section{Extension i: DP internal composition and indexed continuations}

As you'll probably have noticed, we've spent this whole time treating
quantificational DPs such as \textit{every boy} as primitives.

At this point a natural question to ask is: how do determiners compose with
their restrictors?

Surprisingly, the answer isn't as straightforward as you might think.

Naively, we may assume that determiners receive they're standard meaning --
essentially, a function from a predicate to a \textit{continuized} individual.

\ex
$\eval{every} \stackrel{?}{\coloneq} λ P . \semtower{∀y[P y → []]}{y}$
\xe

This will (obviously) work fine for a nominal restrictor.

But, what happens if the restrictor itself contains a quantificational
expression? Consider the following example:

\ex
Every boy with a book left.\hfill $∀ > ∃$
\xe

Let's first compute the meaning of the restrictor \textit{boy with a
  book}:\sidenote{Here, I'm using $\ml{S}_{\wedge}$ as an abbreviation for
  \textit{scopal predicate modification}. In fact, we can lift any binary
  operation into its scopal counterpart.}

\ex
\begin{forest}
  [{$\semtower{∃x[\ml{book} x ∧ []]}{λy . \ml{boy} y ∧ y \ml{with} x}$\\$\ml{S}_{∧}$}
    [{$\semtower{[]}{λ y . \ml{boy} y}$\\boy$^{↑}$}]
    [{$\semtower{∃x[\ml{book} x ∧ []]}{λ y . y \ml{with} x}$\\$\ml{S}$}
      [{$\semtower{[]}{λ xy . y \ml{with} x}$\\with$^{↑}$}]
      [{$\semtower{∃x[\ml{book} x ∧ []]}{x}$} [{a book},roof]]
    ]
  ]
\end{forest}
\xe

What we end of with is a continuized predicate of type
$\type{\semtower{t}{e → t}}$.

How do we compose this with our
determiner of type $\type{(e → t) → \semtower{t}{e}}$?

One possiblity is to simply lift the determiner. There are two problems with
this approach.

\begin{itemize}

  \item This will give \textit{a book} scope over \textit{every} -- here, we're
    interested in the surface scope reading.

  \item Despite potentially being a useful strategy for deriving
    inversely-linked interpretations, this will ultimately allow the inner
    quantifier to take scope outside of the containing DP -- this runs into
    issues with Larson's generalization. We'll come back to
    this.\sidenote{Essentially, there is some quite strong evidence that DP is a
    scope island.}

\end{itemize}

Instead, we're going to pursue the idea that the determiner itself \textit{takes
scope}.\sidenote{This has many precedents in the literature. See, for example, \citet{heim1982}.}

Consider our type constructor $\type{K_{t}}$ -- it takes a type $\type{a}$ and
returns a new type $\type{(a → t) → t}$.

In principle, we could parameterize
$\type{K}$ to any type $\type{r}$ (here $\type{r = t}$). Let's call $\type{r}$ the
\textit{return type}, since it tells us the type of the value we get when the
continuation argument $k$ applies to its argument.

What if applying $k$ to its argument gives back an intermediate result of type
$\type{i}$, which is subsequently transformed into a final result of type
$\type{r}$? We can model this idea using the type constructor
$\type{K_{r}^{i}}$:\sidenote{This ultimately goes back to \cite{wadler1994}.}

\ex
$\type{K_{r}^{i} a ≔ (a → i) → r}$
\xe

\citet{barkerShan2015} generalize tower notation to the more general
type-schema.\sidenote{See also \citet[chapter 3]{Charlowc}.}

\ex Tripartite tower types (def.)\\
$\tower{\type{r}}{\type{i}}{\type{a}} ≔ \type{(a → i) → r}$
\xe

We can think of our existing tower notation as an abbreviation for a tripartite
tower type, where the intermediate and final result types happen to be the same:

\ex Bipartite towers as abbreviations for tripartite towers\\
$\type{\semtower{r}{a} ≔ \tower{r}{r}{a}}$
\xe

Now that we have tripartite tower types, we can think of the restrictor argument
$c$ of \textit{every} as a \textit{continuation argument}.

\ex Standard determiner semantics for \textit{every}\\
$\eval{every} ≔ λ c . \left[λ P . \semtower{∀y[P y → []]}{y}\right] (λ x . c x)$\hfill$\type{(e → t) → \semtower{t}{e}}$
\xe

We can abbreviate the meaning of \textit{every} as a tower, where $c$ is the
continuation argument:

\ex
$\semtower{\left[λ P . \semtower{∀y[P y → []]}{y}\right] (λ x . [])}{x}$\hfill$: \tower{\semtower{\type{t}}{\type{e}}}{\type{t}}{\type{e}}$
\xe


Our existing definition of $\ml{S}$ can be made more type-general, in order to
accommodate tripartite tower types. \textit{Adjacent types} match and cancel out:

\ex
$\ml{S} : \type{\tower{r}{i}{a → b} → \tower{i}{j}{a} → \tower{r}{j}{b}}$
\xe

The actual definition of $\ml{S}$ doesn't change.

\ex \textit{scopal function application} (def.)\\
$m \ml{S} n ≔ λ k . m (λ x . n (λ y . k (x \ml{A} y)))$
\xe

Likewise, the type of \textit{lower} is further generalized; the definition
doesn't change:

\ex
$(↓) : \type{\tower{a}{b}{b} → a}$
\xe

We can now make sense of the idea that the determiner \textit{takes scope}.
Let's consider again the following complex DP:

\ex
Every boy with a book.
\xe

First, we compute the meaning of the restrictor, returning something of type
$\type{\tower{t}{t}{e → t}}$:


\ex Step 1: compute \textit{boy with a book}\\
\begin{forest}
  [{$\semtower{∃x[\ml{book} x ∧ []]}{λy . \ml{boy} y ∧ y \ml{with} x}$}
    [{boy}]
    [{...}
      [{$λ xy . y \ml{with} x$\\with}]
      [{$\semtower{∃x[\ml{book} x ∧ []]}{x}$} [{a book},roof]]
    ]
  ]
\end{forest}
\xe

Recall that \textit{every} is of type
$\tower{\semtower{\type{t}}{\type{e}}}{\type{t}}{\type{e}}$, this is something
that can compose with the restrictor via our more type-general formulation of \ac{sfa}.

\ex
\begin{forest}
  [{$\semtower{∀y[(∃x[\ml{book} x ∧ \ml{boy} y ∧ y \ml{with} x]) → []]}{y}$}
  [{\fbox{$\left[λ P . \semtower{∀y[P y → []]}{y}\right] (λ y . ∃x[\ml{book} x ∧ \ml{boy} y ∧ y \ml{with} x])$}}
  [{$↓$}
  [{$\semtower{\left[λ P . \semtower{∀y[P y → []]}{y}\right] (λ y . ∃x[\ml{book} x ∧ []])}{\ml{boy} y ∧ y \ml{with} x}$\\$\ml{S}$}
    [{$\semtower{\left[λ P . \semtower{∀y[P y → []]}{y}\right] (λ y . [])}{y}}$]
    [{$\semtower{∃x[\ml{book} x ∧ []]}{λy . \ml{boy} y ∧ y \ml{with} x}$} [{boy with a book},roof]]
  ]]]]
\end{forest}
\xe

\begin{tcolorbox}
Exercise
\tcblower
As homework, you can attempt to work out the above derivation yourself. Convince
yourself that the types work out.
\end{tcolorbox}


\section{Extension ii: Exceptional scope of indefinites}

Unlike other quantificational expressions, indefinites don't obey scope islands:

\ex
Roger hopes that some student will be absent.\hfill $∃ > \ml{hope}$
\xe

\citet{Charlowc} outlines a theory of exceptional scope framed in terms of
continuation semantics, compatible with the assumption that scope islands are
obligatorily evaluated.

\citeauthor{Charlowc} assumes that, unlike other
quantificational expressions, indefinites denote sets of individuals:

\pex
\a $\eval{some student} ≔ \set{x|\ml{student} x}$
\a $\eval{some student}: \type{\set{e}}$
\xe

The central component of \citeauthor{Charlowc}'s proposal is a method (let's
call it $⋆$) for lifting a set into a scope taker.\sidenote{If you're familiar
  with the notion of \textit{monads}, you'll noticed that $⋆$ is just the
  \textit{bind} operation of the set monad.

  One of \citeauthor{Charlowc}'s insights is that, for any monad $M$, the bind
  of $M$ provides a way of lifting an inhabitant of $M a$ into a scope taker.
}

\ex
$m^{⋆} ≔ λ k . \bigcup\limits_{x ∈ m} k x$\hfill$⋆: \type{\set{e} → (e → \set{t}) → \set{t}}$
\xe

In fact, we can write $⋆$ using tower notation:

\ex
$m^{⋆} ≔ \semtower{\bigcup\limits_{x ∈ m} []}{x}$
\xe

As you'll notice, $⋆$ gives back a \textit{continuized individual}, but with the
return type generalized to $\type{\set{{t}}}$ rather than $\type{t}$.

Our existing definitions of lift and \ac{sfa} allow us to compose a $⋆$-shifted
indefinite straightforwardly, once the return type is set to $\type{\set{t}}$.
The actual definitions of the operations don't change.

Here I'll show how $⋆$, once combined with our existing mechanisms for doing
scope-taking -- namely lift and \ml{S} -- \textit{predicts} that indefinites can
exceptionally scope out of scope islands.

I'll demonstrate this for our example: \textit{Roger hopes that some student is absent}:

\ex Step 1: lift \textit{some student} into a scope-taker\\
\begin{forest}
  [{$λ k . \bigcup\limits_{x ∈ \ml{student}} k x$}
  [{$⋆$}
  [{$\set{x|\ml{student} x}$\\some student}]
  ]
  ]
\end{forest}
\xe

\ex~ Step 2: scope via lift and \ml{S}\\
\begin{forest}
  [{$λ k . \bigcup\limits_{x ∈ \ml{student}} k (\ml{absent} x)$}
    [{$λ k . \bigcup\limits_{x ∈ \ml{student}} k x$} [{some student$^{⋆}$},roof]]
    [{$λ k . k (λ x . \ml{absent} x)$\\absent$^{↑}$}]
  ]
\end{forest}
\xe

We've reached a scope island now, so we need to \textit{evaluate} it. We can't use our
ordinary lower function -- since we're dealing with sets, we lower by feeding n Partee's
$\ml{IDENT}$, rather than the identity function:

\ex
$m^{⇓} ≔ m (λ x . \set{x})$\hfill $⇓: \type{((a → \set{a}) → \set{a}) → \set{a}}$
\xe

\ex Step 3: evaluate the scope island:\\
\begin{forest}
[{$\set{\ml{absent} x:\ml{student} x}$}
[{$⇓$}
[{$λ k . \bigcup\limits_{x ∈ \ml{student}} k (\ml{absent} x)$} [{some student is absent},roof]]
]
]
\end{forest}
\xe

What we end up with is a set of propositions of the form \textit{$x$ is absent},
where $x$ is a student.

Now we re-lift the scope island into a scope taker via $⋆$:

\ex
Step 4: re-lift via $⋆$\\
\begin{forest}
  [{$λ k . \bigcup_{p ∈ \set{\ml{absent} x:\ml{student} x}} k p$}
  [{$⋆$}
    [{$\set{\ml{absent} x:\ml{student} x}$}]
  ]
  ]
\end{forest}
\xe

Now we scope out the scope-island via lift and \ml{S} -- the indeterminacy
introduced by the indefinite survives, and we get exceptional scope.

\ex
Step 5: scope via lift and $⋆$:\\
\begin{forest}
  [{$\set{\ml{roger hope} p:p∈\set{\ml{absent} x:\ml{student} x}}$}
  [{$⇓$}
  [{$λ k . \bigcup_{p ∈ \set{\ml{absent} x:\ml{student} x}} k (\ml{roger hope }p)$\\$\ml{S}$}
    [{Roger$^{↑}$}]
    [{$\ml{S}$}
      [{hope$^{↑}$}]
      [{$λ k . \bigcup_{p ∈ \set{\ml{absent} x:\ml{student} x}} k p$}]
    ]
  ]]]
\end{forest}
\xe

The result of lowering is equivalent to the following:

\ex
$\set{\ml{roger hope (\ml{absent} x) : \ml{student} x}}$
\xe

In brief, unlike other quantifiers, indefinites introduce indeterminacy (which
we model as a set). Indeterminacy survives obligatory evaluation at a scope island.

Here is the derivation again from a bird's eye view:

\ex
\begin{forest}
  [{$\set{t}$}
  [{$⇓$}
  [{$\type{K_{\set{t}} (t)}$\\\ml{S}}
    [{$\type{K_{\set{t}} (e)}$\\Roger$^{↑}$}]
    [{$\type{K_{\set{t}} (e → t)}$\\\ml{S}}
      [{$\type{K_{\set{t}} (t → e → t)}$\\hope$^{↑}$}]
      [{$\type{K_{\set{t}} t}$}
      [{$⋆$}
      [{$\type{\set{t}}$\\$⇓$}
      [{$\type{K_{\set{t}} e}$\\$\ml{S}$}
        [{$\type{K_{\set{t}} e}$} [{$⋆$} [{$\set{\type{e}}$\\some student}]]]
        [{$\type{K_{\set{t}} (e → t)}$\\is absent$^↑$}]
      ]
      ]
      ]
      ]
    ]
  ]
  ]
  ]
\end{forest}
\xe


% \subsection{Inverse linking}

% \ex
% $\eval{every} = λ c_{e→ t} . λ s_{e → t} . \ml{every} (λ x . c x) (λ y . s y)$
% \xe

% \ex
% $\eval{every'} = λ d_{e→ (e→ t) → t} . λ s_{e → t} . \ml{every} (λ y . (d y s)^{↓}) (λ x . s x)$
% \xe

% \ex
% $λ s . \ml{every} (λ y . ∃x[\ml{book} x ∧ s (\ml{boy} y ∧ y \ml{with} x)]) (λ x . s x)$
% \xe

% \ex
% $\$
% \xe

\printbibliography


\end{document}
