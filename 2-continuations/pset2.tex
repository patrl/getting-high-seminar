\documentclass[nols,twoside,nofonts,nobib,nohyper]{tufte-handout}

\usepackage{fixltx2e}
\usepackage{tikz-cd}
\usepackage{tcolorbox}
\usepackage{appendix}
\usepackage{listings}
\lstset{language=TeX,
       frame=single,
       basicstyle=\ttfamily,
       captionpos=b,
       tabsize=4,
  }

\begin{acronym}
\acro{sfa}{Scopal Function Application}
\acro{fa}{Function Application}
\acro{wco}{Weak Crossover}
\acro{ScoT}{Scope Transparency}
\acro{vfs}{Variable Free Semantics}
\acro{acd}{Antecedent Contained Deletion}
\acro{qr}{Quantifier Raising}
\acro{doc}{Double Object Construction}
\acro{ccp}{Context Change Potential}
\acro{dmg}{Dynamic Montague Grammar}
\acro{dr}{Discourse Referent}
\acro{qp}{Quantificational Phrase}
\acro{dp}{Determiner Phrase}
\acro{dpp}{Dynamic Predication Principle}
\acro{ds}{Dynamic Semantics}
\acro{gq}{Generalized Quantifier}
\acro{npi}{Negative Polarity Item}
\acro{lf}{Logical Form}
\acro{pm}{Predicate Modification}
\acro{pfa}{Pointwise Function Application}
\acro{cg}{Common Ground}
\end{acronym}

\renewcommand*{\acsfont}[1]{\textsc{#1}}


\makeatletter
% Paragraph indentation and separation for normal text
\renewcommand{\@tufte@reset@par}{%
  \setlength{\RaggedRightParindent}{0pt}%
  \setlength{\JustifyingParindent}{0pt}%
  \setlength{\parindent}{0pt}%
  \setlength{\parskip}{\baselineskip}%
}
\@tufte@reset@par

% Paragraph indentation and separation for marginal text
\renewcommand{\@tufte@margin@par}{%
  \setlength{\RaggedRightParindent}{0pt}%
  \setlength{\JustifyingParindent}{0pt}%
  \setlength{\parindent}{0pt}%
  \setlength{\parskip}{\baselineskip}%
}
\makeatother

\setcounter{secnumdepth}{3}

\title{p-set 2}

\date{\today}

\author[Patrick D. Elliott and Martin Hackl]{Patrick~D. Elliott\sidenote{\texttt{pdell@mit.edu}} \& Martin Hackl\sidenote{\texttt{hackl@mit.edu}}}

\addbibresource[location=remote]{/home/patrl/repos/bibliography/elliott_mybib.bib}

\lingset{
  belowexskip=0pt,
  aboveglftskip=0pt,
  belowglpreambleskip=0pt,
  belowpreambleskip=0pt,
  interpartskip=0pt,
  extraglskip=0pt,
  Everyex={\parskip=0pt}
}


% \usepackage{booktabs} % book-quality tables
% \usepackage{units}    % non-stacked fractions and better unit spacing
% \usepackage{lipsum}   % filler text
% \usepackage{fancyvrb} % extended verbatim environments
%   \fvset{fontsize=\normalsize}% default font size for fancy-verbatim environments

% % Standardize command font styles and environments
% \newcommand{\doccmd}[1]{\texttt{\textbackslash#1}}% command name -- adds backslash automatically
% \newcommand{\docopt}[1]{\ensuremath{\langle}\textrm{\textit{#1}}\ensuremath{\rangle}}% optional command argument
% \newcommand{\docarg}[1]{\textrm{\textit{#1}}}% (required) command argument
% \newcommand{\docenv}[1]{\textsf{#1}}% environment name
% \newcommand{\docpkg}[1]{\texttt{#1}}% package name
% \newcommand{\doccls}[1]{\texttt{#1}}% document class name
% \newcommand{\docclsopt}[1]{\texttt{#1}}% document class option name
% \newenvironment{docspec}{\begin{quote}\noindent}{\end{quote}}% command specification environment

\begin{document}

\maketitle% this prints the handout title, author, and date

\textbf{Deadline: }02.20 (i.e., before next class)

\section{Warming up}

Compute the indicated reading of the following sentence:

\ex
Exactly three students gave at least one book to every
girl.\\
\phantom{,}\hfill$∀ > \ml{atLeastOne} > \ml{exactlyThree}$
\xe

Assume (roughly) the following syntax:

\ex
\begin{forest}
  [{...}
    [{...} [{exactly three students},roof]]
    [{...}
    [{}
      [{give}]
      [{...} [{at least one book},roof]]
    ]
      [{...}
        [{to}]
        [{...} [{every student},roof]]
      ]
    ]
  ]
\end{forest}
\xe

Note:

\begin{itemize}

    \item You'll probably want to assume that \textit{to} is semantically
    vacuous, and treat \textit{give} as a type \type{e → e → e → t} function.

    \item Feel free to eschew lambda expressions in favour of tower notation,
    unless you feel like you need extra practice.

    \item In class, we mostly talked about \textit{every} and \textit{some},
    which we can easily represent as first order quantifiers. Here's a general
    recipe for modeling any quantificational DP as a continuized individual,
    taking advantage of a simple equivalence:

    \ex
    $Q ≣ λ k . Q (λ x . k x)$\hfill $\type{(e → t) → t}$
    \xe

    In tower terms:

    \ex
    $\semtower{Q (λ x . [])}{x}$
    \xe

\end{itemize}

\section{More on con/dis-junction}

\subsection{Conjunction practice}

The following sentence has a \enquote{split scope} reading (in my view) that can
be paraphrased as: \textit{John refused to stay in any hotel, and John refused
  to rent any car}.

\ex
John refused to stay in any hotel or rent any car.\hfill $∧ > \ml{refuse} > \ml{any}$
\xe

Assume that a WYSIWYG syntactic structure. Show how we can derive the split scope reading using the machinery introduced in class.


\subsection{Scope of disjunction}

In class, we claimed that our account predicts that the scope of conjunction is
sensitive to scope islands, on the basis of examples like the following:

\ex
John hopes [that some company will hire a main and a cook].\\
\xmark \textit{John hopes that some company will hire a maid, and John hopes
  that some company will hire a cook.}
\xe

Is the scope of \textit{dis}junction sensitive to scope islands? We predict that
it is; can you come up with counterexamples?

\section{DP-internal composition}

\begin{tcolorbox}
N.b. this question is based heavily on section 6 of the handout from session 2.
Consult the handout if you run into trouble.
\end{tcolorbox}

\subsection{Part i}

As you'll probably have noticed, we've spent this whole time treating
quantificational DPs such as \textit{every boy} as primitives.

At this point a natural question to ask is: how do determiners compose with
their restrictors?

Surprisingly, the answer isn't as straightforward as you might think.

Naively, we may assume that determiners receive they're standard meaning --
essentially, a function from a predicate to a \textit{continuized} individual.

\ex
$\eval{every} \stackrel{?}{\coloneq} λ P . \semtower{∀y[P y → []]}{y}$
\xe

But, what happens if the restrictor itself contains a quantificational
expression? Consider the following example:

\ex
Every boy with a book left.\hfill $∀ > ∃$
\xe

Try to compute the meaning of the subject, and explain what goes wrong and why.

\subsection{Part ii}

\citet{barkerShan2015} generalize tower notation to the more general
type-schema.\sidenote{See also \citet[chapter 3]{Charlowc}.}

\ex Tripartite tower types (def.)\\
$\tower{\type{r}}{\type{i}}{\type{a}} ≔ \type{(a → i) → r}$
\xe

We can think of our existing tower notation as an abbreviation for a tripartite
tower type, where the intermediate and final result types happen to be the same:

\ex Bipartite towers as abbreviations for tripartite towers\\
$\type{\semtower{r}{a} ≔ \tower{r}{r}{a}}$
\xe

Now that we have tripartite tower types, we can think of the restrictor argument
$c$ of \textit{every} as a \textit{continuation argument}.

\ex Standard determiner semantics for \textit{every}\\
$\eval{every} ≔ λ c . \left[λ P . \semtower{∀y[P y → []]}{y}\right] (λ x . c x)$\hfill$\type{(e → t) → \semtower{t}{e}}$
\xe

We can abbreviate the meaning of \textit{every} as a tower, where $c$ is the
continuation argument:

\ex
$\semtower{\left[λ P . \semtower{∀y[P y → []]}{y}\right] (λ x . [])}{x}$\hfill$: \tower{\semtower{\type{t}}{\type{e}}}{\type{t}}{\type{e}}$
\xe


Our existing definition of $\ml{S}$ can be made more type-general, in order to
accommodate tripartite tower types. \textit{Adjacent types} match and cancel out:

\ex
$\ml{S} : \type{\tower{r}{i}{a → b} → \tower{i}{j}{a} → \tower{r}{j}{b}}$
\xe

The actual definition of $\ml{S}$ doesn't change.

\ex \textit{scopal function application} (def.)\\
$m \ml{S} n ≔ λ k . m (λ x . n (λ y . k (x \ml{A} y)))$
\xe

Likewise, the type of \textit{lower} is further generalized; the definition
doesn't change:

\ex
$(↓) : \type{\tower{a}{b}{b} → a}$
\xe

Show how generalizing our existing machinery allows us to derive the surface
scope reading of our original sentence:

\ex
Every boy with a book left.
\xe

\end{document}
