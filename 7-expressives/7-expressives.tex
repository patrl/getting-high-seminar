\documentclass[nols,twoside,nofonts,nobib,nohyper]{tufte-handout}

\usepackage{fixltx2e}
\usepackage{tikz-cd}
\usepackage{tcolorbox}
\usepackage{appendix}
\usepackage{listings}
\lstset{language=TeX,
       frame=single,
       basicstyle=\ttfamily,
       captionpos=b,
       tabsize=4,
  }

\begin{acronym}
\acro{sfa}{Scopal Function Application}
\acro{fa}{Function Application}
\acro{wco}{Weak Crossover}
\acro{ScoT}{Scope Transparency}
\acro{vfs}{Variable Free Semantics}
\acro{acd}{Antecedent Contained Deletion}
\acro{qr}{Quantifier Raising}
\acro{doc}{Double Object Construction}
\acro{ccp}{Context Change Potential}
\acro{dmg}{Dynamic Montague Grammar}
\acro{dr}{Discourse Referent}
\acro{qp}{Quantificational Phrase}
\acro{dp}{Determiner Phrase}
\acro{dpp}{Dynamic Predication Principle}
\acro{ds}{Dynamic Semantics}
\acro{gq}{Generalized Quantifier}
\acro{npi}{Negative Polarity Item}
\acro{lf}{Logical Form}
\acro{pm}{Predicate Modification}
\acro{pfa}{Pointwise Function Application}
\acro{cg}{Common Ground}
\end{acronym}

\renewcommand*{\acsfont}[1]{\textsc{#1}}

\usepackage[font=footnotesize]{caption}
\usepackage{soul}

\makeatletter
% Paragraph indentation and separation for normal text
\renewcommand{\@tufte@reset@par}{%
  \setlength{\RaggedRightParindent}{0pt}%
  \setlength{\JustifyingParindent}{0pt}%
  \setlength{\parindent}{0pt}%
  \setlength{\parskip}{\baselineskip}%
}
\@tufte@reset@par

% Paragraph indentation and separation for marginal text
\renewcommand{\@tufte@margin@par}{%
  \setlength{\RaggedRightParindent}{0pt}%
  \setlength{\JustifyingParindent}{0pt}%
  \setlength{\parindent}{0pt}%
  \setlength{\parskip}{\baselineskip}%
}
\makeatother

\NewDocumentCommand\apl{}{\ensuremath{\odot}}
\NewDocumentCommand\aplp{}{\ensuremath{\circledast}}
\NewDocumentCommand\pure{m}{\ensuremath{{#1}^{ρ}}}
\NewDocumentCommand\intlift{m}{\ensuremath{{#1}^{⇈_{\aplp}}}}
\NewDocumentCommand\ap{}{\ensuremath{\mathbin{\circledast}}}
\NewDocumentCommand\pfa{}{\ensuremath{\mathbin{\stackrel{\apl}{\ml{A}}}}}
\NewDocumentCommand\pfap{}{\ensuremath{\mathbin{\stackrel{\aplp}{\ml{A}}}}}
\NewDocumentCommand\conjd{}{\ensuremath{\mathbin{\&}}}

\usepackage{multicol}

\setcounter{secnumdepth}{3}

\title{Varieties of projective content:\\expressives continued\thanks{24.979: Topics in
    semantics\\\noindent\textit{Getting high: Scope, projection, and evaluation order}}}

\author[Patrick D. Elliott and Martin Hackl]{Patrick~D. Elliott \& Martin Hackl}

\addbibresource[location=remote]{/home/patrl/repos/bibliography/elliott_mybib.bib}

\lingset{
  belowexskip=0pt,
  aboveglftskip=0pt,
  belowglpreambleskip=0pt,
  belowpreambleskip=0pt,
  interpartskip=0pt,
  extraglskip=0pt,
  Everyex={\parskip=0pt}
}

\usepackage{float}


% \usepackage{booktabs} % book-quality tables
% \usepackage{units}    % non-stacked fractions and better unit spacing
% \usepackage{lipsum}   % filler text
% \usepackage{fancyvrb} % extended verbatim environments
%   \fvset{fontsize=\normalsize}% default font size for fancy-verbatim environments

% % Standardize command font styles and environments
% \newcommand{\doccmd}[1]{\texttt{\textbackslash#1}}% command name -- adds backslash automatically
% \newcommand{\docopt}[1]{\ensuremath{\langle}\textrm{\textit{#1}}\ensuremath{\rangle}}% optional command argument
% \newcommand{\docarg}[1]{\textrm{\textit{#1}}}% (required) command argument
% \newcommand{\docenv}[1]{\textsf{#1}}% environment name
% \newcommand{\docpkg}[1]{\texttt{#1}}% package name
% \newcommand{\doccls}[1]{\texttt{#1}}% document class name
% \newcommand{\docclsopt}[1]{\texttt{#1}}% document class option name
% \newenvironment{docspec}{\begin{quote}\noindent}{\end{quote}}% command specification environment

\begin{document}

\maketitle% this prints the handout title, author, and date

\section{Non-local readings}

In the examples we've analyzed so far, the expressive adjective composes directly with the individual towards which the expressive attitude is directed. Surface compositionality can therefore be straightforwardly achieved.

\citet{gutzmann2019chap4} argues extensively that \acp{ea} give rise to what he calls \textit{non-local readings}. I'll take his empirical claims to be essentially correct -- the questions we'll be asking here will be \textit{why} and \textit{how}.

We've actually already seen many examples of non-local readings.

\ex
The [\hl{frakking} cat] is being affectionate for once.\hfill$\sad (ιx[\ml{cat} x])$\label{cat1}
\xe

(\ref{cat1}) can convey that the speaker has a negative attitude towards whatever \textit{the cat} refers to, despite the fact that \textit{frakking} takes as its sister just the {\sc np} cat.

Importantly, (\ref{cat1}) is compatible with (i) the speaker having a positive attitude towards the situation, and (ii) the speaker having a positive attitude towards cats in general.

Similarly, the following examples can convey that the speaker has a negative attitude towards \textit{the fact that the cat peed on the couch}

\ex~
The \hl{frakking} cat (which I love) is peeing on my favorite couch.\hfill$\sad p$
\xe

\ex~
The cat is peeing on my favorite \hl{frakking} couch.$\sad p$
\xe

\section{Gutzmann's {\sc agree}-based account}

In order to account for non-local readings, \citet{gutzmann2019chap4} claims that \acp{ea} come with an uninterpretable expressive feature, and the heads of constituents which be the target of the expressive attitude come with an unvalued, interpretable expressive feature.

\begin{figure}
  \centering
  \caption{Gutzmann's {\sc agree}-based account}\label{fig:agree}
  \begin{forest}
    [{DP}
      [{D} [{the\\{[iEx:\_\_]}}]]
      [{NP}
        [{AP} [{A\\frakking\\{[uEx:$\sad$]}}]]
        [{NP} [{dog},roof]]
      ]
    ]
  \end{forest}
  %
  \hspace{5em}
  %
    \begin{forest}
    [{DP}
      [{D} [{the\\{[iEx:$\sad$]}}]]
      [{NP}
        [{AP} [{A\\frakking\\{\st{[uEx:$\sad$]}}}]]
        [{NP} [{dog},roof]]
      ]
    ]
  \end{forest}
\end{figure}

The feature on \textit{frakking} values the feature on \textit{the} via upwards \textsc{agree}, and the uninterpretable feature is deleted. This is illustrated in figure \ref{fig:agree}.

Some (obvious) objections:

\begin{itemize}

    \item Find me a language with some overt realization of expressive agreement!

    \item The syntactic restrictions on non-local readings seem to pattern with restrictions on scope (as we'll see later) -- the agree based account is missing a generalization.

    \item Nothing insightful to say about the interaction between expressive adjectives and quantificational determiners.

\end{itemize}

Instead, I'll pursue a scope-based account of non-local readings, using continuations.\sidenote{This material is based on \cite{elliott-fuck}.}

\section{Scope via continuations -- a recap}

Scopal meanings (i.e., expressions that take a scope argument $k$) can be abbreviated using tower notation, as we've seen in previous classes, and which we'll be making use of in what follows:

\ex Tower notation (def.)\\
$\semtower{f []}{x} ≔ λ k . f (k x)$
\xe

As well as scopal meanings, scopal types can be abbreviated using tower notation as follows:

\ex~ Tower types (def.)\\
$\typetower{\type{r}}{\type{a}} ≔ \type{(a → r) → r}$
\xe

N.b. the type-shifters we've been using to compose scopal meanings don't presuppose \textit{anything} about the return type $\type{r}$.

\ex
\textit{lift} (def.)\\
$a^{↑} ≔ \semtower{[]}{a}$\hfill$(↑):\type{a → \typetower{r}{a}}$
\xe


\ex~
\acf{sfa} (def.)\\
$\semtower{f []}{x} \ml{S} \semtower{g []}{y} ≔
\semtower{f (g [])}{x \ml{A} y}$\hfill$\ml{S}:\begin{aligned}[t]
  &\type{\typetower{r}{a → b} → \typetower{r}{a} →
    \typetower{r}{b}}\\
  &\type{\typetower{r}{a} → \typetower{r}{a → b} →
    \typetower{r}{b}}
  \end{aligned}$
\xe

When discussing quantificational scope, we assumed that the return type was $\type{t}$, e.g.:

\ex
$\eval{everyone} ≔ \semtower{∀ x[]}{x}$\hfill$\type{\typetower{t}{e}}$
\xe

In the following, in order to model expressive scope, we'll assume that the return type is a \textit{fancy} type, namely $\type{e · t}$.


\subsection{Lifting multidimensional values into scope-takers}

We can now recast our old meaning for \textit{frakking} as an identity function with an expressive side-effect:

\ex
$\ml{frakking}_{S} ≔ \semtower{\ml{frakking} []}{id}$\hfill$\type{\semtower{e · t}{a → a}}$
\xe

It might be useful to consider the de-sugared (flat) definition:

\ex
$\ml{frakking}_{S} ≔ λ k . \ml{frakking} (k id)$
\xe

$\ml{frakking}_{S}$ encodes two meaning components:

\begin{itemize}

  \item It contributes an identity function locally, and

  \item waits for a fancy individual in order to evaluate its scope.

\end{itemize}

This generalizes our non-scopal treatment of \acp{ea}, as illustrated below. Note that the definition of expressive \textit{lower} doesn't use the identity functional, but rather $ρ$. Looking at the type of expressive lower should tell you why.

\ex Expressive lower (def.)
$m^{↓} ≔ m ρ$\hfill$↓:\type{\semtower{a · t}{a} → a · t}$
\xe

Here's an example of an expressive adjective composing with a proper name via \ac{sfa}. The result is immediately lowered.

\begin{figure}
\centering
\caption{\enquote{fracking Starbuck}}
\begin{forest}
  [{$\ml{starbuck} · \sad \ml{starbuck}$}
    [{$\semtower{\ml{frakking} []}{\ml{starbuck}}$}
      [{$\semtower{\ml{frakking} []}{id}$\\frakking$_{S}$}]
      [{$\semtower{[]}{\ml{starbuck}}$\\Starbuck$^{↑}$}]
    ]
  ]
\end{forest}
\end{figure}

DP-level readings are accounted for by assuming that expressive lower is \textit{delayed}, as shown in figure \ref{fig:dp-level}.

\begin{figure}
  \centering
  \caption{\enquote{The fracking cat}}\label{fig:dp-level}
  \begin{forest}
    [{$ιx[\ml{cat} x] · \sad (ιx[\ml{cat} x])$}
    [{$\semtower{\ml{frakking} []}{ιx[\ml{cat} x]}$}
      [{$\semtower{[]}{λ P . ιx[P x]}$\\the$^{↑}$}]
      [{$\semtower{\ml{frakking} []}{λ x . \ml{dog} x}$}
        [{$\semtower{\ml{frakking} []}{id}$\\frakking$_{S}$}]
        [{$\semtower{[]}{λ x . \ml{cat} x}$\\cat$^{↑}$}]
      ]
    ]
    ]
  \end{forest}
\end{figure}

One way of accounting for clausal readings without positing a polysemous entry for the expressive adjective is to invoke a proposition-to-individual shift. This is sketched out in figure \ref{fig:clausal}.\semtower{Perhaps a more natural approach is to adopt an ontology with \textit{events}, and treat the event as the target of the expressive attitude. This will ultimately be a possible route, but first we need to say something about how expressive adjectives interact with existential quantification.}

\begin{figure}
\centering
\caption{\enquote{The frakking cat peed outside.}}\label{fig:clausal}
\begin{forest}
  [{$(\ml{peed-outside} ιx[\ml{cat} x])^{∩} · \sad (\ml{peed-outside} ιx[\ml{cat} x])^{∩}$}
  [{$\semtower{\ml{frakking} []}{(\ml{peed-outside} ιx[\ml{cat} x])^{∩}}$}
    [{$∩^{↑}$}]
    [{$\semtower{\ml{frakking} []}{\ml{peed-outside} ιx[\ml{cat} x]}$}
      [{$\semtower{\ml{frakking} []}{ιx[\ml{cat} x]}$} [{the frakking cat},roof]]
      [{$\semtower{[]}{λ x . \ml{peed-outside} x}$} [{peed outside$^{↑}$},roof]]
    ]
 ]]
\end{forest}
\end{figure}

\subsection{Expressive adjectives and scope islands}

\begin{description}

    \item[Conjecture] so-called \enquote{non-local readings} of \acp{ea} are a scopal phenomenon.

    \item[Prediction] Non-local readings of \acp{ea} should be sensitive to scope islands.

\end{description}

\citet{gutzmann2019chap4} provides extensive argumentation that non-local readings of \acp{ea} are subject to syntactic restrictions -- they are sensitive to syntactic islands such as relative clauses, but crucially also cannot extend out of finite clauses, just like other scope-takers.

\ex
Peter said [that the dog ate the frakking cake].\\
\cmark $\sad (\ml{the dog at the cake})$\\
\cmark $\sad (\ml{the cake})$\\
\xmark $\sad (\ml{Peter said that the dog ate the cake})$\\
\xmark $\sad \ml{Peter}$
\xe

\ex~
The dog that ate the frakking cake is hungry.\\
\cmark $\sad (\ml{the dog ate the cake})$\\
\cmark $\sad (\ml{the cake})$\\
\xmark $\sad (\ml{The dog that ate the cake is hungry})$\\
\xmark $\sad (\ml{The dog that ate the cake})$\\
\xe

The sensitivity of \acp{ea} to scope islands falls out as a \textit{prediction} of the semantics we assigned them.

Consider the semantics of an unevaluated relative clause with an expressive side-effect:

\ex
$\eval{[that ate the frakking cake]} = \semtower{\ml{fracking} []}{λ y . y \ml{ate the cake}}$\hfill$\type{\semtower{e · t}{e → t}}$
\xe

The scope of the expressive cannot be evaluated since the bottom of the tower isn't (and can't be shifted to) type $\type{e}$.

The scope of the expressive must therefore be evaluated \textit{inside} of the relative clause.

One thing that's important to note -- expressive side-effects \textit{once evaluated} are predicted to survive through scope islands.

To see why, consider the semantics of an \textit{evaluated} relative clause with expressive side effects:

\ex
$\eval{[that ate the frakking cake]} = (λ y . y \ml{ate} ιx[\ml{cake} x]) · \sad (ιx[\ml{cake} x])$
\xe

The evaluated relative clause can be \textit{re-lifted} into an expressive scope-taker via expressive bind, and composition can continue.

\ex Expressive bind (def.)\\
$(x · p)^{⋆} ≔ λ k . (k x)^{\ml{A}} · ((k x)^{\ml{E}} ∧ p)$\hfill$⋆:\type{a · t → (a → b · t) → b · t}$
\xe

\ex~
$\eval{that ate the frakking cat}^{⋆} = \semtower{(id · \sad (ιx[\ml{cake} x])) ⊛ []}{λ y . y \ml{ate} ιx[\ml{cake} x]}$\hfill$\type{\semtower{\type{b · t}}{\type{e → t}}}$
\xe

\subsection{Quantification, binding, and expressives}

When uttered by a speaker who likes cats, the following example can express a negative attitude towards whichever cat happens to be being affectionate -- the resolution of the expressive attitude is therefore \textit{indeterminate}.

A first crack at approximating the reading we're interested in is given below:

\ex
A frakking cat is being affectionate for once.\hfill\xmark $∃x[\sad x]$
\xe

This isn't right -- it would fail to guarantee that the target of the expressive attitude is the same as the cat being affectionate.

Rather, it seems like we want the existential quantifier to take scope over \textit{both} the descriptive and the expressive content. Something like: $∃x[(\ml{cat} x ∧ \ml{affectionate} x) · \sad x]$. It's not clear how to accomplish this compositionally, however.

By way of contrast:

\ex
Every fucking cat is being affectionate for once.\hfill$∀x[\sad x]$
\xe

In order to capture the interaction between expressives and indefinites, we'll need to fold in alternatives.


\printbibliography

\end{document}

% LocalWords:  definedness frakking
